\section{Proof Reuse Over Time}
\label{sec:key1}

We can view proof refactoring and repair---like the refactoring we just saw---as a form of 
\textit{proof reuse}~\cite{Ringer2019, felty1994generalization, caplan1995logical, pons2000generalization, johnsen2004theorem}, % TODO consider citation list
or reusing proofs about one specification to derive proofs about another specification.
The difference from standard proof reuse is that, for proof refactoring and repair, the specification about which
we are reusing proofs ceases to exist. In other words:

\begin{quote}
\textbf{Insight 1}:
Proof refactoring (Section~\ref{sec:refactoring}) and repair (Section~\ref{sec:repair}) are both just 
proof reuse %
over time. The key to supporting both is to build a generic proof reuse
tool that can handle the additional challenges imposed by the reuse occuring \textit{over time} (Section~\ref{sec:time}). 
\end{quote}

\subsection{Refactoring as Reuse across Equivalences}
\label{sec:refactoring}

The example with \lstinline{list} was a simple refactoring, or a semantics-preserving change.
Thus, \toolname used the old proof of \lstinline{rev_app_distr} defined over \lstinline{Old.list}
to generate the new proof of \lstinline{rev_app_distr} defined over \lstinline{New.list}.
Since this was a refactoring, this required no additional information from the user, and the resulting functions
and proofs behaved exactly the same way.

\begin{figure}
\begin{minipage}{0.48\textwidth}
\begin{lstlisting}
f T (l : Old.list T) : New.list T :=
  Old.list_rect
    T
    (fun (l : Old.list T) => New.list T)
    New.nil
    (fun (t : T) _ (IHl : New.list T) =>
      New.cons T t IHl)
    l.

Lemma section:
  $\forall$ T (l : Old.list T), g T (f T l) = l.
Proof.
  intros T l. symmetry. induction l.
  - reflexivity.
  - simpl. rewrite <- IHl. reflexivity.
Defined.
\end{lstlisting}
\end{minipage}
\hfill
\begin{minipage}{0.48\textwidth}
\begin{lstlisting}
g T (l : New.list T) : Old.list T :=
  New.list_rect
    T
    (fun (l : New.list T) => Old.list T)
    (fun (t : T) _ (IHl : Old.list T) =>
      Old.cons T t IHl)
    Old.nil
    l.

Lemma retraction:
  $\forall$ T (l : New.list T), f T (g T l) = l.
Proof.
  intros T l. symmetry. induction l.
  - simpl. rewrite <- IHl. reflexivity.
  - reflexivity.
Defined.
\end{lstlisting}
\end{minipage}
\caption{Two functions between \lstinline{Old.list} and \lstinline{New.list} (top) that form an equivalence (bottom).}
\label{fig:equivalence}
\end{figure}

More formally, \toolname was able to find a pair of two functions that map between \lstinline{Old.list}
and \lstinline{New.list} and prove that they are mutual inverses (Figure~\ref{fig:equivalence}).
\toolname could then use that information to configure its proof term transformation to refactor
functions and proofs about \lstinline{Old.list} to functons and proofs about \lstinline{New.list} with
the same exact behavior.

Whenever there exist two functions between two types that are mutual inverses,
there is a \textit{type equivalence}~\cite{univalent2013homotopy} between them, denoted $\simeq$:

\begin{lstlisting}
Old.list $\simeq$ New.list
\end{lstlisting}
Refactoring corresponds to proof reuse across these equivalences.

\subsection{Repair as Reuse across Refinements}
\label{sec:repair}

Repair, in contrast, refers to changes that are \textit{not} semantics-preserving---they require additional information
from the proof engineer.
Repair, too, corresponds to proof reuse across equivalences, but not between the old and new types.
Rather, repair is the special case where one or both of the types is equivalent to a \textit{refinement}
of the other type.
This means that it is possible to reduce repair to refactoring, as long as the proof engineer supplies the additional information.

\begin{figure}
\begin{minipage}{0.40\textwidth}
   \lstinputlisting[firstline=1, lastline=4]{listtovect.tex}
\end{minipage}
\hfill
\begin{minipage}{0.58\textwidth}
   \lstinputlisting[firstline=6, lastline=9]{listtovect.tex}
\end{minipage}
\caption{A vector (right) is a list (left) indexed by its length.}
\label{fig:listtovect}
\end{figure}

Consider an example from \textsc{Devoid}: changing a list to a length-indexed vector (Figure~\ref{fig:natrepair}).
Lists and vectors are not equivalent.
To fix our functions about lists and get equivalent functions about vectors at a particular length, we need to prove
an invariant about the length.
To fix our proofs about those functions, we need to show how our invariants relate to one another.
In other words, we need to port our functions and proofs across this equivalence:

\begin{lstlisting}
{ l : list T | length l = n } $\simeq$ vector T n.
\end{lstlisting}

Section~\ref{sec:search} shows how we do this in detail with \toolname.
Essentially, the additional proof about the length of the list manifests as a proof obligation for the user.
The rest is straightforward.

\subsection{A Tool for Proof Reuse Over Time}
\label{sec:time}

The problem of proof reuse across equivalences is known as \textit{transport}. % TODO cite
Thus, any tool that can refactor or repair proofs implements transport across the classes
of equivalences that it supports.
But we cannot use just any transport method: we must transport functions and proofs \textit{over time}.

That proof refactoring and repair are reuse \textit{over time} dictates that proofs
must no longer refer in any way to the old specification, since the old specification no longer exists.
This means that most transport methods % TODO cite
are not immediately useful for refactoring and repair.

The goal of a proof refactoring and repair tool is, in essence, to
define a transport method over a broad set of changes that
removes references to the old specification, rather than converting back and forth
like standard transport methods.
For example, the refactored proof of \lstinline{rev_app_distr} and the updated application function \lstinline{app}
from Section~\ref{sec:overview} refers only to \lstinline{New.list} and not to \lstinline{Old.list}.
It inducts directly over \lstinline{New.list} and never, at any point, converts to an \lstinline{Old.list}.
Thus, we need \lstinline{Old.list} around only until we call the \lstinline{Repair} command;
after that, we can remove the type and all of the functions and proofs about it, and replace them all with
our functions and proofs about \lstinline{New.list}.

\toolname accomplishes this using a configurable proof term transformation (Section~\ref{sec:key2}).
The proof term transformation implements transport across equivalences, but in a way that removes
references to the old type.
The configuration corresponds to a particular equivalence, and tells the proof term transformation how to transform
constructors, eliminators, identity, and equalities.
Section~\ref{sec:search} includes many example configurations as applied to real proof refactoring and repair scenarios.


