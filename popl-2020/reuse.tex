\section{Proof Reuse Over Time}
\label{sec:key1}

We can view proof refactoring and repair---like the refactoring we just saw---as a form of 
\textit{proof reuse}~\cite{Ringer2019, felty1994generalization, caplan1995logical, pons2000generalization, johnsen2004theorem}, % TODO consider citation list
or reusing proofs about one specification to derive proofs about another specification.
The difference from standard proof reuse is that, for proof refactoring and repair, the specification about which
we are reusing proofs ceases to exist. In other words:

\begin{quote}
\textbf{Insight 1}:
Proof refactoring (Section~\ref{sec:refactoring}) and repair (Section~\ref{sec:repair}) are both just 
proof reuse %
over time. The key to supporting both is to build a generic proof reuse
tool that can handle the additional challenges imposed by the reuse occuring \textit{over time} (Section~\ref{sec:time}). 
\end{quote}

\subsection{Refactoring as Reuse across Equivalences}
\label{sec:refactoring}

The example with \lstinline{list} was a simple refactoring, or a semantics-preserving change.
Thus, \toolname used the old proof of \lstinline{rev_app_distr} defined over \lstinline{Old.list}
to generate the new proof of \lstinline{rev_app_distr} defined over \lstinline{New.list}.
Since this was a refactoring, this required no additional information from the user, and the resulting functions
and proofs behaved exactly the same way.

\begin{figure}
\begin{minipage}{0.46\textwidth}
Definition f (T : Type) (l : Old.list T) : New.list T :=
  Old.list_rect
    T
    (fun (l : Old.list T) => New.list T)
    New.nil
    (fun (t : T) (l : Old.list T) (IHl : New.list T) =>
      New.cons T t IHl)
    l.

Lemma section:
  forall (T : Type) (l : list T), g T (f T l) = l.
Proof.
  intros T l. symmetry. induction l as [|t l0 H].
  - reflexivity.
  - simpl. rewrite <- H. reflexivity.
Defined.
\end{minipage}
\hfill
\begin{minipage}{0.46\textwidth}
Definition g (T : Type) (l : New.list T) : Old.list T :=
  New.list_rect
    T
    (fun (l : New.list T) => Old.list T)
    (fun (t : T) (l : New.list T) (IHl : Old.list T) =>
      Old.cons T t IHl)
    Old.nil
    l.

Lemma retraction:
  forall T l, f T (g T l) = l.
Proof.
  intros T l. symmetry. induction l as [t l0 H|].
  - simpl. rewrite <- H. reflexivity.
  - reflexivity.
Defined.
\end{minipage}
\caption{Two functions between \lstinline{Old.list} and \lstinline{New.list} (top) that form an equivalence (bottom).}
\label{fig:equivalence}
\end{figure}

More formally, \toolname was able to find a pair of two functions that map between \lstinline{Old.list}
and \lstinline{New.list} and prove that they are mutual inverses (Figure~\ref{fig:equivalence}).
\toolname could then use that information to configure its proof term transformation to refactor
functions and proofs about \lstinline{Old.list} to functons and proofs about \lstinline{New.list} with
the same exact behavior.

Whenever there exist two functions between two types that are mutual inverses,
there is a \textit{type equivalence}~\cite{univalent2013homotopy} between them, denoted $\simeq$:

\begin{lstlisting}
Old.list $\simeq$ New.list
\end{lstlisting}
Refactoring corresponds to proof reuse across these equivalences.

\subsection{Repair as Reuse across Refinements}

Repair, in contrast, refers to changes that are \textit{not} semantics-preserving---they require additional information
from the user.
Repair, too, corresponds to proof reuse across equivalences, but not between the old and new types.
Rather, repair is the special case where one or both of the types is equivalent to a \textit{refinement}
of the other type.

\begin{figure}
\begin{minipage}{0.46\textwidth}
   \lstinputlisting[firstline=1, lastline=3]{nat.tex}
\end{minipage}
\hfill
\begin{minipage}{0.46\textwidth}
   \lstinputlisting[firstline=5, lastline=7]{nat.tex}
\end{minipage}
\caption{The updated \lstinline{nat} (right) is the old \lstinline{nat} (left) starting at 1 instead of 0 (\codediff{orange}).}
\label{fig:natrepair}
\end{figure}

Consider, for example, changing the interpretation of the natural numbers \lstinline{nat} to start at 1 instead of 0 (Figure~\ref{fig:natrepair}).
The data structure does not change here either than a name, but our \textit{interpretation} of it does change.
We could choose infinitely many equivalences here, most simply:

\begin{lstlisting}
Old.nat $\simeq$ New.nat
\end{lstlisting}
But this particular equivalence would not do what we want: it would refactor the functions and proofs to rename 0 to 1, rather than
actually changing the behavior of functions and proofs appropriately.
If we want our new functions and proofs to actually reason about the natural numbers starting at 1, we want some additional information.
We can show an equivalence between \lstinline{New.nat} and a \textit{refinement} of \lstinline{Old.nat}:

\begin{figure}
\begin{minipage}{0.46\textwidth}
Definition f (s : { n : Old.nat | eqb n 0 = false}) : New.nat :=
  Old.nat_rect
    (fun (n : Old.nat) => eqb n 0 = false -> New.nat)
    (fun (H : eqb 0 0 = false) =>
       False_rect _
         (eq_ind (eqb 0 0) (bool_rect _ True False) I false H))
    (fun (n : nat) _ _ =>
      Old.nat_rect
        (fun (n : Old.nat) => New.nat)
        1
        (fun (n : Old.nat) (IH : New.nat) => New.S IH)
        n)
    ($\pi_l$ s)
    ($\pi_r$ s).

Lemma section :
  forall s, g (f s) = s.
Proof.
  intros s. induction s as [n H]. induction n.
  - discriminate.
  - induction n; simpl; apply eq_exist_uncurried. 
    + exists eq_refl. symmetry. apply UIP_refl_bool.
    + exists (f_equal S (proj1_sig_eq (IHn H))). 
      eapply eq_trans; eauto using UIP_refl_bool.
Defined.
\end{minipage}
\hfill
\begin{minipage}{0.46\textwidth}
Definition g (n : New.nat) : { n : Old.nat | eqb n 0 = false} :=
  New.nat_rect
    (fun (n : New.nat) => { n : Old.nat | eqb n 0 = false})
    ($\exists$ (Old.S 0) eq_refl)
    (fun (n : New.nat) (IH : { n : Old.nat | eqb n 0 = false}) =>
      $\exists$ (Old.S ($\pi_l$ IH)) eq_refl)
    n.








Lemma retraction :
  forall n, f (g n) = n.
Proof.
  intros n. induction n.
  - auto.
  - rewrite <- IHn at 2. simpl. unfold f. simpl. remember (g n) as s. induction s.
    induction x; try discriminate; auto.
Defined.
\end{minipage}
\caption{Two functions between a refinement of \lstinline{Old.nat} and \lstinline{New.nat} (top) that form an equivalence (bottom).}
\label{fig:equivalence2}
\end{figure}

\begin{lstlisting}
{ n : Old.nat | eqb n 0 = false } $\simeq$ New.nat 
\end{lstlisting}
One possible proof of this equivalence is in Figure~\ref{fig:equivalence2}.
The gist of this equivalence is that to repair proofs about \lstinline{Old.nat} functions to proofs about \lstinline{New.nat} functions with the desired behavior,
we need the additional restriction that those numbers are nonzero. 
This restriction manifests as a proof obligation for the proof engineer.
\toolname cannot prove this particular equivalence automatically, but it can
port functions and proofs across it via the manual configuration workflow. % TODO check

\subsection{A Tool for Proof Reuse Over Time}
\label{sec:time}

The problem of proof reuse across equivalences is known as \textit{transport}. % TODO cite
Thus, any tool that can refactor or repair proofs implements transport across the classes
of equivalences that it supports.
But we cannot use just any transport method: we must transport functions and proofs \textit{over time}.

That proof refactoring and repair are reuse \textit{over time} dictates that proofs
must no longer refer in any way to the old specification, since the old specification no longer exists.
This means that most transport methods % TODO cite
are not immediately useful for refactoring and repair.

The goal of a proof refactoring and repair tool is, in essence, to
define a transport method over a broad set of changes that
removes references to the old specification, rather than converting back and forth
like standard transport methods.
For example, the refactored proof of \lstinline{rev_app_distr} and the updated application function \lstinline{app}
from Section~\ref{sec:overview} refers only to \lstinline{New.list} and not to \lstinline{Old.list}.
It inducts directly over \lstinline{New.list} and never, at any point, converts to an \lstinline{Old.list}.
Thus, we need \lstinline{Old.list} around only until we call the \lstinline{Repair} command;
after that, we can remove the type and all of the functions and proofs about it, and replace them all with
our functions and proofs about \lstinline{New.list}.

\toolname accomplishes this using a configurable proof term transformation (Section~\ref{sec:meat}).
The proof term transformation implements transport across equivalences, but in a way that removes
references to the old type.
The configuration corresponds to a particular equivalence, and tells the proof term transformation how to transform
constructors, eliminators, identity, and equalities.
Section~\ref{sec:search} includes many example configurations as applied to real proof refactoring and repair scenarios.


