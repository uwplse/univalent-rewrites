\section{Problem Definition}
\label{sec:key1}

Proof refactoring and proof repair are similar problems: both refer to adapting proofs in response to changes
in specifications. Refactoring refers to semantics-preserving changes, and repair refers to non-semantics-preserving
changes. We can view both as a form of 
\textit{proof reuse}~\cite{Ringer2019, felty1994generalization, caplan1995logical, pons2000generalization, johnsen2004theorem}, % TODO consider citation list
or reusing proofs about one specification to derive proofs about another specification.
The difference is that in standard proof reuse, both specifications continue to exist, whereas with proof refactoring and repair,
one of the specifications does not.

\begin{quote}
\textbf{Insight 1}:
Proof repair is a form of proof reuse (Section~\ref{sec:repair}), with the additional
challenge that the specification about which
we are reusing proofs ceases to exist.
The key to supporting proof repair is to build a proof reuse
tool that can handle that additional challenge (Section~\ref{sec:time}).
\end{quote}

\subsection{Refactoring as Reuse across Equivalences}
\label{sec:refactoring}

The example with \lstinline{list} was a simple refactoring, or a semantics-preserving change.
Thus, \toolname used the old proof of \lstinline{rev_app_distr} defined over \lstinline{Old.list}
to generate the new proof of \lstinline{rev_app_distr} defined over \lstinline{New.list}.
Since this was a refactoring, this required no additional information from the user, and the resulting functions
and proofs behaved exactly the same way.

\begin{figure}
\begin{minipage}{0.48\textwidth}
\begin{lstlisting}
(@\codeauto{f}@) T (l : Old.list T) : New.list T :=
  Old.list_rect T
    (fun (l : Old.list T) => New.list T)
    New.nil
    (fun (t : T) _ (IHl : New.list T) =>
      New.cons T t IHl)
    l.

Lemma (@\codeauto{section}@):
  $\forall$ T (l : Old.list T), (@\codeauto{g}@) T ((@\codeauto{f}@) T l) = l.
Proof.
  intros T l. symmetry. induction l.
  - reflexivity.
  - simpl. rewrite <- IHl. reflexivity.
Defined.
\end{lstlisting}
\end{minipage}
\hfill
\begin{minipage}{0.48\textwidth}
\begin{lstlisting}
(@\codeauto{g}@) T (l : New.list T) : Old.list T :=
  New.list_rect T
    (fun (l : New.list T) => Old.list T)
    (fun (t : T) _ (IHl : Old.list T) =>
      Old.cons T t IHl)
    Old.nil
    l.

Lemma (@\codeauto{retraction}@):
  $\forall$ T (l : New.list T), (@\codeauto{f}@) T ((@\codeauto{g}@) T l) = l.
Proof.
  intros T l. symmetry. induction l.
  - simpl. rewrite <- IHl. reflexivity.
  - reflexivity.
Defined.
\end{lstlisting}
\end{minipage}
\caption{Two functions between \lstinline{Old.list} and \lstinline{New.list} (top) that form an equivalence (bottom).}
\label{fig:equivalence}
\end{figure}

More formally, \toolname was able to find a pair of two functions that map between \lstinline{Old.list}
and \lstinline{New.list} and prove that they are mutual inverses, generating the functions and tactic
proofs in Figure~\ref{fig:equivalence} automatically.
\toolname could then use that information to configure its proof term transformation to refactor
functions and proofs about \lstinline{Old.list} to functons and proofs about \lstinline{New.list} with
the same exact behavior.

Whenever there exist two functions between two types that are mutual inverses,
there is a \textit{type equivalence}~\cite{univalent2013homotopy} between them, denoted $\simeq$:

\begin{lstlisting}
Old.list $\simeq$ New.list
\end{lstlisting}
Refactoring corresponds to proof reuse across these equivalences.

\subsection{Repair as Reuse across Refinements}
\label{sec:repair}

Repair, in contrast, refers to changes that are \textit{not} semantics-preserving---they require additional information
from the proof engineer.
Repair, too, corresponds to proof reuse across equivalences, but not between the old and new types.
Rather, repair is the special case where the equivalence that we port functions and proofs along
is an equivalence between \textit{refinements}.
This means that it is possible to reduce repair to refactoring, as long as the proof engineer supplies the additional information in
order to construct proofs about the refinement.

\begin{figure}
\begin{minipage}{0.40\textwidth}
   \lstinputlisting[firstline=1, lastline=4]{listtovect.tex}
\end{minipage}
\hfill
\begin{minipage}{0.58\textwidth}
   \lstinputlisting[firstline=6, lastline=9]{listtovect.tex}
\end{minipage}
\caption{A vector (right) is a list (left) indexed by its length.}
\label{fig:listtovect}
\end{figure}

Consider an example from \textsc{Devoid}: changing a list to a length-indexed vector (Figure~\ref{fig:listtovect}).
\textsc{Devoid} implemented a refactoring between lists and \textit{vectors of some length}, since:

\begin{lstlisting}
list T $\simeq$ $\Sigma$ (n : nat) . vector T n.
\end{lstlisting}
This was enough to automatically refactor a lemma about lists:

\begin{lstlisting}
$\forall$ {A B} (l1 : list A) (l2 : list B),
  zip_with pair l1 l2 = zip l1 l2.
\end{lstlisting}
to a lemma about vectors of some length:

\begin{lstlisting}
$\forall$ {A B} (l1 : (@\codediff{$\Sigma$(n : nat).vector A n}@)) (l2 : (@\codediff{$\Sigma$(n : nat).vector B n}@)),
  zip_with pair l1 l2 = zip l1 l2.
\end{lstlisting}
recursively refactoring dependencies \lstinline{zip} and \lstinline{zip_with}.
It was not enough, however, to help the proof engineer get from that to a proof about vectors \textit{at a particular index}:

\begin{lstlisting}
$\forall$ {A B} (@\codediff{n}@) (l1 : (@\codediff{vector A n}@)) (l2 : (@\codediff{vector B n}@)),
  zip_with pair (@\codediff{n}@) l1 l2 = zip (@\codediff{n}@) l1 l2.
\end{lstlisting}

More desirable is a repair that takes us from lists \textit{at a particular length} to vectors of that length:

\begin{lstlisting}
{ l : list T | length l = n } $\simeq$ vector T n.
\end{lstlisting}
\toolname can handle this.
This is a repair because our old specification \lstinline{list} shows up as part of a refinement with an additional proof obligation
about its length. Practically, when we change our specifications to refer to \lstinline{vector} instead of \lstinline{list},
to fix our functions and proofs, we must additionally prove invariants about the lengths of our lists.
\toolname makes it easy to separate out that proof obligation, and then automates the rest.

\subsection{A Tool for Proof Refactoring and Repair}
\label{sec:time}

The problem of proof reuse across equivalences is known as \textit{transport}. % TODO cite
Thus, any tool that can refactor or repair proofs implements transport across the classes
of equivalences that it supports---and so can handle proof reuse as well.
But not every proof reuse tool or transport method can support proof refactoring and repair.
A proof refactoring and repair tool must produce proofs
that no longer refer in any way to the old specification, since the old specification no longer exists.

The goal of a proof refactoring and repair tool is, in essence, to
define a transport method over a broad set of changes that
removes references to the old specification, rather than converting back and forth
like standard transport methods.
For example, the refactored proof of \lstinline{rev_app_distr} and its dependencies
from Section~\ref{sec:overview} refer only to \lstinline{New.list} and not to \lstinline{Old.list}.
The proof inducts directly over \lstinline{New.list} and never, at any point, converts to an \lstinline{Old.list}.
Thus, we need \lstinline{Old.list} around only until we call the \lstinline{Repair} command;
after that, we can remove the type and all of the functions and proofs about it, and replace them all with
our functions and proofs about \lstinline{New.list}.

\toolname accomplishes this using a configurable proof term transformation (Section~\ref{sec:key2}).
The proof term transformation implements transport across equivalences, but in a way that removes
references to the old type.
The configuration corresponds to a particular equivalence, and tells the proof term transformation how to transform
constructors, eliminators, and equalities.
Section~\ref{sec:search} includes many example configurations as applied to real proof refactoring and repair scenarios.


