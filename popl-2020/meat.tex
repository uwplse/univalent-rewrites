\section{Proof Reuse Over Time}
\label{sec:meat}

We aimed to build a tool that can support both proof refactoring and proof repair.
Proof refactoring and proof repair are similar problems: both refer to adapting proofs in response to changes
in specifications. In Section~\ref{sec:overview}, for example, our specification was the theorem statement
for \lstinline{rev_app_distr}, which changed when our \lstinline{list} type changed.
In response, we had to refactor the proof.
This was a refactoring because the change was \textit{semantics-preserving}:
for any function or proof over the old version of \lstinline{list}, we could define a function or proof over the
new version of \lstinline{list} that behaved the same way, and that function or proof had the behavior that we wanted.

Repair, in contrast, refers to changes that are \textit{not} semantics-preserving.
One example might be changing a specification to use positive rather than natural numbers.
We could define a refactoring corresponding to this change by shifting every number by one,
but this would lead to functions with strange behavior.
It is likely more desirable to follow the lead of the Coq standard library:

\begin{lstlisting}
Nat2Pos.id : forall (n : nat), n <> 0 -> Pos.to_nat (Pos.of_nat n) = n.
\end{lstlisting}
In other words, to repair our proofs about \lstinline{nat} functions to proofs about \lstinline{positive} functions with the desired behavior,
we need the additional restriction that those numbers are nonzero. 
This restriction manifests as a proof obligation for the user.

In both of these cases, we can think of any tool that refactors or repairs our proofs as
\textit{reusing} the proofs about our old specification to derive proofs about our new specification,
either with no new information (in the case of refactoring) or with some additional proof obligations (in the case of repair).
This brings us to our first key insight:

\begin{quote}
\textbf{Key Insight 1}:
Proof refactoring and repair are both just 
proof reuse %~\cite{Ringer2019, felty1994generalization, caplan1995logical, pons2000generalization, johnsen2004theorem}
\textit{over time}. The key to supporting both is to build a generic proof reuse
tool that can handle the additional challenges imposed by the reuse occuring over time. 
\end{quote}
In particular, we can reduce the problems of proof refactoring and repair to proof reuse across \textit{type equivalences}~\cite{univalent2013homotopy},
or pairs of functions that map back and forth between two types and are mutual inverses.
Refactoring corresponds to porting a proof across a type equivalence between the old specification and the new specification,
while repair corresponds to porting a proof across a type equivalence between \textit{a refinement} of the old specification
and the new specification (or similarly in the opposite direction).
Note that there can be infinitely many equivalences corresponding to any change in specification,
so the choice in equivalence is to some degree an art that depends on what the user wants.
In the refactoring example, we use this equivalence:

\begin{lstlisting}
Coq.Init.Datatypes.list $\simeq$ list
\end{lstlisting}
while in the repair example, we use this equivalence:

\begin{lstlisting}
{ n : nat & n <> O } $\simeq$ positive
\end{lstlisting}

The problem of proof reuse across equivalences is known as \textit{transport}. % TODO cite
However, most transport methods % TODO cite
produce functions and proofs that still refer to the old specification.
The main challenge imposed be reuse occurring over time is that the refactored or repaired functions and proofs
may no longer refer in any way to the old specification, since the old specification no longer exists.
There is one exception that we are aware of: \textsc{DEVOID}~\cite{Ringer2019} defines a proof term transformation
that transports programs and proofs across equivalences for a particular class of changes.
This brings us to our second key insight:

\begin{quote}
\textbf{Key Insight 2}:
The proof term transformation from the \textsc{DEVOID} proof reuse tool can be generalized
to build such a generic proof reuse tool, and the result is configurable both by the developer and by the user.
\end{quote}
This generalized proof term transformation forms the core of \toolname.

Of course, as in \textsc{DEVOID}, such a proof term transformation produces a proof term,
while the proof engineer typically writes proof scripts made up of tactics.
Feeding the user proof terms does not make for a very good user experience.
This brings us to our final insight:

\begin{quote}
\textbf{Key Insight 3}: The transformed proof terms can then be translated back to tactics.
\end{quote} 

These insights led us to design \toolname as a configurable (Section~\ref{sec:art}) proof term transformation (Section~\ref{sec:transform})
combined with a Gallina to Ltac decompiler (Section~\ref{sec:meatdecompile}).
This gives us the \textbf{Configure}, \textbf{Transform}, \textbf{Decompile} we saw in Section~\ref{sec:overview}.

\subsection{Configure}
\label{sec:art}

The transformation is configurable by equivalence, but choosing the configuration (either by providing direct input or writing a 
search procedure for an entire class of equivalences) is a bit of an art: There can be infinitely many equivalences that correpond to a 
given change in specification, only some of which are useful. Furthermore, different configurations based on those equivalences
can be more or less useful than others. Finally, certain proof obligations for configuring the transformation can be tricky.
Thankfully, once that art is done, we know what it means for it to be correct.

The correctness criteria for the configuration relate \lstinline{DepConstr}, \lstinline{DepElim}, \lstinline{IdEta}, and \lstinline{RewEta}
in a particular way.
This goes back to the equivalence and equality thing from the previous section.
Formal thing here, intuition for what it means for the whole thing, explanation, and example.
Section~\ref{sec:search} shows some particular instantions of this and their applicability to real programs and proofs.

%Nicolas proved the first of these a while ago
%for the equivalence in the DEVOID ITP paper.\footnote{\url{https://github.com/CoqHott/univalent_parametricity/commit/7dc14e69942e6b3302fadaf5356f9a7e724b0f3c}}

Note about decidability of matching: when all of this is correct, what this means is that you \textit{can} always
run one of the transformation rules. But that doesn't mean you \textit{should}. Depends what the user wants,
and in some cases, would not terminate (refinement types, unpacking indexed types). Implementation section will
discuss how we actually decide which ones to run so user doesn't need to apply transport by hand over and over again,
and discussion section describes some cool ideas for doing this nicely with type-based search in the future.

\paragraph{Preserving Equivalence} Explanation and example.

\paragraph{Preserving Equality} Explanation and example.

\subsection{Transform}
\label{sec:transform}

Figure~\ref{fig:final} shows the configurable proof term transformation,
which is parameterized over two equivalent types (rule \lstinline{Equivalence}).
Most of it is standard with the exception of the four rules that correspond to the configuration:

\begin{enumerate}
\item Rules \lstinline{Dep-Constr} and \lstinline{Dep-Elim} for transforming constructors and eliminators 
\item Rules \lstinline{Id-Eta} and \lstinline{Rew-Eta} for transforming identity and equalities
\end{enumerate}
These transformations rules, taken together, both induce a particular equivalence and ensure that the transformation
preserves it and produces well-typed terms.
In short, (1) ensures the transformation preserves the \textit{equivalence} between our old and new specifications,
while (2) ensures that the transformation respects \textit{equalities} in the original proof.
The former of these just means that it implements transport, while the latter deals specifically
with the fact that we do not have univalence (which would state that equivalence is equivalent to equality),
so for the transformation to produce terms that type-check, we must transform equalities too.
This is a problem the marriage of univalence and parametricity notes as intractable in general---it turns out
to be easily describable, but there are many challenges to implemention which we will note in Section~\ref{sec:implementation}.

\paragraph{Constructors and Eliminators} Explanation, intuition, simple example.

\paragraph{Identity and Rewrites} Explanation, intuition, simple example.

\begin{figure}
\begin{mathpar}
\mprset{flushleft}
\small
\hfill\fbox{$\Gamma$ $\vdash$ $t$ $\Uparrow$ $t'$}\\

\inferrule[Dep-Elim]
  { \Gamma \vdash p_{a} \Uparrow p_b \\ \Gamma \vdash \vec{f_{a}}\phantom{l} \Uparrow \vec{f_{b}} }
  { \Gamma \vdash \mathrm{DepElim}(a,\ p_{a}) \vec{f_{a}} \Uparrow \mathrm{DepElim}(b,\ p_b) \vec{f_{b}} }

\inferrule[Dep-Constr]
{ \Gamma \vdash \vec{t}_{a} \Uparrow \vec{t}_{b} } %\\ TODO must we explicitly lift A to B if we want to handle parameters/indices?
{ \Gamma \vdash \mathrm{DepConstr}(j,\ A)\ \vec{t}_{a} \Uparrow \mathrm{DepConstr}(j,\ B)\ \vec{t}_{b}  }

\inferrule[Id-Eta]
  { \\ }
  { \Gamma \vdash \mathrm{IdEta}(A) \Uparrow \mathrm{IdEta}(B) }

\inferrule[Rew-Eta]
  { \Gamma \vdash c_A \Uparrow c_B \\ \Gamma \vdash q_A \Uparrow q_B \\ \Gamma \vdash e_A \Uparrow e_B }
  { \Gamma \vdash \mathrm{RewEta}(j, A, q_A, t_A) \Uparrow \mathrm{RewEta}(j, B, q_B, t_B) }

\inferrule[Equivalence]
  { \\ }
  { \Gamma \vdash A\ \Uparrow B }

\inferrule[Constr]
{ \Gamma \vdash T \Uparrow T' \\ \Gamma \vdash \vec{t} \Uparrow \vec{t'} }
{ \Gamma \vdash \mathrm{Constr}(j,\ T)\ \vec{t} \Uparrow \mathrm{Constr}(j,\ T')\ \vec{t'} }

\inferrule[Ind]
  { \Gamma \vdash T \Uparrow T' \\ \Gamma \vdash \vec{C} \Uparrow \vec{C'}  }
  { \Gamma \vdash \mathrm{Ind} (\mathit{Ty} : T) \vec{C} \Uparrow \mathrm{Ind} (\mathit{Ty} : T') \vec{C'} }

\inferrule[Elim] % TODO wait why do we have c here when it clearly refers to the term we eliminate over? um
  { \Gamma \vdash c \Uparrow c' \\ \Gamma \vdash Q \Uparrow Q' \\ \Gamma \vdash \vec{f} \Uparrow \vec{f'}}
  { \Gamma \vdash \mathrm{Elim}(c, Q) \vec{f} \Uparrow \mathrm{Elim}(c', Q') \vec{f'}  }

%% Application
\inferrule[App]
 { \Gamma \vdash f \Uparrow f' \\ \Gamma \vdash t \Uparrow t'}
 { \Gamma \vdash f t \Uparrow f' t' }

% Lamda
\inferrule[Lam]
  { \Gamma \vdash T \Uparrow T' \\ \Gamma,\ t : T \vdash b \Uparrow b' }
  {\Gamma \vdash \lambda (t : T).b \Uparrow \lambda (t : T').b'}

% Product
\inferrule[Prod]
  { \Gamma \vdash T \Uparrow T' \\ \Gamma,\ t : T \vdash b \Uparrow b' }
  {\Gamma \vdash \Pi (t : T).b \Uparrow \Pi (t : T').b'}
\end{mathpar}
\caption{Proof term transformation.}
\label{fig:final}
\end{figure}


%First we need that \lstinline{DepElim} over $A$ into \lstinline{DepConstr} over $B$ and \lstinline{DepElim} over $B$ into
%\lstinline{DepConstr} over $A$ form an equivalence between $A$ and $B$. When that's true, I think it should hold that \lstinline{DepElim} over $A$
%and \lstinline{DepElim} over $B$ are in univalent relation with one another. If not, then that's an extra condition.
%Finally, we need the transformation to preserve definitional equalities. Not sure about the general case, but for vectors and lists,
%we need:

%\begin{lstlisting}
%  $\forall$ A l (f : $\forall$ (l : sigT (Vector.t A)), l = l),
%    vect_dep_elim A (fun l => l = l) (f nil) (fun t s _ => f (cons t s)) l = f (id_eta l).
%\end{lstlisting}
%and:

%\begin{lstlisting}
%Definition elim_id (A : Type) (s : {H : nat & t A H}) :=
%  vect_dep_elim
%    A
%    (fun _ => {H : nat & t A H})
%    nil
%    (fun (h : A) _ IH =>
%      cons h IH)
%    s.

% $\forall$ A h s,
%    exists (H : cons h (elim_id A s) = elim_id A (cons h s)),
%      H = eq_refl.
%\end{lstlisting}
%More generally, for each constructor index $j$, define:

%\begin{lstlisting}
%  eqc (j, B) (f : $\forall$ b : B, b = b) :=
%    fun ... (* TODO get the hypos from the type of the eliminator *) =>
%      f (DepConstr (j, B)) (* TODO args *)%%

  %elim_id := (* TODO *)
%\end{lstlisting}
%Then we need:

%\begin{enumerate}
%\item $\forall b f, \mathrm{DepElim}(b,\ p_{b}) \{\mathrm{eqc} (1, B) f, \ldots, \mathrm{eqc} (n, B) f\} = f (\mathrm{IdEta}(A) a) $
%\item Something relating the constructors and \lstinline{elim_id} to reflexivity
%\end{enumerate}
%and similarly for $A$.

%Really the point of these conditions is that from them, with some restrictions on input terms, we can get
%that lifting terms gives us the same type that we'd get from lifting the type. But there are still
%some restrictions (see the few that fail).

%It's probably not always possible to define these three things for every equivalence.
%Could generalize by rewriting. But this lets us avoid the rewriting problem from Nicolas' paper.

% TODO how does this get us something like primitive projections? Just makes IdEta definitionally equal to regular Id?

% TODO so we can probably just frame search in terms of DepConstr and DepElim and then generate proofs about this on an ad-hoc basis
% and get away with not including the specific details of our instantiations. We can give examples instead, give intuition, and say we generate
% the proofs in Coq

%For the second one we need not just an eliminator rule but also an identity rule.
%DEVOID assumed primitive projections which let them get away without thinking of this,
%but then had this weirdly ad-hoc ``repacking'' thing in their implementation.
%It turns out this is just a more general identity rule, which basically says what
%the identity function should lift to so that the transformation preserves definitional equalities.
%Actually deciding when to run this rule is one of the biggest challenges in practice,
%so we'll talk about that more in the implementation section.

\subsection{Decompile}
\label{sec:meatdecompile}

high-level decompiler stuff, maybe merge with other one

