%% For double-blind review submission, w/o CCS and ACM Reference (max submission space)
\documentclass[acmsmall,review,anonymous]{acmart}\settopmatter{printfolios=true,printccs=false,printacmref=false}
%% For double-blind review submission, w/ CCS and ACM Reference
%\documentclass[acmsmall,review,anonymous]{acmart}\settopmatter{printfolios=true}
%% For single-blind review submission, w/o CCS and ACM Reference (max submission space)
%\documentclass[acmsmall,review]{acmart}\settopmatter{printfolios=true,printccs=false,printacmref=false}
%% For single-blind review submission, w/ CCS and ACM Reference
%\documentclass[acmsmall,review]{acmart}\settopmatter{printfolios=true}
%% For final camera-ready submission, w/ required CCS and ACM Reference
%\documentclass[acmsmall]{acmart}\settopmatter{}


%% Journal information
%% Supplied to authors by publisher for camera-ready submission;
%% use defaults for review submission.
\acmJournal{PACMPL}
\acmVolume{1}
\acmNumber{CONF} % CONF = POPL or ICFP or OOPSLA
\acmArticle{1}
\acmYear{2018}
\acmMonth{1}
\acmDOI{} % \acmDOI{10.1145/nnnnnnn.nnnnnnn}
\startPage{1}

%% Copyright information
%% Supplied to authors (based on authors' rights management selection;
%% see authors.acm.org) by publisher for camera-ready submission;
%% use 'none' for review submission.
\setcopyright{none}
%\setcopyright{acmcopyright}
%\setcopyright{acmlicensed}
%\setcopyright{rightsretained}
%\copyrightyear{2018}           %% If different from \acmYear

%% Bibliography style
\bibliographystyle{ACM-Reference-Format}
%% Citation style
%% Note: author/year citations are required for papers published as an
%% issue of PACMPL.
\citestyle{acmauthoryear}   %% For author/year citations


%%%%%%%%%%%%%%%%%%%%%%%%%%%%%%%%%%%%%%%%%%%%%%%%%%%%%%%%%%%%%%%%%%%%%%
%% Note: Authors migrating a paper from PACMPL format to traditional
%% SIGPLAN proceedings format must update the '\documentclass' and
%% topmatter commands above; see 'acmart-sigplanproc-template.tex'.
%%%%%%%%%%%%%%%%%%%%%%%%%%%%%%%%%%%%%%%%%%%%%%%%%%%%%%%%%%%%%%%%%%%%%%


%% Some recommended packages.
\usepackage{booktabs}   %% For formal tables:
                        %% http://ctan.org/pkg/booktabs
\usepackage{subcaption} %% For complex figures with subfigures/subcaptions
                        %% http://ctan.org/pkg/subcaption

\usepackage{enumerate} % for lists
\usepackage{listings} % for code
\usepackage{xspace} % so we don't need to figure out spacing after \toolname every time
\usepackage{mathpartir} % for inference rules
\usepackage{scalerel} % to render definitional equality
\usepackage{bbm} % to render N for the natural numbers
\usepackage{syntax} % for code highlighting
\usepackage{xcolor} % for code highlighting
\usepackage{multirow} % for tables

% Nice rendering of Coq code
\lstdefinelanguage{coq}{
    keywords={Repair, module, Theorem, Proof, Record, Lemma, Definition, Abort, Qed, forall, Inductive, Type, Prop, Set, fun, fix, forall, Require, Import, Fixpoint, match, end, with, as, return, struct, Qed, Defined, let},
    basicstyle=\linespread{0.95}\small\ttfamily,
    keywordstyle=\color{blue},
    commentstyle=\itshape\rmfamily,
    showstringspaces=false,
    columns=flexible,
    breaklines=true,
    texcl=true,
    mathescape=true,
    tabsize=4,
    stringstyle=\color{brown},
    escapeinside={(@}{@)},
}

\lstset{language=coq} % default

\newcommand\toolname{\textsc{Carrot}\xspace} % tool name
\newcommand\company{Company\xspace} % company name
\newcommand\A{$A$\xspace} % recurring type A
\newcommand\B{$B$\xspace} % recurring type B
\newcommand{\reducedstrut}{\vrule width 0pt height .9\ht\strutbox depth .9\dp\strutbox\relax} % for \codediff and \codeauto
\newcommand{\codediff}[1]{%
  \begingroup
  \setlength{\fboxsep}{0pt}%
  \colorbox{orange!25}{\reducedstrut#1\/}%
  \endgroup
} % to highlight the difference between two code blocks
\newcommand{\codesima}[1]{%
  \begingroup
  \setlength{\fboxsep}{0pt}%
  \colorbox{orange!25}{\reducedstrut#1\/}%
  \endgroup
} % to highlight the similarities between two code blocks
\newcommand{\codesimb}[1]{%
  \begingroup
  \setlength{\fboxsep}{0pt}%
  \colorbox{red!25}{\reducedstrut#1\/}%
  \endgroup
} % to highlight the similarities between two code blocks
\newcommand{\codesimc}[1]{%
  \begingroup
  \setlength{\fboxsep}{0pt}%
  \colorbox{yellow!25}{\reducedstrut#1\/}%
  \endgroup
} % to highlight the similarities between two code blocks
\newcommand{\codesimd}[1]{%
  \begingroup
  \setlength{\fboxsep}{0pt}%
  \colorbox{green!25}{\reducedstrut#1\/}%
  \endgroup
} % to highlight the similarities between two code blocks
\newcommand{\codesime}[1]{%
  \begingroup
  \setlength{\fboxsep}{0pt}%
  \colorbox{blue!25}{\reducedstrut#1\/}%
  \endgroup
} % to highlight the similarities between two code blocks
\newcommand{\codeauto}[1]{%
  \begingroup
  \setlength{\fboxsep}{0pt}%
  \colorbox{cyan!30}{\reducedstrut#1\/}%
  \endgroup
} % to highlight automatically-generated terms

\begin{document}

%% Title information
\title[]{Proof Repair by Proof Term Transformation}         %% [Short Title] is optional;
                                        %% when present, will be used in
                                        %% header instead of Full Title.
%\titlenote{}             %% \titlenote is optional;
                                        %% can be repeated if necessary;
                                        %% contents suppressed with 'anonymous'
%\subtitle{Subtitle}                     %% \subtitle is optional
%\subtitlenote{with subtitle note}       %% \subtitlenote is optional;
                                        %% can be repeated if necessary;
                                        %% contents suppressed with 'anonymous'


%% Author information
%% Contents and number of authors suppressed with 'anonymous'.
%% Each author should be introduced by \author, followed by
%% \authornote (optional), \orcid (optional), \affiliation, and
%% \email.
%% An author may have multiple affiliations and/or emails; repeat the
%% appropriate command.
%% Many elements are not rendered, but should be provided for metadata
%% extraction tools.

%% Author with single affiliation.
\author{First1 Last1}
\authornote{with author1 note}          %% \authornote is optional;
                                        %% can be repeated if necessary
\orcid{nnnn-nnnn-nnnn-nnnn}             %% \orcid is optional
\affiliation{
  \position{Position1}
  \department{Department1}              %% \department is recommended
  \institution{Institution1}            %% \institution is required
  \streetaddress{Street1 Address1}
  \city{City1}
  \state{State1}
  \postcode{Post-Code1}
  \country{Country1}                    %% \country is recommended
}
\email{first1.last1@inst1.edu}          %% \email is recommended

%% Author with two affiliations and emails.
\author{First2 Last2}
\authornote{with author2 note}          %% \authornote is optional;
                                        %% can be repeated if necessary
\orcid{nnnn-nnnn-nnnn-nnnn}             %% \orcid is optional
\affiliation{
  \position{Position2a}
  \department{Department2a}             %% \department is recommended
  \institution{Institution2a}           %% \institution is required
  \streetaddress{Street2a Address2a}
  \city{City2a}
  \state{State2a}
  \postcode{Post-Code2a}
  \country{Country2a}                   %% \country is recommended
}
\email{first2.last2@inst2a.com}         %% \email is recommended
\affiliation{
  \position{Position2b}
  \department{Department2b}             %% \department is recommended
  \institution{Institution2b}           %% \institution is required
  \streetaddress{Street3b Address2b}
  \city{City2b}
  \state{State2b}
  \postcode{Post-Code2b}
  \country{Country2b}                   %% \country is recommended
}
\email{first2.last2@inst2b.org}         %% \email is recommended


%% Abstract
%% Note: \begin{abstract}...\end{abstract} environment must come
%% before \maketitle command
\begin{abstract}
We show a new approach to automatically repairing proofs in the Coq proof assistant.
Our approach combines a configurable proof term transformation with a proof term to tactic script decompiler.
The result is a flexible proof repair tool with tactic integration.
We have used this tool to support industrial integration with Coq,
ease development with dependent types,
support a benchmark from a user study, and
port functions and proofs between unary and binary natural numbers.
Our experiences inform new ideas we hope will scale proof repair
beyond small-scale industrial use.
\end{abstract}

%% 2012 ACM Computing Classification System (CSS) concepts
%% Generate at 'http://dl.acm.org/ccs/ccs.cfm'.
\begin{CCSXML}
<ccs2012>
<concept>
<concept_id>10011007.10011006.10011008</concept_id>
<concept_desc>Software and its engineering~General programming languages</concept_desc>
<concept_significance>500</concept_significance>
</concept>
<concept>
<concept_id>10003456.10003457.10003521.10003525</concept_id>
<concept_desc>Social and professional topics~History of programming languages</concept_desc>
<concept_significance>300</concept_significance>
</concept>
</ccs2012>
\end{CCSXML}

\ccsdesc[500]{Software and its engineering~General programming languages}
\ccsdesc[300]{Social and professional topics~History of programming languages}
%% End of generated code


%% Keywords
%% comma separated list
\keywords{proof engineering, proof repair, proof reuse, proof evolution}  %% \keywords are mandatory in final camera-ready submission


%% \maketitle
%% Note: \maketitle command must come after title commands, author
%% commands, abstract environment, Computing Classification System
%% environment and commands, and keywords command.
\maketitle

%% Body
\section{Introduction}

Program verification with interactive theorem provers has come a long way since its inception,
especially when it comes to the scale of programs that can be verified.
The seL4~\cite{Klein2009} verified operating system kernel, for example,
is the effort of a team of proof engineers over more than twenty years and spanning more than
a million lines of proof.
Given historical critique of verification~\cite{DeMillo1977} (emphasis ours):

\begin{quote}
A \textit{sufficiently fanatical researcher}
might be willing to devote \textit{two or 
three years} to verifying a significant 
piece of software if he could be 
assured that the software would remain stable.
\end{quote}
we can conclude that, since 1977, either verification has become much easier,
or our researchers have become much more fanatical. Unfortunately, not all has changed (emphasis still ours):

\begin{quote}
But real-life programs need to 
be maintained and modified. 
There is \textit{no reason to believe} that verifying a modified program is any 
easier than verifying the original the 
first time around.
\end{quote}
Tools that can automatically refactor or repair proofs~\cite{wibergh2019, WhitesidePhD, Dietrich2013, adams2015, Bourke12, Roe2016, robert2018, pumpkinpatch}
give us reason to believe that verifying a modified program can sometimes be easier than verifying the original the first time
around, even when proof engineers do not follow good development processes,
or when change occurs outside of proof engineers' control~\cite{PGL-045}.
Still, maintaining verified programs can be challenging: it means keeping not just the programs, but also the
specifications and proofs about those programs up-to-date.
This remains so difficult that sometimes, even experts give up in the face of change~\cite{replica}.

We make progress on two open challenges in \textit{proof repair}, the problem of automatically updating proofs in response
to changes in programs or specifications:

\begin{enumerate}
\item Proof repair tools support limited classes of changes like non-structural changes~\cite{pumpkinpatch} or a predefined set
of changes~\cite{robert2018, wibergh2019}, and these are not informed by the needs of proof engineers~\cite{replica}.
\item Proof repair tools are not yet integrated with typical proof engineering workflows like tactics~\cite{PGL-045, pumpkinpatch, robert2018}.
\end{enumerate}
Our progress towards these challenges leverages three key insights:

\begin{enumerate}
\item Proof repair is a form of proof reuse---reusing proofs about one specification to derive proofs about another specification---with 
the additional challenge that one of the specifications may cease to exist.
The key to supporting proof repair is to build a proof reuse
tool that can handle that additional challenge (Section~\ref{sec:key1}). 
\item A configurable proof term transformation can be used to build such a proof repair tool,
and the result can handle many different kinds of changes (Section~\ref{sec:key2}).
\item The transformed proof terms can then be translated back to tactic scripts (Section~\ref{sec:decompiler}).
\end{enumerate}

These insights informed our design of 
\toolname\footnote{Real name withheld for double-blind review.} (Configurable Approach to Repairing and Refactoring Outdated Tactics), a plugin for Coq 8.8.
\toolname combines a configurable proof term transformation,
search procedures to configure the proof term transformation,
and a prototype decompiler from proof terms back to tactics.
The result is a flexible proof repair tool that: 

\begin{enumerate}
\item supports changes informed by proof engineers and not supported by other repair tools, and
\item produces tactic scripts as part of better workflow integration.
\end{enumerate}
\toolname can support changes like porting non-dependent types to certain dependent types
and changing inductive structure---anything described by an equivalence or refinement with a certain form
detailed in Section~\ref{sec:key2}.

Our main technical advances are techniques for transforming proof terms directly to repair broken proofs, while our decompiler up to the tactic level is important for usability in Coq.
We demonstrate flexibility and usability with four case studies (Section~\ref{sec:search}), which show that \toolname:

\begin{enumerate}
\item can support variants of a benchmark from a user study of Coq proof engineers,
\item can simplify dependently-typed programming, %automating manual steps from previous work,
\item can help proof engineers port functions and proofs from unary to binary numbers, and
\item has helped an industrial proof engineer integrate Coq with a company workflow and write proofs about an implementation of the TLS Handshake Protocol.
\end{enumerate}
%Our experiences drive ideas (Section~\ref{sec:discussion}) that open the door to better proof engineering tools.


\section{A Simple Motivating Example}
\label{sec:overview}

\begin{figure*}
\begin{minipage}{0.46\textwidth}
   \lstinputlisting[firstline=1, lastline=3]{listswap.tex}
\end{minipage}
\hfill
\begin{minipage}{0.46\textwidth}
   \lstinputlisting[firstline=5, lastline=7]{listswap.tex}
\end{minipage}
\vspace{-0.3cm}
\caption{The updated \lstinline{list} (right) is the old \lstinline{list} (left) with its two constructors swapped (\codediff{orange}).}
\label{fig:listswap}
\end{figure*}

\toolname is available on Github.\footnote{\url{https://github.com/uwplse/pumpkin-pi}}
Consider a simple example of using \toolname: fixing broken list proofs after swapping the two list constructors (Figure~\ref{fig:listswap}).
This change is inspired by a similar change from a user study of proof engineers (Section~\ref{sec:replica}).
Even such a simple change can cause trouble in proofs, like this proof from the Coq standard library:\footnote{We use induction instead of pattern matching.}

\begin{lstlisting}
Lemma rev_app_distr {A} :
  $\forall$ (x y : list A), rev (x ++ y) = rev y ++ rev x.(@\vspace{-0.04cm}@)
Proof.(@\vspace{-0.04cm}@)
  induction x as [| a l IHl].(@\vspace{-0.04cm}@)
  induction y as [| a l IHl].(@\vspace{-0.04cm}@)
  simpl. auto.(@\vspace{-0.04cm}@)
  simpl. rewrite app_nil_r; auto.(@\vspace{-0.04cm}@)
  intro y. simpl.(@\vspace{-0.04cm}@)
  rewrite (IHl y). rewrite app_assoc; trivial.(@\vspace{-0.04cm}@)
Qed.
\end{lstlisting}
This theorem says that appending two lists and reversing the result behaves the same way as appending
the reverse of the second list onto the reverse of the first list.
When we change the \lstinline{list} type, the proof no longer works.
%This theorem statement \lstinline{rev_app_distr} defined over the old version of \lstinline{list} is our \textit{old specification}.
%When we change the \lstinline{list} type, we get the \textit{new specification}.
%But the \textit{old proof} or tactic script no longer works with this new specification.
To repair this proof with \toolname, we just run this command:

\begin{lstlisting}
Repair Old.list New.list in rev_app_distr.
\end{lstlisting}
assuming the old and new list types from Figure~\ref{fig:listswap} are in modules \lstinline{Old} and \lstinline{New}.
This suggests an updated proof script that succeeds (in $\codeauto{light blue}$ to denote that it is produced automatically
by \toolname):

\begin{lstlisting}[backgroundcolor=\color{cyan!30}]
Proof.(@\vspace{-0.04cm}@)
  intros x. induction x as [a l IHl|]; intro y0.(@\vspace{-0.04cm}@)
  - simpl. rewrite (IHl y0). simpl.(@\vspace{-0.04cm}@)
    rewrite (app_assoc (rev y0) (rev l) (a::[])).(@\vspace{-0.04cm}@)
    auto.(@\vspace{-0.04cm}@)
  - induction y0 as [a l IHl|].(@\vspace{-0.04cm}@)
    + simpl. rewrite (app_nil_r (rev l) (a::[])).(@\vspace{-0.04cm}@)
      auto.(@\vspace{-0.04cm}@)
    + auto.(@\vspace{-0.04cm}@)
Qed.
\end{lstlisting}
where the dependencies (\lstinline{rev}, \lstinline{++}, \lstinline{app_assoc}, and \lstinline{app_nil_r}) have
also been updated automatically.
If we'd like, we can manually modify this to something that more closely matches the style of the original proof:

\begin{lstlisting}
Proof.(@\vspace{-0.04cm}@)
  induction x as [a l IHl|].(@\vspace{-0.04cm}@)
  intro y. simpl.(@\vspace{-0.04cm}@)
  rewrite (IHl y). rewrite app_assoc; trivial.(@\vspace{-0.04cm}@)
  induction y as [a l IHl|].(@\vspace{-0.04cm}@)
  simpl. rewrite app_nil_r; auto.(@\vspace{-0.04cm}@)
  simpl. auto.(@\vspace{-0.04cm}@)
Qed.
\end{lstlisting}
We can even repair the entire list module from the Coq standard library all at once by running the \lstinline{Repair module}
command; the results of this are in \href{https://github.com/uwplse/pumpkin-pi/blob/master/plugin/coq/Swap.v}{\lstinline{Swap.v}}.
When we are done, we can get rid of \lstinline{Old.list} entirely.

The key to success here is taking advantage of Coq's structured proof term language:
Coq compiles every proof script to a proof term in a language called Gallina that is based on the calculus of inductive 
constructions---\toolname repairs that term.
\toolname then decompiles the repaired proof term back to a proof script that the proof engineer can maintain.
Here, \toolname transforms the proof term Coq compiles \lstinline{rev_app_distr} to,
and then decompiles that transformed proof term to the proof script in light blue.

In contrast, updating the poorly structured proof script directly would not be straightforward.
Even for the simple proof script above, grouping tactics by line, there are $6! = 720$ permutations of this proof script.
It is not clear which lines to swap since these tactics do not have a semantics beyond the searches their evaluation performs.
Furthermore, just swapping lines is not enough: even for such a simple change, we must also swap
arguments, so \lstinline{induction x as [| a l IHl]} becomes \lstinline{induction x as [a l IHl|]}.
Handling even swapping constructors this way would require a search procedure that would not generalize to other changes.
\citet{robert2018} describes the challenges of repairing tactics in detail.

By instead transforming proof terms, \toolname is able to try just $1$ rather than $720$ candidates.
By decompiling the transformed proof term, \toolname is able to suggest a tactic script in the end.
As later sections show, this approach is much more general than just permuting constructors.




\section{Problem Definition}
\label{sec:key1}

\toolname is a tool for \textit{proof repair}.
Proof repair is the problem of updating a broken proof in response to a change in a program or specification~\cite{PGL-045, pumpkinpatch}---in the
case of \toolname, a change in a type definition that corresponds to an equivalence (Section~\ref{sec:scope}).
Given an equivalence between types \A and \B,
\toolname repairs functions and proofs defined over \A to instead refer to \B.
It does this in a way that allows for removing references to \A, which is essential for proof repair,
since \A may be an old version of an updated type (Section~\ref{sec:repair}).

 %We can view proof repair as a form of 
%\textit{proof reuse}~\cite{Ringer2019, felty1994generalization, caplan1995logical, pons2000generalization, johnsen2004theorem}, % TODO consider citation list
%or reusing proofs about one specification (say, from another library, or from within the same proof development)
%to derive proofs about another specification.
%The difference is that in standard proof reuse, both of these specifications continue to exist.
%In contrast, proof repair is the process of reusing proofs across \textit{two versions of a single specification},
%only one of which---the new version---must continue to exist.
%That is, the old version of the specification may be removed after updating proofs to use the new version (Section~\ref{sec:repair}).
%The key to supporting proof repair is to build a proof reuse tool that can handle that additional challenge (Section~\ref{sec:time}).

%\begin{quote}
%\textbf{Insight 1}:
%Proof repair is a form of proof reuse---reusing proofs about one specification to derive proofs about another specification---with 
%the additional challenge that one of the specifications may cease to exist (Section~\ref{sec:repair}).
%The key to supporting proof repair is to build a proof reuse
%tool that can handle that additional challenge (Section~\ref{sec:time}).
%\end{quote}

\subsection{Scope: Type Equivalences}
\label{sec:scope}

\toolname supports proof repair in response to changes in type definitions
that correspond to \textit{type equivalences}~\cite{univalent2013homotopy},
or pairs of functions that map between two types and are mutual inverses.
Figure~\ref{fig:equivalence} shows a type equivalence between the two versions of \lstinline{list}
from Section~\ref{sec:overview}, Figure~\ref{fig:listswap} that \toolname discovered and proved automatically~\circled{1}.
When such a type equivalence between two types \A and \B exists, we say those types are \textit{equivalent} (denoted \A $\simeq$ \B). % for example:
%
%\begin{lstlisting}
%Old.list $\simeq$ New.list
%\end{lstlisting}

To give some intuition for what kinds of changes can be described by equivalences, we preview two changes below.
See Table~\ref{fig:changes} in Section~\ref{sec:search} for more examples.

\subsubsection*{Factoring out Constructors}
\label{sec:ex1}

\begin{figure}
\begin{minipage}{0.48\columnwidth}
\lstinputlisting[firstline=1, lastline=3]{equiv2.tex}
\end{minipage}
\hfill
\begin{minipage}{0.48\columnwidth}
\lstinputlisting[firstline=5, lastline=7]{equiv2.tex}
\end{minipage}
\vspace{-0.3cm}
\caption{The type \lstinline{J} (right) is \lstinline{I} (left) with \lstinline{A} and \lstinline{B} factored out to \lstinline{bool} (Coq standard library).}
\label{fig:equivalence2}
\end{figure}

Consider using \toolname to port functions and proofs across the change from the type \lstinline{I} to the type \lstinline{J} 
in Figure~\ref{fig:equivalence2}.
\lstinline{J} can be viewed as \lstinline{I} with its two constructors \lstinline{A} and \lstinline{B} pulled out to a
new hypothesis of type \lstinline{bool} for a single constructor.

With \toolname, the proof engineer can repair functions and proofs about \lstinline{I} to instead use \lstinline{J},
as long as she first configures \toolname to describe which constructor 
of \lstinline{I} maps to \lstinline{true} and which maps to \lstinline{false}.
This information about constructor mappings induces an equivalence \lstinline{I }$\simeq$\lstinline{ J}
along which \toolname repairs functions and proofs.
The repository shows an example of this, mapping \lstinline{A} to \lstinline{true} and \lstinline{B} to false,
and repairing proofs of De Morgan's laws~\circled{2}. % constr_refactor.v
%
%It uses \toolname to automatically repair functions and proofs over \lstinline{I}, like:

%\begin{lstlisting}
%Theorem demorgan_1 : $\forall$ (i1 i2 : I),(@\vspace{-0.04cm}@)
%  neg (and i1 i2) = or (neg i1) (neg i2).(@\vspace{-0.04cm}@)
%Proof.(@\vspace{-0.04cm}@)
%  intros i1 i2.(@\vspace{-0.04cm}@)
%  induction i1; auto.(@\vspace{-0.04cm}@)
%Qed.
%\end{lstlisting}
%to corresponding functions and proofs over \lstinline{J}, like:
%
%\begin{lstlisting}[backgroundcolor=\color{cyan!30}]
%Theorem demorgan_1 : $\forall$ (j1 j2 : J),(@\vspace{-0.04cm}@)
%  neg (and j1 j2) = or (neg j1) (neg j2).(@\vspace{-0.04cm}@)
%Proof.(@\vspace{-0.04cm}@)
%  intros j1 j2.(@\vspace{-0.04cm}@)
%  induction j1 (@\codediff{as [b]. induction b as [ | ]}@); auto.(@\vspace{-0.04cm}@)
%Qed.
%\end{lstlisting}
%These repaired functions and proofs refer to \lstinline{J} in place of \lstinline{I}.
%Otherwise, they behave the same way as the functions and proofs over \lstinline{I} up to the equivalence between
%\lstinline{I} and \lstinline{J}---Section~\ref{sec:repair} explains this intuition more formally.

\subsubsection*{Adding a Dependent Index}
\label{sec:ex2}

\begin{figure*}
\begin{minipage}{0.40\textwidth}
   \lstinputlisting[firstline=1, lastline=3]{listtovect.tex}
\end{minipage}
\hfill
\begin{minipage}{0.58\textwidth}
   \lstinputlisting[firstline=5, lastline=7]{listtovect.tex}
\end{minipage}
\vspace{-0.3cm}
\caption{A vector (right) is a list (left) indexed by its length.}
\label{fig:listtovect}
\end{figure*}

At first glance, the word \textit{equivalence} may seem to imply that \toolname can support only changes in
which the proof engineer does not add or remove information.
But equivalences are more powerful than they may seem.
For example, in cubical type theory, with the help of quotient types, it is possible to form an equivalence
from a relation, even when the relation is not an equivalence~\cite{angiuli2020internalizing}.

While Coq lacks quotient types,
it is possible to achieve a similar outcome and use \toolname for changes that add or remove information
when those changes can be expressed as equivalences between $\Sigma$ types or sum types.
The idea is, when possible, to separate out the new information
into a projection of a $\Sigma$ type or a constructor of a sum type.
Proofs about this new information become the proof obligation for the proof engineer,
and \toolname automates the rest.

Consider, for example, changing a list to a length-indexed vector (Figure~\ref{fig:listtovect}).
\toolname can repair functions and proofs about lists to functions and proofs about vectors of a particular length~\circled{3}, % Example.v
since:

\begin{lstlisting}
$\Sigma$(l : list T).length l = n $\simeq$ vector T n.
\end{lstlisting}
%
%(\href{https://github.com/uwplse/pumpkin-pi/blob/master/plugin/coq/examples/Example.v}{\lstinline{Example.v}}).%%
%
%With the proof reuse tool \textsc{Devoid}~\cite{Ringer2019},
%it is possible to repair proofs about lists to proofs about vectors of \textit{some} length, since:
%
%\begin{lstlisting}
%packed_vect T := $\Sigma$(n : nat).vector T n.
%list T $\simeq$ packed_vector T.
%\end{lstlisting}
%This is enough to automatically repair a lemma about lists:
%
%\begin{lstlisting}
%$\forall$ {A B} (l1 : list A) (l2 : list B),(@\vspace{-0.04cm}@)
%  zip_with pair l1 l2 = zip l1 l2.
%\end{lstlisting}
%to a lemma about vectors of some length:
%
%\begin{lstlisting}
%$\forall$ {A B} (l1 : (@\codediff{packed\_vect A}@)) (l2 : (@\codediff{packed\_vect B}@)),(@\vspace{-0.04cm}@)
%  zip_with pair l1 l2 = zip l1 l2.
%\end{lstlisting}
%recursively updating dependencies \lstinline{zip} and \lstinline{zip_with}.
%It is not enough, however, to help the proof engineer get from that to a proof about vectors \textit{of a particular length}:
%
%\begin{lstlisting}
%$\forall$ {A B} (@\codediff{n}@) (l1 : (@\codediff{vector A n}@)) (l2 : (@\codediff{vector B n}@)),(@\vspace{-0.04cm}@)
%  zip_with pair (@\codediff{n}@) l1 l2 = zip (@\codediff{n}@) l1 l2.
%\end{lstlisting}
%\textsc{Devoid} leaves this step to the proof engineer.
%\toolname, in contrast, can handle this step as well (\href{https://github.com/uwplse/pumpkin-pi/blob/master/plugin/coq/examples/Example.v}{\lstinline{Example.v}}).
%The key is to repair functions and proofs across this equivalence:
From the proof engineer's perspective, after updating specifications from \lstinline{list} to \lstinline{vector},
to fix her functions and proofs, she must additionally prove invariants about the lengths of her lists.
\toolname makes it easy to separate out that proof obligation, then automates the rest.
%Section~\ref{sec:search} shows this and other case studies using \toolname to repair real proofs
%informed by the needs of proof engineers.

\subsection{Goal: Transport with a Twist}
\label{sec:repair}

The goal of a tool for proof repair across type equivalences is to implement a kind of proof reuse known as \textit{transport},
but in a way that is suitable for repair.
A transport method takes an input term $t$ and produces an output term $t'$ that is \textit{equal up to transport}
along an equivalence $A \simeq B$ (denoted $t \equiv_{A \simeq B} t'$).
Informally, equality up to transport means that if $t$ is a function, then $t'$ behaves the same way modulo the equivalence;
if $t$ is a proof, then $t'$ proves the same theorem the same way modulo the equivalence.
For example, in Section~\ref{sec:overview}, the original append function \lstinline{++} over \lstinline{Old.list}
and the updated append function \lstinline{++} over \lstinline{New.list} that \toolname produces are
equal up to transport along the equivalence from Figure~\ref{fig:equivalence}, since:

\begin{lstlisting}
$\forall$ T (l1 l2 : Old.list T),
  swap T (l1 ++ l2) = (swap T l1) ++ (swap T l2).
\end{lstlisting}
by \lstinline{app_ok}~\circled{1}.
The original \lstinline{rev_app_distr} is equal to the transformed proof along the same equivalence,
since it proves the same thing the same way as the transformed proof up to the same equivalence, and up to the changes in \lstinline{++}
and \lstinline{rev}.

Note that for any equivalent \A and \B, there can be many equivalences $A \simeq B$.
Equality up to transport is along a \textit{particular} equivalence, though we erase this in the notation.
Also note that, as Coq lacks univalence, and so equality up to transport cannot be stated in Coq in general,
this notation should be interpreted in the context of a univalent metatheory.
The formal details of equality up to transport in such a theory can be found in \citet{univalent2013homotopy}, and an approximation in Coq without univalence can be found in \citet{tabareau2017equivalences}.

Transport typically works by applying the functions that make up the equivalence to convert
inputs and outputs back and forth between equivalent types.
This approach would not be suitable for repair, since it does not make it possible to remove the old specification.
\toolname implements transport in a way that allows for removing references to the old specification---using a proof term transformation.

%The goal of a proof repair tool like \toolname is to define a transport method that
%can remove references to the old specification, %rather than converting back and forth like standard transport methods.
%%That way, the proof repair tool can produce proofs that no longer refer in any way to the old specification,
%since the old specification may no longer exist.

%Section~\ref{sec:overview} showed a simple case of this: \toolname
%reused the proof of \lstinline{rev_app_distr} defined over \lstinline{Old.list}
%to generate a new proof of \lstinline{rev_app_distr} defined over equivalent \lstinline{New.list}.
%Furthermore, it did so in a way that removed all references to \lstinline{Old.list} in the proof
%and in its dependencies.
%That way, after calling \lstinline{Repair}, \lstinline{Old.list} could be removed.

%\subsection{A Tool for Proof Repair Across Equivalences}
%\label{sec:time}





\section{A Configurable Proof Term Transformation}
\label{sec:key2}

At the heart of \toolname is a configurable proof term transformation for transporting
proofs across equivalences. 
This proof term transformation is a generalization of the proof term transformation from 
\textsc{Devoid}~\cite{Ringer2019}, which solved this problem for a particular class of equivalences.
\toolname moves the reasoning specific to that class of equivalences into the configuration. 

\begin{quote}
\textbf{Insight 2}:
A configurable proof term transformation can be used to build such a proof repair tool,
and the result can handle many different kinds of changes.
\end{quote}

This section starts by introducing the configuration (Section~\ref{sec:configurable}),
then introduces the proof term transformation that builds on that (Section~\ref{sec:generic}).
It then gives intuition for what it means for a configuration to be correct (Section~\ref{sec:art}),
and finally describes the additional work needed to go from this transformation to the implementation (Section~\ref{sec:implementation}). 

\paragraph{Conventions}
All terms that we introduce in this section are in CIC$_{\omega}$ (the core calculus of Coq) with primitive eliminators,
the syntax for which is in Figure~\ref{fig:syntax}.
The typing rules are standard.
We assume the existence of an inductive type $\Sigma$ with constructor $\exists$ and projections $\pi_l$ and $\pi_r$.
Throughout, we use $\vec{t}$ and $\{t_1, \ldots, t_n\}$ to denote lists of terms.

\begin{figure}
\small
\begin{grammar}
<i> $\in \mathbbm{N}$, <v> $\in$ Vars, <s> $\in$ \{ Prop, Set, Type<i> \}

<t> ::= <v> | <s> | $\Pi$ (<v> : <t>) . <t> | $\lambda$ (<v> : <t>) . <t> | <t> <t> | \\ 
Ind (<v> : <t>)\{<t>,\ldots,<t>\} | Constr (<i>, <t>) | Elim(<t>, <t>)\{<t>,\ldots,<t>\}
\end{grammar}
\vspace{-0.3cm}
\caption{Syntax for CIC$_\omega$, from \citet{Timany2015FirstST}.}
\label{fig:syntax}
\end{figure}

\subsection{The Configuration}
\label{sec:configurable}

The configuration is the key to building a proof term transformation that can support many different classes of changes.
%Before introducing the proof term transformation, we will describe the configuration, which in effect specifies the behavior
%of the transformation.
%It instantiates the proof term transformation to two equivalent types \A and \B, so that the proof term transformation
%can transform terms defined over \A to terms defined over \B instead.
%The behavior of the proof term transformation at a particular equivalence hinges on correct configuration.
At a high level, the configuration helps the transformation achieve two goals:

\begin{enumerate}
\item preserve equality up to transport along the equivalence between \A and \B, and
\item produce well-typed terms.
\end{enumerate}
This configuration is a pair of pairs:

\begin{lstlisting}
((DepConstr, DepElim), (Eta, Iota))
\end{lstlisting}
each of which corresponds to one of the two goals, namely:

\begin{enumerate}
\item \lstinline{DepConstr} and \lstinline{DepElim} define how to transform constructors and eliminators, thereby preserving the equivalence (Section~\ref{sec:equivalence}), and 
\item \lstinline{Eta} and \lstinline{Iota} define how to transform $\eta$-expansion and $\iota$-reduction of constructors and eliminators, thereby producing well-typed terms (Section~\ref{sec:equality}).
\end{enumerate}
Each of these is defined in CIC$_{\omega}$ for any given equivalence.
%\textbf{Configure} passes this configuration to \textbf{Transform}.
The four parts of this configuration must relate to one another in a particular way in order for the proof
term transformation to work correctly (Section~\ref{sec:art}).

\subsubsection{Preserving the Equivalence}
\label{sec:equivalence}

To preserve the equivalence, the configuration ports terms over \A to terms over \B by viewing each
term of type \B as if it is an \A.
This way, the rest of the transformation can replace values of \A with values of \B and
inductive proofs about \A with inductive proofs about \B, then recursively transform
subterms without changing the order or number of arguments.

The two configuration parts responsible for preserving this are \lstinline{DepConstr}
and \lstinline{DepElim} (\textit{dependent constructors} and \textit{eliminators}).
These describe how to construct and eliminate \A and \B, wrapping the types with a common inductive structure.
There must be the same number of dependent constructors and cases in dependent eliminators for \A and \B,
even if \A and \B are types with different numbers of constructors.

\begin{figure}
\begin{minipage}{0.48\textwidth}
\begin{lstlisting}
DepConstr(0, list T) : list T :=(@\vspace{-0.04cm}@)
  Constr((@\codediff{0}@), list T).(@\vspace{-0.04cm}@)
DepConstr(1, list T) t l : list T :=(@\vspace{-0.04cm}@)
  Constr ((@\codediff{1}@), list T) t l.(@\vspace{-0.04cm}@)
(@\vspace{-0.14cm}@)
DepElim(l, P) { p$_{\mathtt{nil}}$, p$_{\mathtt{cons}}$ } : P l :=(@\vspace{-0.04cm}@)
  Elim(l, P) { (@\codediff{p$_{\mathtt{nil}}$}@), (@\codediff{p$_{\mathtt{cons}}$}@) }.
\end{lstlisting}
\end{minipage}
\hfill
\begin{minipage}{0.48\textwidth}
\begin{lstlisting}
DepConstr(0, list T) : list T :=(@\vspace{-0.04cm}@)
  Constr((@\codediff{1}@), list T).(@\vspace{-0.04cm}@)
DepConstr(1, list T) t l : list T :=(@\vspace{-0.04cm}@)
  Constr((@\codediff{0}@), list T) t l.(@\vspace{-0.04cm}@)
(@\vspace{-0.14cm}@)
DepElim(l, P) { p$_{\mathtt{nil}}$, p$_{\mathtt{cons}}$ } : P l :=(@\vspace{-0.04cm}@)
  Elim(l, P) { (@\codediff{p$_{\mathtt{cons}}$}@), (@\codediff{p$_{\mathtt{nil}}$}@) }.
\end{lstlisting}
\end{minipage}
\vspace{-0.3cm}
\caption{The dependent constructors and eliminators for old (left) and new (right) \lstinline{list}.}
\label{fig:listconfig}
\end{figure}

For the \lstinline{list} change from Section~\ref{sec:overview},
the configuration that \toolname discovers uses the dependent constructors
and eliminators in Figure~\ref{fig:listconfig}. The dependent constructors for \lstinline{Old.list}
are the normal constructors \lstinline{nil} and \lstinline{cons} with the order unchanged,
while the dependent constructors for \lstinline{New.list} swap \lstinline{nil} and \lstinline{cons}
back to the original order.
Similarly, the dependent eliminator for \lstinline{Old.list} is the normal eliminator for \lstinline{Old.list},
while the dependent eliminator for \lstinline{New.list} swaps cases.

These constructors and eliminators can be \textit{dependent}.
One example of this arises from implementing the change from \lstinline{list T} to $\Sigma$\lstinline{(n : nat).vector T n}.
The configuration \toolname discovers for this configures the dependent constructors to pack the index into an existential, for example:

\begin{lstlisting}
DepConstr(0, $\Sigma$(n : nat).vector T n) : $\Sigma$(n : nat).vector T n :=(@\vspace{-0.04cm}@)
  $\exists$ (Constr(0, nat)) (Constr(0, vector T)).
\end{lstlisting}
and the eliminator it discovers eliminates over the projections:

\begin{lstlisting}
DepElim(s, P) { f$_0$ f$_1$ } : P ($\exists$ ($\pi_l$ s) ($\pi_r$ s)) :=(@\vspace{-0.04cm}@)
  Elim($\pi_r$ s, $\lambda$ (n : nat) (v : vector T n) . P ($\exists$ n v)) {(@\vspace{-0.04cm}@)
    f$_0$,(@\vspace{-0.04cm}@)
    ($\lambda$ (t : T) (n : nat) (v : vector T n) . f$_1$ t ($\exists$ n v))(@\vspace{-0.04cm}@)
  }. 
\end{lstlisting}

In both of these examples, the only interesting work moves into the configuration:
the configuration for the first example swaps constructors and cases,
and the configuration for the second example implements the constructor and eliminator rules from the \textsc{Devoid} transformation.
That way, the rest of the \toolname transformation does not need to add, drop, or reorder arguments.
%In essence, all of the difficult work moves into the configuration.
Furthermore, both examples use automatic configuration, which means that the \textbf{Configure} component of \toolname is able to 
discover \lstinline{DepConstr} and \lstinline{DepElim} from just the types \A and \B, taking care of even the difficult work.

\subsubsection{Producing Well-Typed Terms}
\label{sec:equality}

The other configuration parts \lstinline{Eta} and \lstinline{Iota} deal with producing well-typed terms,
in particular by transporting equalities.
A naive proof term transformation in a non-univalent language, as noted in \citet{tabareau2019marriage},
may fail to generate well-typed terms if it does not consider the problem of transporting equalities.
Otherwise, if the transformation transforms a term \lstinline{t : T} to some \lstinline{t' : T'}, it does not necessarily
hold that it transforms \lstinline{T} to \lstinline{T'}.

\lstinline{Eta} and \lstinline{Iota} describe how to transport equalities.
More formally, they define $\eta$-expansion and $\iota$-reduction of \A and \B,
which may be propositional rather than definitional, and so must be explicit in the transformation.
$\eta$-expansion describes how to expand a term to apply a constructor to an eliminator in a way that preserves propositional equality,
and is important for defining dependent eliminators~\cite{nlab:eta-conversion}.
$\iota$-reduction ($\beta$-reduction for inductive types) describes how to reduce an elimination of a constructor~\cite{nlab:beta-reduction}.

\begin{figure}
\begin{minipage}{0.48\textwidth}
   \lstinputlisting[firstline=1, lastline=8]{nattobin.tex}
\end{minipage}
\hfill
\begin{minipage}{0.48\textwidth}
   \lstinputlisting[firstline=10, lastline=17]{nattobin.tex}
\end{minipage}
\vspace{-0.3cm}
\caption{Unary (left) and binary (right) natural numbers.}
\label{fig:nattobin}
\end{figure}

The configuration for the change from \lstinline{list} to \lstinline{$\Sigma$(n : nat).vector T n} has propositional \lstinline{Eta}.
To describe \lstinline{Eta}, it is enough to use the standard definition of $\eta$-expansion for $\Sigma$:

\begin{lstlisting}
Eta ($\Sigma$(n : nat).vector T n) := $\lambda$ (s : $\Sigma$(n : nat).vector T n).$\exists$ ($\pi_l$ s) ($\pi_r$ s).
\end{lstlisting}
which is propositional and not definitional in Coq.
Thanks to this, we can forego the assumption that our language has primitive projections (definitional $\eta$ for $\Sigma$).

Each \lstinline{Iota}---one per constructor---describes and proves the $\iota$-reduction behavior
of \lstinline{DepElim} on the corresponding case.
This is needed, for example, to port proofs about unary numbers \lstinline{nat} to
proofs about binary numbers \lstinline{N} (Figure~\ref{fig:nattobin}).
While we can in fact define \lstinline{DepConstr} and \lstinline{DepElim} to induce an equivalence
between them (see Section~\ref{sec:bin}), we run into trouble reasoning about applications of \lstinline{DepElim},
since proofs about \lstinline{nat} that hold by reflexivity do not necessarily hold by reflexivity over \lstinline{N}. 
For example, in Coq, while \lstinline{S (n + m)  = S n + m} holds by reflexivity over \lstinline{nat},
when we define \lstinline{+} with \lstinline{DepElim} over \lstinline{N},
the corresponding theorem over \lstinline{N} does not hold by reflexivity.

To transform proofs about \lstinline{nat} to proofs about \lstinline{N}, we must transform \textit{definitional} $\iota$-reduction over \lstinline{nat} to explicit \textit{propositional} $\iota$-reduction over \lstinline{N}.
For our choice of configuration in Section~\ref{sec:bin},
$\iota$-reduction is definitional over \lstinline{nat}, since a proof of:

\begin{lstlisting}
$\forall$ P p$_\texttt{0}$ p$_\texttt{S}$ n, DepElim((@\codediff{DepConstr(1, nat) n}@), P) { p$_\texttt{0}$ p$_\texttt{S}$ } = (@\codediff{p$_\texttt{S}$}@) n (DepElim(n, P) { p$_\texttt{0}$ p$_\texttt{S}$ }).
\end{lstlisting}
goes through by reflexivity.
However, the corresponding $\iota$ rule for \lstinline{N} is propositional, since:

\begin{lstlisting}
$\forall$ P p$_\texttt{0}$ p$_\texttt{S}$ n, DepElim((@\codediff{DepConstr(1, N) n}@), P) { p$_\texttt{0}$ p$_\texttt{S}$ } = (@\codediff{p$_\texttt{S}$}@) n (DepElim(n, P) { p$_\texttt{0}$ p$_\texttt{S}$ }).
\end{lstlisting}
no longer holds by reflexivity.
The \lstinline{Iota} rules are exactly rewrites along proofs of the theorems above for the successor case.
The transformation replaces rewrites by reflexivity over \lstinline{nat} to rewrites by propositional equalities over \lstinline{N},
that way \lstinline{DepElim} behaves the same over \lstinline{nat} and \lstinline{N}.

Taken together over both \A and \B, \lstinline{Iota} describes how the inductive structures of \A and \B differ from each other.
The transformation requires that \lstinline{DepElim} over \A and over \B always have the same inductive structure
as each other, so if \A and \B themselves have the same 
inductive structure (if they are \textit{ornaments}~\cite{mcbride}),
then if $\iota$-reduction is definitional over \lstinline{DepElim} on \A, it will be possible to choose
\lstinline{DepElim} with definitional $\iota$ on \B.
Otherwise, if \A and \B have different inductive structures, as with \lstinline{nat} and \lstinline{N},
then definitional $\iota$ over one would become propositional $\iota$ over the other.
For the case of \lstinline{nat} and \lstinline{N},
the need for propositional $\iota$ was noted as far back as \citet{magaud2000changing}.
\lstinline{Iota} in the configuration encodes this more generally.

\subsection{The Proof Term Transformation}
\label{sec:generic}

\begin{figure}
\begin{mathpar}
\mprset{flushleft}
\small
\hfill\fbox{$\Gamma$ $\vdash$ $t$ $\Uparrow$ $t'$}\vspace{-0.4cm}\\

\inferrule[Dep-Elim]
  { \Gamma \vdash a \Uparrow b \\ \Gamma \vdash p_{a} \Uparrow p_b \\ \Gamma \vdash \vec{f_{a}}\phantom{l} \Uparrow \vec{f_{b}} }
  { \Gamma \vdash \mathrm{DepElim}(a,\ p_{a}) \vec{f_{a}} \Uparrow \mathrm{DepElim}(b,\ p_b) \vec{f_{b}} }

\inferrule[Dep-Constr]
{ \Gamma \vdash \vec{t}_{a} \Uparrow \vec{t}_{b} } %\\ TODO must we explicitly lift A to B if we want to handle parameters/indices?
{ \Gamma \vdash \mathrm{DepConstr}(j,\ A)\ \vec{t}_{a} \Uparrow \mathrm{DepConstr}(j,\ B)\ \vec{t}_{b}  }

\inferrule[Eta]
  { \\ }
  { \Gamma \vdash \mathrm{Eta}(A) \Uparrow \mathrm{Eta}(B) }

\inferrule[Iota]
  { \Gamma \vdash q_A \Uparrow q_B \\ \Gamma \vdash t_A \Uparrow t_B }
  { \Gamma \vdash \mathrm{Iota}(j,\ A,\ q_A,\ t_A) \Uparrow \mathrm{Iota}(j,\ B,\ q_B,\ t_B) }

\inferrule[Equivalence]
  { \\ }
  { \Gamma \vdash A\ \Uparrow B }

\inferrule[Constr]
{ \Gamma \vdash T \Uparrow T' \\ \Gamma \vdash \vec{t} \Uparrow \vec{t'} }
{ \Gamma \vdash \mathrm{Constr}(j,\ T)\ \vec{t} \Uparrow \mathrm{Constr}(j,\ T')\ \vec{t'} }

\inferrule[Ind]
  { \Gamma \vdash T \Uparrow T' \\ \Gamma \vdash \vec{C} \Uparrow \vec{C'}  }
  { \Gamma \vdash \mathrm{Ind} (\mathit{Ty} : T) \vec{C} \Uparrow \mathrm{Ind} (\mathit{Ty} : T') \vec{C'} }

%% Application
\inferrule[App]
 { \Gamma \vdash f \Uparrow f' \\ \Gamma \vdash t \Uparrow t'}
 { \Gamma \vdash f t \Uparrow f' t' }

\inferrule[Elim] % TODO wait why do we have c here when it clearly refers to the term we eliminate over? um
  { \Gamma \vdash c \Uparrow c' \\ \Gamma \vdash Q \Uparrow Q' \\ \Gamma \vdash \vec{f} \Uparrow \vec{f'}}
  { \Gamma \vdash \mathrm{Elim}(c, Q) \vec{f} \Uparrow \mathrm{Elim}(c', Q') \vec{f'}  }

% Lamda
\inferrule[Lam]
  { \Gamma \vdash T \Uparrow T' \\\\ \Gamma,\ t : T \vdash b \Uparrow b' }
  {\Gamma \vdash \lambda (t : T).b \Uparrow \lambda (t : T').b'}

% Product
\inferrule[Prod]
  { \Gamma \vdash T \Uparrow T' \\\\ \Gamma,\ t : T \vdash b \Uparrow b' }
  {\Gamma \vdash \Pi (t : T).b \Uparrow \Pi (t : T').b'}
\end{mathpar}
\vspace{-0.3cm}
\caption{Proof term transformation for transporting terms across an equivalence $A \simeq B$ described by configuration \lstinline{((DepConstr, DepElim), (Eta, Iota))}.}
\label{fig:final}
\end{figure}

Figure~\ref{fig:final} shows the proof term transformation $\Gamma \vdash t \Uparrow t'$ that forms the core of \toolname.
%Like the transformation from \textsc{Devoid},
The transformation is parameterized over two equivalent types \A and \B (\textsc{Equivalence})
as well as the configuration terms, which appear in the transformation explicitly.
It assumes fully $\eta$-expanded functions.

The transformation is simple because it moves the bulk of the work into the configuration,
and because it represents the configuration explicitly.
It does not fully describe the search procedure for transforming terms that \toolname implements.
Before running the transformation, \toolname unifies the 
input proof term with applications of the terms in the configuration for \A. 
The transformation then transforms the applications of the terms in the configuration of \A
to applications of the terms in the configuration for \B.
Reducing the result produces the output term defined over \B.

Figure~\ref{fig:appswap1} shows this with the list append function \lstinline{++} from Section~\ref{sec:overview}.
To update the append function (top left), \toolname
identifies implicit applications of \lstinline{DepConstr} and \lstinline{DepElim} and expands them (bottom left).
In this case, expansion works just by matching against \lstinline{Constr} and \lstinline{Elim}, 
though this can be more difficult when \lstinline{DepConstr} and \lstinline{DepElim} over \A are dependent.
After expansion, the transformation then recursively substitutes in \lstinline{New.list}
for \lstinline{Old.list}, which moves \lstinline{DepConstr} and \lstinline{DepElim}
to construct and eliminate over the updated type (bottom right).
Finally, this reduces to a term with swapped constructors and cases (top right).

The goal of the proof term transformation is to preserve equality up to transport along the equivalence $A \simeq B$,
while no longer referring to the old specification.
That is, we need that $\Gamma \vdash t \Uparrow t' \rightarrow t \equiv_{A \simeq B} t'$, and that $t'$ refers to \B in place of \A.
%This is the same as the correctness criterion for the program transformation from \textsc{Devoid} that this is based on,
%with the transformation generalized to handle other equivalences beyond the class that \textsc{Devoid} supports.
The key steps in this transformation that make this possible are porting functions and proofs along the configuration corresponding
to a particular equivalence (\textsc{Dep-Constr}, \textsc{Dep-Elim}, \textsc{Eta}, and \textsc{Iota}).
The rest is straightforward.

Note that this goal for correctness is metatheoretical:
stating and proving that two terms are equal up to transport is in general not possible in a language like Coq
without additional axioms, though it is in some cases realizable with an external tool like the
univalent parametricity framework~\cite{tabareau2017equivalences}.
\toolname does not yet generate these proofs as we were focused on building a usable tool with axiomatic freedom and few dependencies.
Coq ensures that all terms that plugins produce are well-typed; for now, %as with \textsc{Devoid},
the proof engineer must vet the transformed specifications herself to ensure that they specify the expected behavior.

\begin{figure}
\begin{minipage}{0.49\textwidth}
\begin{lstlisting}
(* 1: original term *)(@\vspace{-0.04cm}@)
$\lambda$ (T : Type) (l m : list T) .(@\vspace{-0.04cm}@)
 (@\codediff{Elim}@)(@\vspace{-0.04cm}@)
   (l, $\lambda$(l: Old.list T).list T $\rightarrow$ list T))(@\vspace{-0.04cm}@)
 {(@\vspace{-0.04cm}@)
   ($\lambda$ m . m),(@\vspace{-0.04cm}@)
   ($\lambda$ t _ IHl m.(@\vspace{-0.04cm}@)
      (@\codediff{Constr}@)(1, Old.list T) t (IHl m))(@\vspace{-0.04cm}@)
 } m.(@\vspace{-0.04cm}@)
(@\vspace{-0.14cm}@)
(* 2: after unifying (@\texttt{with}@) configuration *)(@\vspace{-0.04cm}@)
$\lambda$ (T : Type) (l m : list T) .(@\vspace{-0.04cm}@)
 (@\codediff{DepElim}@)(@\vspace{-0.04cm}@)
   (l, $\lambda$(l: Old.list T).list T $\rightarrow$ list T))(@\vspace{-0.04cm}@)
 {(@\vspace{-0.04cm}@)
   ($\lambda$ m . m)(@\vspace{-0.04cm}@)
   ($\lambda$ t _ IHl m.(@\vspace{-0.04cm}@)
      (@\codediff{DepConstr}@)(1, Old.list T) t (IHl m))(@\vspace{-0.04cm}@)
 } m.
\end{lstlisting}
\end{minipage}
\hfill
\begin{minipage}{0.49\textwidth}
\begin{lstlisting}
(* 4: reduced to final term *)(@\vspace{-0.04cm}@)
$\lambda$ (T : Type) (l m : list T) .(@\vspace{-0.04cm}@)
 (@\codediff{Elim}@)(@\vspace{-0.04cm}@)
   (l, $\lambda$(l: New.list T).list T $\rightarrow$ list T))(@\vspace{-0.04cm}@)
 {(@\vspace{-0.04cm}@)
   ($\lambda$ t _ IHl m.(@\vspace{-0.04cm}@)
      (@\codediff{Constr}@)(0, New.list T) t (IHl m)),(@\vspace{-0.04cm}@)
   ($\lambda$ m . m)(@\vspace{-0.04cm}@)
 } m.(@\vspace{-0.04cm}@)
(@\vspace{-0.14cm}@)
(* 3: ported to New.list *)(@\vspace{-0.04cm}@)
$\lambda$ (T : Type) (l m : list T) .(@\vspace{-0.04cm}@)
 DepElim(@\vspace{-0.04cm}@)
   (l, $\lambda$(l: (@\codediff{New.list T}@)).list T $\rightarrow$ list T))(@\vspace{-0.04cm}@)
 {(@\vspace{-0.04cm}@)
   ($\lambda$ m . m)(@\vspace{-0.04cm}@)
   ($\lambda$ t _ IHl m.(@\vspace{-0.04cm}@)
      DepConstr(1, (@\codediff{New.list T}@)) t (IHl m))(@\vspace{-0.04cm}@)
 } m.
\end{lstlisting}
\end{minipage}
\vspace{-0.3cm}
\caption{Swapping cases of the append function, with names fully qualified only when needed for clarity, counterclockwise: 1) the input, 2) the term unified with the configuration, 3) the term ported to the updated type, and 4) the term reduced to the final output.}
\label{fig:appswap1}
\end{figure}

\subsection{Specifying Correct Configuration}
\label{sec:art}

Both when designing a search procedure for an automatic configuration and when
configuring \toolname manually, choosing a correct and useful configuration is important,
and it is not always straightforward. This section specifies what it means for these
to be correct and gives some intuition as to why.
Section~\ref{sec:search} shows some useful example configurations.

The configuration instantiates the proof term transformation to a particular equivalence between \A and \B.
Choosing an equivalence is a bit of an art:
there can be infinitely many equivalences that correspond to a 
given change in specification, only some of which are useful (for example, \href{https://github.com/uwplse/ornamental-search/blob/master/plugin/coq/playground/refine_unit.v}{refine_unit.v} proves that any \A is equivalent to \lstinline{unit} refined by \A).
Beyond that, even once we have chosen an equivalence, we could define many possible configurations that correspond
to the equivalence, some of which will produce functions and proofs that are more useful or efficient than others
(consider \lstinline{DepElim} converting through several intermediate types).

Thankfully, once the art is done, we at least understand what it means for it to be \textit{correct art}.
The correctness criteria for the configuration relate \lstinline{DepConstr}, \lstinline{DepElim}, \lstinline{Eta}, and \lstinline{Iota}
in a way that preserves equivalence (Section~\ref{sec:equivalence}) coherently with equality (Section~\ref{sec:equality}).

To preserve equivalence, we need that \lstinline{DepElim} and \lstinline{DepConstr} together induce an equivalence between \A and \B,
formed by one function that eliminates \A and constructs \B, and another function that eliminates \B and constructs \A:

\begin{lstlisting}
f : A $\rightarrow$ B := DepElim(a, $\lambda$(a : A).B){ $\lambda$ ... DepConstr(0, B) ..., ... }(@\vspace{-0.04cm}@)
g : B $\rightarrow$ A := DepElim(b, $\lambda$(b : B).A){ $\lambda$ ... DepConstr(0, A) ..., ... }
\end{lstlisting}
In addition, we need that:

\begin{enumerate}
\item $\forall$ \lstinline{j}, \lstinline{DepConstr(j, A)} $\equiv_{A \simeq B}$ \lstinline{DepConstr(j, B)}, and
\item $\forall$ \lstinline{(a : A) (b : B) (P : A $\rightarrow$ Type) (Q : B $\rightarrow$ Type)},\\ \lstinline{a} $\equiv_{A \simeq B}$ \lstinline{b} $\rightarrow$ \lstinline{P} $\equiv_{A \simeq B}$ \lstinline{Q} $\rightarrow$ \lstinline{DepElim(a, P)} $\equiv_{A \simeq B}$ \lstinline{DepElim(b, Q)}.
\end{enumerate}
This had previously been proven on the change from \lstinline{list T} to $\Sigma$\lstinline{(n : nat).vector T n)} in the 
univalent parametricity framework\footnote{\url{https://github.com/CoqHott/univalent_parametricity/commit/7dc14e69942e6b3302fadaf5356f9a7e724b0f3c}}---we build on that intuition.
That these are equal up to transport along the equivalence means that replacing dependent constructors (respectively eliminators) of \A
with dependent constructors (respectively eliminators) of \B will preserve equality up to transport for those subterms.
Furthermore, since CIC$_{\omega}$ is constructive, the \textit{only} way to construct an \A (respectively \B) is to use its constructors,
and the \textit{only} way to eliminate an \A (respectively \B) is to apply its eliminator.
Finally, since these form an equivalence, all ways of constructing or eliminating \A and \B are covered by these dependent constructors and eliminators.
So, as long as we are able to expand all implicit applications of \lstinline{DepConstr} and \lstinline{DepElim},
\textsc{Dep-Constr} and \textsc{Dep-Elim} should preserve correctness of the transformation and cover all values and eliminations of \A and \B.

To ensure coherence with equality, we need \lstinline{Eta} and \lstinline{Iota} to correctly prove the $\eta$ and $\iota$ rules.
For \lstinline{Eta}, we need it to have the same definitional behavior as the dependent eliminator:

\begin{lstlisting}
DepElim(a, P) { f$_0$, ..., f$_n$ } : P (Eta(A) a)
\end{lstlisting}
and similarly for \B.

Each \lstinline{Iota} needs to prove and rewrite along the simplification or refolding behavior that corresponds to a case of the dependent eliminator, in other words: % TODO do we need eta here?

\begin{lstlisting}
Iota(A) :(@\vspace{-0.04cm}@)
  $\forall$ P $\vec{f}$ $\vec{x}$ (Q : P (DepConstr(j, A) $\vec{x}$) $\rightarrow$ Type),(@\vspace{-0.04cm}@)
    Q (DepElim((@\codediff{DepConstr(j, A) $\vec{x}$}@), P) $\vec{f}$) $\rightarrow$ (@\vspace{-0.04cm}@)
    Q ((@\codediff{$\vec{f}$[j]}@) ... (DepElim(IH$_0$, P) $\vec{f}$) ... (DepElim(IH$_n$, P) $\vec{f}$) ...)
\end{lstlisting}
where each \lstinline{IH}$_i$ is each recursive occurrence of \A in the eliminator case,
and similarly for \B.
Together, these induce proofs of \lstinline{section} and \lstinline{retraction} (by induction) that show \lstinline{f} and \lstinline{g}
are mutual inverses and so form an equivalence.
The intuition here is that preserving the reduction behavior
of the eliminators and constructors should be sufficient, since again those are the only ways we can construct or eliminate our types.

\textbf{Configure} implements search procedures that discover functions \lstinline{f} and \lstinline{g} for certain classes of
equivalences, and also
generate proofs \lstinline{section} and \lstinline{retraction} that these functions form an equivalence.
Proving all of the correctness criteria for a given configuration requires either a special framework~\cite{tabareau2017equivalences}
or a univalent type theory~\cite{univalent2013homotopy}.
Thankfully, the proof engineer does not need to prove the correctness criteria for a configuration in order to use \toolname.
Rather, the correctness criteria simply need to hold in order for the transformation to work.

\subsection{The Tool}
\label{sec:implementation}

The configurable proof term transformation helped us build a flexible proof repair and reuse tool.
However, it alone was not enough to build a tool that reaches real proof engineers.
This section describes a sample of the implementation challenges that we encountered and how we solved them.
Section~\ref{sec:discussion} elaborates on the remaining challenges and our plans to address them in the future.

\paragraph{From CIC$_{\omega}$ to Coq}

We must handle language differences to scale from CIC$_{\omega}$ to Coq.
We use an existing command called \lstinline{Preprocess}~\cite{Ringer2019} to turn pattern matching and fixpoints into 
applications of eliminators.
We handle non-primitive projections using \lstinline{Eta}, and we handle refolding of constants in constructors
using \lstinline{DepConstr}.

\paragraph{Matching Against Preconditions}

%It is easy to \textit{describe} the proof term transformation, but it is much more difficult to implement it.
%This is because
The proof term transformation only describes what transformation rules are applicable when,
but it does not describe how to actually check that the precondition holds.
In many cases, this check is not purely syntactic---we really want to know if a term \textit{unifies}
with an application of \lstinline{DepConstr}, for example, not whether it applies the term exactly.
This is especially pronounced with definitional \lstinline{Eta} and \lstinline{Iota},
which typically show up contracted in real code.
This problem is exactly why \citet{tabareau2019marriage} speculated that converting definitional to propositional equalities
like we do with \lstinline{Iota} may, in general, be intractable.

In practice, we find that unification is sometimes not enough to identify an implicit application of one of the configuration terms.
Each of our search procedures for automatic configuration in turn implements custom unification heuristics that tell \toolname
how to identify and expand these implicit applications before applying the transformation.
In the worst case, proof engineers can provide hints to \toolname in the form of explicit annotations.

\paragraph{Termination \& Intent}

Another challenge with implementing the proof term transformation is deciding whether to run a rule that matches at all.
That is, when the correctness criteria for a configuration hold and a subterm matches a rule, this suggests that \toolname \textit{can}
run the transformation rule, but it does not necessarily mean that it \textit{should}.
In some cases, repeatedly running a matching transformation rule would result in nontermination.
For example, if \B is a refinement of \A, then we can always run \textsc{Equivalence}
over and over again, forever.
%\textsc{Devoid} ruled out this case by simply prohibiting the case where \B refers to \A, but we found it sometimes
%useful to support this case.
We thus include some simple termination checks in our code.

Even when termination is guaranteed, whether to run a matching transformation rule
depends on intent.
For example, our industrial proof engineer sometimes wished to port only some occurrences of \A,
especially when \A was a tuple that could appear elsewhere
with a different meaning.
\toolname has some support for this using an interactive workflow.
%We helped the proof engineer do this by interacting with \toolname using a particular workflow.
%We plan to support this automatically using type-directed search in the future.

\paragraph{Reaching Real Proof Engineers}
Many of our design decisions in implementing \toolname were informed by our partnership with
an industrial proof engineer (see Section~\ref{sec:industry}).
For example, we found that the proof engineer rarely had the patience to wait more than ten seconds
for \toolname to port a function or proof.
In response, we implemented aggressive caching (with an option to disable the cache), even caching intermediate subterms that
we encounter in the course of running our proof term transformation.
We also added an option to tell \toolname not to $\delta$-reduce certain terms. %or recurse into certain modules.
% set certain terms or modules as opaque to \toolname, to prevent unnecessary $\delta$-reduction.

The experiences of proof engineers also informed features that we exposed.
For example, we implemented special search procedures to generate custom eliminators to make it easier to reason about
types refined by equalities like $\Sigma$\lstinline{(l : list T).length l = n}
by breaking them into parts and reasoning separately about the projections.
These features along with our tactic decompiler helped with integration into proof engineering workflows.

%First we need that \lstinline{DepElim} over $A$ into \lstinline{DepConstr} over $B$ and \lstinline{DepElim} over $B$ into
%\lstinline{DepConstr} over $A$ form an equivalence between $A$ and $B$. When that's true, I think it should hold that \lstinline{DepElim} over $A$
%and \lstinline{DepElim} over $B$ are in univalent relation with one another. If not, then that's an extra condition.
%Finally, we need the transformation to preserve definitional equalities. Not sure about the general case, but for vectors and lists,
%we need:

%\begin{lstlisting}
%  $\forall$ A l (f : $\forall$ (l : sigT (Vector.t A)), l = l),
%    vect_dep_elim A (fun l => l = l) (f nil) (fun t s _ => f (cons t s)) l = f (id_eta l).
%\end{lstlisting}
%and:

%\begin{lstlisting}
%Definition elim_id (A : Type) (s : {H : nat & t A H}) :=
%  vect_dep_elim
%    A
%    (fun _ => {H : nat & t A H})
%    nil
%    (fun (h : A) _ IH =>
%      cons h IH)
%    s.

% $\forall$ A h s,
%    exists (H : cons h (elim_id A s) = elim_id A (cons h s)),
%      H = eq_refl.
%\end{lstlisting}
%More generally, for each constructor index $j$, define:

%\begin{lstlisting}
%  eqc (j, B) (f : $\forall$ b : B, b = b) :=
%    fun ... (* TODO get the hypos from the type of the eliminator *) =>
%      f (DepConstr (j, B)) (* TODO args *)%%

  %elim_id := (* TODO *)
%\end{lstlisting}
%Then we need:

%\begin{enumerate}
%\item $\forall b f, \mathrm{DepElim}(b,\ p_{b}) \{\mathrm{eqc} (1, B) f, \ldots, \mathrm{eqc} (n, B) f\} = f (\mathrm{Eta}(A) a) $
%\item Something relating the constructors and \lstinline{elim_id} to reflexivity
%\end{enumerate}
%and similarly for $A$.

%Really the point of these conditions is that from them, with some restrictions on input terms, we can get
%that lifting terms gives us the same type that we'd get from lifting the type. But there are still
%some restrictions (see the few that fail).

%It's probably not always possible to define these three things for every equivalence.
%Could generalize by rewriting. But this lets us avoid the rewriting problem from Nicolas' paper.

% TODO how does this get us something like primitive projections? Just makes Eta definitionally equal to regular Id?

% TODO so we can probably just frame search in terms of DepConstr and DepElim and then generate proofs about this on an ad-hoc basis
% and get away with not including the specific details of our instantiations. We can give examples instead, give intuition, and say we generate
% the proofs in Coq

%For the second one we need not just an eliminator rule but also an identity rule.
%DEVOID assumed primitive projections which let them get away without thinking of this,
%but then had this weirdly ad-hoc ``repacking'' thing in their implementation.
%It turns out this is just a more general identity rule, which basically says what
%the identity function should lift to so that the transformation preserves definitional equalities.
%Actually deciding when to run this rule is one of the biggest challenges in practice,
%so we'll talk about that more in the implementation section.


\section{Decompiling Proof Terms to Tactics}
\label{sec:decompiler}

\textbf{Transform} produces a proof term,
while the proof engineer typically writes and maintains proof scripts made up of tactics.
We improve user experience thanks the realization that, since Coq's proof term language Gallina is very structured,
we can decompile these Gallina terms to Ltac proof scripts for the proof engineer to maintain. In other words:

\begin{quote}
\textbf{Insight 3}: The transformed proof terms can then be translated back to tactics.
\end{quote}

The \textbf{Decompile} component implements a prototype of this translation.
This decompiler prototype has shown early promising results.
For example, it produced the automatically generated tactic proof for \codeauto{\lstinline{rev_app_distr}} 
in Section~\ref{sec:overview}, as well as the tactic proofs of \lstinline{section}
and \lstinline{retraction} in Section~\ref{fig:equivalence}.

The output language for the implementation of \textbf{Decompile} is Ltac, the proof script language for Coq.
Ltac can be confusing to reason about, since Ltac tactics can refer to Gallina terms, and the semantics of Ltac depends both on the
semantics of Gallina and on the implementation of proof search procedures written in OCaml.
To give a sense of how the decompiler works without the clutter of these proof search details, we start by defining a toy
decompiler from CIC$_{\omega}$ to a simple subset of Ltac containing just a few predefined tactics (Section~\ref{sec:first}).
We then explain how we scale that up to the actual implementation (Section~\ref{sec:second}).

\subsection{A Toy Decompiler}
\label{sec:first}

\begin{figure}
\small
\begin{grammar}
<v> $\in$ Vars, <t> $\in$ CIC$_{\omega}$

<p> ::= intro <v> |  rewrite <t> <t> | symmetry | apply <t> | induction <t> <t> \{ <p>, \ldots, <p> \} | split \{ <p>, <p> \} | left | right | <p> . <p>
\end{grammar}
\caption{Qtac syntax.}
\label{fig:ltacsyntax1}
\end{figure}

The toy decompiler takes CIC$_{\omega}$ terms and produces tactics in a toy version of Ltac which we call Qtac.\footnote{Pronounced \textit{cute-tac}.}
The syntax for Qtac is in Figure~\ref{fig:ltacsyntax1}.
Qtac includes hypothesis introduction (\lstinline{intro},
rewriting by equalities (\lstinline{rewrite}), symmetry (\lstinline{symmetry}) of equality,
application of a term to prove the goal (\lstinline{apply}), induction over terms (\lstinline{induction}),
case splitting of conjunctions (\lstinline{split}),
constructors of disjunctions (\lstinline{left} and \lstinline{right}), and
composition (\lstinline{.}).

Unlike in Ltac, in Qtac, \lstinline{induction} and \lstinline{rewrite} always take a motive explicitly, rather than relying on a unification engine.
Smilarly, \lstinline{apply} applies only the function without inferring any arguments, and leaves those arguments to proof obligations.
The implementation reasons about Ltac and so does not make these assumptions.

\begin{figure}
\begin{mathpar}
\mprset{flushleft}
\small
\hfill\fbox{$\Gamma$ $\vdash$ $t$ $\Rightarrow$ $p$}\\

\inferrule[Intro]
  { \Gamma,\ n : T \vdash b \Rightarrow p }
  { \Gamma \vdash \lambda (n : T) . b \Rightarrow \mathrm{intro}\ n.\ p }

\inferrule[Symmetry]
  { \Gamma \vdash H \Rightarrow p }
  { \Gamma \vdash \mathtt{eq\_sym}\ H \Rightarrow \mathrm{symmetry}.\ p } \\

\inferrule[Split]
  { \Gamma \vdash l \Rightarrow p \\ \Gamma \vdash r \Rightarrow q }
  { \Gamma \vdash \mathrm{Constr}(0,\ \wedge)\ l r \Rightarrow \mathrm{split} \{ p, q \}.\ }

\inferrule[Left]
  { \Gamma \vdash H \Rightarrow p }
  { \Gamma \vdash \mathrm{Constr}(0,\ \vee)\ H \Rightarrow \mathrm{left}.\ p }

\inferrule[Right]
  { \Gamma \vdash H \Rightarrow p }
  { \Gamma \vdash \mathrm{Constr}(1,\ \vee)\ H \Rightarrow \mathrm{right}.\ p } \\

\inferrule[Rewrite]
  { \Gamma \vdash H_1 : x = y \\ \Gamma \vdash H_2 \Rightarrow p }
  { \Gamma \vdash \mathrm{Elim}(H_1,\ P) \{ x,\ H_2,\ y \} \Rightarrow \mathrm{symmetry}.\ \mathrm{rewrite}\ P\ H_1.\ p }

\inferrule[Induction]
  { \Gamma \vdash \vec{f} \Rightarrow \vec{p} }
  { \Gamma \vdash \mathrm{Elim}(t,\ P)\ \vec{f} \Rightarrow \mathrm{induction}\ P\ t\ \vec{p} }

\inferrule[Apply]
  { \Gamma \vdash t \Rightarrow p }
  { \Gamma \vdash f t \Rightarrow \mathrm{apply}\ f.\ p }

\inferrule[Base]
  { \\ }
  { \Gamma \vdash t \Rightarrow \mathrm{apply}\ t }
\end{mathpar}
\caption{Qtac decompiler semantics.}
\label{fig:someantics}
\end{figure}

The semantics for the toy decompiler are in Figure~\ref{fig:someantics} (assuming $=$, \lstinline{eq_sym}, $\wedge$, and $\vee$ are defined as in Coq).
This decompiler works like the real decompiler: it accepts a proof term and generates a candidate proof script that attempts to prove the same theorem.
As with the real decompiler, the baseline for success of the toy decompiler is the naive proof script
that applies the entire proof term with the \lstinline{apply} tactic.
Such a proof script will always work, but will often be unreadable.
The decompiler defaults to this baseline behavior (\textsc{Base}).

Otherwise, the goal of the decompiler is to improve on that baseline as much as possible,
or else produce a candidate proof script that is close enough that the proof engineer can manually massage it into something that
both works and is maintainable.
It does this by recursing over the proof term and constructing a proof script using a predefined set of tactics.

For the toy decompiler, this is fairly straightforward: Lambda terms become introduction of hypotheses (\textsc{Intro}), since they introduce new bindings
in the environment of the body. Applications of \lstinline{eq_sym} become symmetry of equality (\textsc{Symmetry}).
Constructors of conjunction and disjunction become map to the respective tactics (\textsc{Split}, \textsc{Left}, and \textsc{Right}).
Applications of equality eliminators compose symmetry (to orient the rewrite direction with the goal) with rewrites (\textsc{Rewrite}),
and all other applications of eliminators become induction (\textsc{Induction}).
The remaining applications become apply tactics (\textsc{Apply}).
In all cases, the decompiler recurses on the remaining body, breaking into cases when relevant, until no other preconditions match.
At that point the \textsc{Base} case holds, and we are done.

While the toy compiler is very simple, only a few simple changes are needed
to move this from CIC$_{\omega}$ to Coq.
Furthermore, the result can already handle some of the example proofs \toolname has produced.
The generated proof term of \codeauto{\lstinline{rev_app_distr}} with swapped list constructors from Section~\ref{sec:overview},
for example, consists only of induction, rewriting, simplification, and reflexivity.
The proof term for the base case:

\begin{lstlisting}
fun (@\codesimb{(y0 : list A)}@) =>
  (@\codesima{list_rect}@) _ _
    (fun (@\codesima{a l IHl}@) =>
      (@\codesimc{eq_ind_r}@) _ (@\codesimd{eq_refl}@) (@\codesimc{(app_nil_r (rev l) (a::[]))}@))
    (@\codesime{eq_refl}@)
    (@\codesima{y0}@)
\end{lstlisting}
decompiles to this proof script:

\begin{lstlisting}
- (@\codesimb{intro y0.}@) (@\codesima{induction y0 as [a l IHl|]}.@)
  + (@\codesimc{simpl. rewrite (app_nil_r (rev l) (a::[])).}@) (@\codesimd{reflexivity.}@)
  + (@\codesime{reflexivity.}@)
\end{lstlisting}
where corresponding terms and tactics are highlighted with the same color, and nothing else is highlighted for clarity.
There are very few differences from the toy decompiler needed to produce this,
for example handling of rewrites in both directions (\lstinline{eq_ind_r} as opposed to \lstinline{eq_ind}),
simplifying to handle motive inference for rewrites,
and turning applications of \lstinline{eq_refl} into reflexivity.

In fact, since \toolname uses the \lstinline{Preprocess} command from \textsc{Devoid}, \textit{all} of the proof terms that \toolname
produces will use induction and rewriting rather than fixpoints and pattern matching.
Because we have control over output terms, even a toy decompiler gets us pretty far.

% TODO add any new things RanDair implements, like exists

\subsection{Scaling Up}
\label{sec:second}

The toy decompiler abstracts a lot of the details that make Ltac so useful to proof engineers---and so painful to 
reason about automatically.
This section discusses how we scale up from this toy decompiler to a prototype Gallina to Ltac decompiler,
and how we imagine the decompiler continuing to evolve from there.

\paragraph{Second Pass}
The toy decompiler reasons about tactics one subterm at a time, and produces tactics only from a predefined set of tactics.
This does not always match the thought processes of proof engineers.
To produce a more natural set of tactics, \textbf{Decompile} operates in two passes: first it runs something that looks a lot
like the toy decompiler, and then it modifies those tactics to produce a more natural proof script.
For example, the first pass, like the toy decompiler, produces a single \lstinline{intro} per hypothesis in a lambda abstraction;
the second pass combines each sequence of \lstinline{intro} tactics into an \lstinline{intros} tactic.

The prototype decompiler still has no special reasoning for decision procedures and custom tactics, which are common
in some proof engineering styles.
This second pass will be the natural integration point for these.
Our current plan is to take additional input in this pass, either directly from the proof engineer
or by feeding in the original proof script from before \textbf{Transform}.
We can then iteratively replace tactics with custom tactics and decision procedures, and check the result see if it works.
We also plan to support tacticals like \lstinline{;} and \lstinline{try}.
For now, massaging the output to use these is left to the proof engineer.

\paragraph{Induction and Rewriting}
The toy decopmiler includes simpler and more predictable versions of \lstinline{rewrite} and \lstinline{induction}
than those found in Coq. The implementation of \textbf{Decompile} includes additional logic to reason about these tactics.
For example, it assumes that there is only one \lstinline{rewrite} direction. Coq has two rewrite directions,
and so the decompiler infers the right direction based on the motive used.

It also assumes that both tactics take the inductive motive explicitly.
In Coq, however, both tactics infer the motive automatically.
Consequentially, Coq will sometimes infer the wrong motive without manipulation of goals and hypotheses,
or will fail to infer a motive at all.
This is especially common for the \lstinline{rewrite} tactic, which is purely syntactic.
To handle induction, the decompiler strategically use the \lstinline{revert} tactic to manipulate the goal
so that Coq can better infer the motive.
To handle rewrites, it uses the \lstinline{simpl} tactic to refold the goal before rewriting.
Neither of these approaches are guaranteed to work, so the proof engineer may sometimes need to tweak the output proof script appropriately.
We have found that even if we pass Coq's induction principle an explicit motive, Coq still sometimes fails due
to unrepresented assumptions.
Long term, using another tactic like \lstinline{change} to manipulate hypotheses and goals before applying these tactics
may help with cases for which Coq cannot infer the correct motive.

\paragraph{Manipulating Hypotheses}
Changing from Qtac to Ltac is not the only challenge in writing the decompiler---we also scale from CIC$_{\omega}$ to Coq.
This introduces let bindings, which are generated by tactics like \lstinline{rewrite in}, \lstinline{apply in}, and \lstinline{pose}.
\textbf{Decompile} implements support for \lstinline{rewrite in} and \lstinline{apply in} similarly to how it implements support for
\lstinline{rewrite} and \lstinline{apply}, but with two differences:

\begin{enumerate}
\item it ensures that the unmanipulated hypothesis does not occur in body of the let expression,
\item it swaps the direction of the rewrite, and
\item it checks for generated subgoals and recurses into those subgoals.
\end{enumerate}
In all other cases, the implementation uses the \lstinline{pose} tactic, a catch-all for let bindings.

\paragraph{Pretty Printing}
After decompiling proof terms, there is one final step to present the information to the proof engineer: pretty printing.
Like the toy decompiler, the implementation of \textbf{Decompile} represents its output language using a predefined grammar of Ltac tactics,
albeit one that is larger than Qtac.
It maintains the internal recursive proof structure as it goes, then uses that proof structure to print proofs of subgoals using bullet points.
It displays the resulting proof script to the proof engineer, who can then modify it as needed to ensure that it works correctly
and is maintainable.
For convenience, it includes scripts that automate the process of printing all of these tactic proofs to a Coq file,
in case the proof engineer does not want such an interactive workflow.
\toolname keeps all output proof terms from the proof term transformation in the Coq environment as a fallback in case the decompiler does not succeed.
Once the proof engineer has this new proof, she can remove the old specifications, functions, and proofs, using the refactored or repaired
versions from then on.



\section{Search}
\label{sec:search}


\section{Related Work}
\label{sec:related}

% TODO check survey paper for more stuff/better stuff
% TODO also remove any content that is too similar with survey paper

We discuss related work in proof repair, proof refactoring, proof reuse, and proof design.
More can be found in a recent survey of proof engineering~\cite{PGL-045}.

\paragraph{Proof Repair}

%\toolname is not the first proof repair tool.
\textsc{Pumpkin Patch}~\cite{pumpkinpatch}, like \toolname, is also a proof repair tool for Coq.
The search procedures in \textbf{Configure} are based partly on ideas from \textsc{Pumpkin Patch}, which includes
similar search procedures for discovering patches to fix broken proofs.
Unlike \toolname, \textsc{Pumpkin Patch} does not apply the patches that it finds,
handle changes in structure, or include support for tactics beyond the use of hints.
\toolname addresses these limitations.

Proof repair can be viewed as a form of \textit{program repair}~\cite{Monperrus:2018:ASR:3177787.3105906, Gazzola:2018:ASR:3180155.3182526}
specific to the domain of proof assistants.
Proof assistants like Coq are a good fit for program repair: A recent paper~\cite{Qi:2015:APP:2771783.2771791} 
recommends that program repair tools draw on extra information
such as specifications or example patches. In Coq, specifications and examples 
are rich and widely available: specifications thanks to dependent types,
and examples thanks to constructivism.

\paragraph{Proof Refactoring}

Proof repair is closely related to proof refactoring~\cite{WhitesidePhD}. 
%and a number of the changes that \toolname supports can be viewed as refactorings.
The proof refactoring tool Levity~\cite{Bourke12} for Isabelle/HOL has seen large-scale industrial use.
Levity focuses on the specific task of moving lemmas,
whereas \toolname focuses on evolving proofs in the presence of a flexible notion of type equivalences.
Chick~\cite{robert2018} and RefactorAgda~\cite{wibergh2019} are proof refactoring tools that
also support a few changes that can be viewed as repairs~\cite{PGL-045}.
%Chick operates over a Gallina-like language, while RefactorAgda is implemented in Agda.
Unlike \toolname, these tools support primarily syntactic changes and do not have tactic support.
% changes these tools support are still primarily syntactic,
%and neither of these tools have tactic support.

A few proof refactoring tools operate directly over tactics:
POLAR~\cite{Dietrich2013} refactors proof scripts in languages based on Isabelle/Isar~\cite{Wenzel2007isar},
CoqPIE~\cite{Roe2016} is an IDE with support for simple refactorings of Ltac scripts, and
Tactician~\cite{adams2015} is a proof script refactoring tool that focuses on switching between tactics and tacticals.
This approach is not tractable for handling more complex changes;
\citet{robert2018} discusses the challenges in detail.

\paragraph{Proof Reuse}

%Proof repair is proof reuse with the additional constraint that one specification ceases to exist.
A few proof reuse tools work by proof term transformation and so can be repurposed for proof repair.
The namesake of this paper is based on a proof reuse tool~\cite{Johnsen2004}
that generalizes theorems in Isabelle/HOL.
\toolname's proof term transformation generalizes the proof term transformation from \textsc{Devoid}~\cite{Ringer2019},
which transforms proofs along the specific class of changes called algebraic ornaments~\cite{mcbride}.
\citet{magaud2000changing} implements a proof term transformation for translating proofs between
unary and binary natural numbers. 
The latter two of these tools fit into configurations for \toolname,
and none implements tactic support in Coq like \toolname does.
The expansion algorithm from \citet{magaud2000changing} may further improve \toolname
by automating some of the manual steps in expanding \lstinline{Iota}.

The \toolname proof term transformation implements transport across equivalences.
Transport is realizable as a function in univalent type theories~\cite{univalent2013homotopy}.
The univalent parametricity framework~\cite{tabareau2017equivalences} realizes univalent transport for certain types
in a non-univalent type theory, only sometimes relying on additional axioms beyond the core type theory of Coq.
While powerful, neither of these approaches to transport remove references to the old type, making them poorly suited for repair.

Recent work~\cite{tabareau2019marriage} extends univalent parametricity with 
a white-box transformation that may be well suited for proof repair.
However, it relies on proof obligations from the proof engineer beyond those imposed by \toolname.
%that establish what is effectively the correctness criteria
%for the configuration in \toolname, while \toolname needs only that it holds metatheoretically.
In addition, it does not include search procedures like \toolname to discover new equivalences from types,
and it does not include tactic script generation like \toolname.
Finally, it does not implement any support for porting definitional equalities to propositional equalities,
instead relying on the original black-box functionality of the univalent parametricity framework to handle those cases;
the \lstinline{Iota} rule addresses this problem in \toolname and is based on lessons learned from reading that article.
The most fruitful progress may come from integrating these tools together to take advantage of the benefits that both offer.

The univalent parametricity framework implements type-directed search, a feature that \toolname does not yet support.
The framework achieves this using type classes~\cite{Sozeau2008}; this does not always scale well~\cite{tabareau2019marriage}.
Both \toolname and univalent parametricity could benefit from implementing type-directed search using e-graphs~\cite{egraph1}.
Of particular interest are those developed for congruence in Cubical Agda~\cite{egraph6},
which prove the theorem \lstinline{hcongr_ideal}~\cite{egraph7} necessary to use e-graphs not derivable in CIC$_{\omega}$,
and which should allow for efficient and elegant automatic transport.

\paragraph{Proof Design}

Much of the proof engineering work focuses on designing proofs
to be robust to change, rather than fixing already broken proofs.
This can take the form of design principles, like using 
information hiding techniques~\cite{Woos:2016:PCF:2854065.2854081, Klein:2014:CFV:2584468.2560537}
or any of the structures~\cite{Chrzaszcz2003, Sozeau2008, Saibi:PhD} for encoding interfaces in Coq.
%thereby localizing the burden of change to the interface.
Design and repair are complementary: design requires foresight, whereas repair can be applied retroactively.
Repair can help with changes that occur outside of the proof engineer's control,
or with changes that are difficult to protect against even with informed design.

Another approach to this is to use heavy proof automation, for example through
program-specific proof automation~\cite{Chlipala:2013:CPD:2584504}
%implementations of decision procedures~\cite{Pugh1991},
or general-purpose hammers~\cite{Blanchette2016b, Blanchette2013, Kaliszyk2014, Czajka2018}.
The degree to which proof engineers rely on automation varies, as seen in the data from the
\textsc{REPLica} user study of Coq proof engineers~\cite{replica}.
Automation-heavy proof engineering styles localize the burden of change to the automation itself,
but can result in terms that are large and slow to type check,
and tactics that can be difficult to debug.
While these approaches are also complementary, more work will be needed for \toolname to better support 
proof developments in this style.




\section{Conclusions \& Future Work}
\label{sec:discussion}

We showed how to combine a configurable proof term transformation with a tactic decompiler to build \toolname,
a proof repair and reuse tool that is flexible and useful for real proof engineering scenarios.
\toolname has helped an industrial proof engineer integrate Coq with a company workflow,
and has supported benchmarks common in the proof engineering community.

Moving forward, our goal is to make proofs easier to repair and reuse regardless of proof engineering expertise.
We want to reach more proof engineers, and we want \toolname to integrate seamlessly with Coq.
We conclude with a discussion of some of the challenges that remain.

% encountered in scaling up the \toolname proof term 
%transformation (Section~\ref{sec:problems}), and how we believe ideas from the rewrite system and constraint
%solver communities can address those challenges and improve the state of the art in proof engineering (Section~\ref{sec:egraph}).
%Our hope is to inspire research communities to come together and open the door to better tools for proof reuse and repair.

\paragraph{Future Work}

We encountered three challenges in scaling up the \toolname transformation:

\begin{enumerate}
\item \textbf{Multiple Equivalences}: Deciding when to run the transformation rules is left to the implementation.
\toolname automates the most basic case of this: changing \textit{every} occurrence of \A to \B.
This can lead to confusing or undesired behavior, especially when there are multiple matching equivalences for a subterm.
The ideal would be a type-directed search procedure.
\item \textbf{Nontermination}: Naively applying the transformation can result in nontermination when the output type refers to the input type.
\toolname includes termination checks, but these termination checks are ad-hoc and do not capture every potentially nonterminating use case.
\item \textbf{Custom Unification Heuristics}: \toolname with manual configuration is not always smart enough to match the appropriate rule in the proof term
transformation without the proof engineer manually expanding the input term.
Proof engineers would benefit from the ability to add custom unification heuristics to \toolname
and expand the input terms automatically.
\end{enumerate}
We hope to solve all three of these challenges elegantly and efficiently using \textit{e-graphs}~\cite{egraph1},
a data structure that is used in the constraint solver and rewrite system communities for managing equivalences.
E-graphs are built specifically to deal with multiple equivalences,
remove the burden of ad-hoc reasoning about termination,
and make it simple for anyone to extend a system with new
rewrite rules---even ones that can call out to external procedures~\cite{egraph5} 
like unification heuristics implemented in OCaml.
The one hurdle is to adapt them to use a univalent definition of equality.
In cubical type theory, this adaptation is simple~\cite{egraph6}; we hope to repurpose this insight.

Beyond that, we believe that the biggest gains will come from continuing to improve the prototype decompiler.
Two particularly helpful features would be support for common search procedures and support for custom tactics.
We hope to use user input to guide this process, as analyzing Ltac directly is unlikely to be fruitful.
Some improvements could come from better tactics themselves---like better handling of explicitly passed 
motives in the \lstinline{induction} tactic, or a more structured tactic language.
With that, we believe that the future of seamless and powerful proof repair and reuse for all is within reach.
We hope you will join us in bringing it to life.





%% Acknowledgments
\begin{acks}                            %% acks environment is optional
                                        %% contents suppressed with 'anonymous'
  %% Commands \grantsponsor{<sponsorID>}{<name>}{<url>} and
  %% \grantnum[<url>]{<sponsorID>}{<number>} should be used to
  %% acknowledge financial support and will be used by metadata
  %% extraction tools.
  This material is based upon work supported by the
  \grantsponsor{GS100000001}{National Science
    Foundation}{http://dx.doi.org/10.13039/100000001} under Grant
  No.~\grantnum{GS100000001}{nnnnnnn} and Grant
  No.~\grantnum{GS100000001}{mmmmmmm}.  Any opinions, findings, and
  conclusions or recommendations expressed in this material are those
  of the author and do not necessarily reflect the views of the
  National Science Foundation.
\end{acks}


%% Bibliography
\bibliography{paper.bib}


%% Appendix
%\appendix
%\section{Appendix}

%Text of appendix \ldots

\end{document}
