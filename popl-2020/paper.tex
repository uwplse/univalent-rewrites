%% For double-blind review submission, w/o CCS and ACM Reference (max submission space)
\documentclass[acmsmall]{acmart}\settopmatter{printfolios=true,printccs=false,printacmref=false}
%% For double-blind review submission, w/ CCS and ACM Reference
%\documentclass[acmsmall,review,anonymous]{acmart}\settopmatter{printfolios=true}
%% For single-blind review submission, w/o CCS and ACM Reference (max submission space)
%\documentclass[acmsmall,review]{acmart}\settopmatter{printfolios=true,printccs=false,printacmref=false}
%% For single-blind review submission, w/ CCS and ACM Reference
%\documentclass[acmsmall,review]{acmart}\settopmatter{printfolios=true}
%% For final camera-ready submission, w/ required CCS and ACM Reference
%\documentclass[acmsmall]{acmart}\settopmatter{}


%% Journal information
%% Supplied to authors by publisher for camera-ready submission;
%% use defaults for review submission.
\acmJournal{PACMPL}
\acmVolume{1}
\acmNumber{CONF} % CONF = POPL or ICFP or OOPSLA
\acmArticle{1}
\acmYear{2018}
\acmMonth{1}
\acmDOI{} % \acmDOI{10.1145/nnnnnnn.nnnnnnn}
\startPage{1}

%% Copyright information
%% Supplied to authors (based on authors' rights management selection;
%% see authors.acm.org) by publisher for camera-ready submission;
%% use 'none' for review submission.
\setcopyright{none}
%\setcopyright{acmcopyright}
%\setcopyright{acmlicensed}
%\setcopyright{rightsretained}
%\copyrightyear{2018}           %% If different from \acmYear

%% Bibliography style
\bibliographystyle{ACM-Reference-Format}
%% Citation style
%% Note: author/year citations are required for papers published as an
%% issue of PACMPL.
\citestyle{acmauthoryear}   %% For author/year citations


%%%%%%%%%%%%%%%%%%%%%%%%%%%%%%%%%%%%%%%%%%%%%%%%%%%%%%%%%%%%%%%%%%%%%%
%% Note: Authors migrating a paper from PACMPL format to traditional
%% SIGPLAN proceedings format must update the '\documentclass' and
%% topmatter commands above; see 'acmart-sigplanproc-template.tex'.
%%%%%%%%%%%%%%%%%%%%%%%%%%%%%%%%%%%%%%%%%%%%%%%%%%%%%%%%%%%%%%%%%%%%%%


%% Some recommended packages.
\usepackage{booktabs}   %% For formal tables:
                        %% http://ctan.org/pkg/booktabs
\usepackage{subcaption} %% For complex figures with subfigures/subcaptions
                        %% http://ctan.org/pkg/subcaption

\usepackage{enumerate} % for lists
\usepackage{listings} % for code
\usepackage{xspace} % so we don't need to figure out spacing after \toolname every time
\usepackage{mathpartir} % for inference rules
\usepackage{scalerel} % to render definitional equality
\usepackage{bbm} % to render N for the natural numbers
\usepackage{syntax} % for code highlighting
\usepackage{xcolor} % for code highlighting
\usepackage{multirow} % for tables

% Nice rendering of Coq code
\lstdefinelanguage{coq}{
    keywords={Repair, module, Theorem, Proof, Record, Lemma, Definition, Abort, Qed, forall, Inductive, Type, Prop, Set, fun, fix, forall, Require, Import, Fixpoint, match, end, with, as, return, struct, Qed, Defined, let},
    basicstyle=\linespread{0.95}\small\ttfamily,
    keywordstyle=\color{blue},
    commentstyle=\itshape\rmfamily,
    showstringspaces=false,
    columns=flexible,
    breaklines=true,
    texcl=true,
    mathescape=true,
    tabsize=4,
    stringstyle=\color{brown},
    escapeinside={(@}{@)},
}

\lstset{language=coq} % default

\newcommand\toolname{\textsc{Carrot}\xspace} % tool name
\newcommand\company{Company\xspace} % company name
\newcommand\A{$A$\xspace} % recurring type A
\newcommand\B{$B$\xspace} % recurring type B
\newcommand{\reducedstrut}{\vrule width 0pt height .9\ht\strutbox depth .9\dp\strutbox\relax} % for \codediff and \codeauto
\newcommand{\codediff}[1]{%
  \begingroup
  \setlength{\fboxsep}{0pt}%
  \colorbox{orange!25}{\reducedstrut#1\/}%
  \endgroup
} % to highlight the difference between two code blocks
\newcommand{\codesima}[1]{%
  \begingroup
  \setlength{\fboxsep}{0pt}%
  \colorbox{orange!25}{\reducedstrut#1\/}%
  \endgroup
} % to highlight the similarities between two code blocks
\newcommand{\codesimb}[1]{%
  \begingroup
  \setlength{\fboxsep}{0pt}%
  \colorbox{red!25}{\reducedstrut#1\/}%
  \endgroup
} % to highlight the similarities between two code blocks
\newcommand{\codesimc}[1]{%
  \begingroup
  \setlength{\fboxsep}{0pt}%
  \colorbox{yellow!25}{\reducedstrut#1\/}%
  \endgroup
} % to highlight the similarities between two code blocks
\newcommand{\codesimd}[1]{%
  \begingroup
  \setlength{\fboxsep}{0pt}%
  \colorbox{green!25}{\reducedstrut#1\/}%
  \endgroup
} % to highlight the similarities between two code blocks
\newcommand{\codesime}[1]{%
  \begingroup
  \setlength{\fboxsep}{0pt}%
  \colorbox{blue!25}{\reducedstrut#1\/}%
  \endgroup
} % to highlight the similarities between two code blocks
\newcommand{\codeauto}[1]{%
  \begingroup
  \setlength{\fboxsep}{0pt}%
  \colorbox{cyan!30}{\reducedstrut#1\/}%
  \endgroup
} % to highlight automatically-generated terms

\begin{document}

%% Title information
\title[]{Proof Repair Across Type Equivalences}         %% [Short Title] is optional;
                                        %% when present, will be used in
                                        %% header instead of Full Title.
%\titlenote{}             %% \titlenote is optional;
                                        %% can be repeated if necessary;
                                        %% contents suppressed with 'anonymous'
%\subtitle{Subtitle}                     %% \subtitle is optional
%\subtitlenote{with subtitle note}       %% \subtitlenote is optional;
                                        %% can be repeated if necessary;
                                        %% contents suppressed with 'anonymous'


%% Author information
%% Contents and number of authors suppressed with 'anonymous'.
%% Each author should be introduced by \author, followed by
%% \authornote (optional), \orcid (optional), \affiliation, and
%% \email.
%% An author may have multiple affiliations and/or emails; repeat the
%% appropriate command.
%% Many elements are not rendered, but should be provided for metadata
%% extraction tools.

%% Author with single affiliation.
\author{Talia Ringer}

\affiliation{
  \institution{University of Washington}  
  \country{USA}                    %% \country is recommended
}
\email{tringer@cs.washington.edu}          %% \email is recommended

%% Author with two affiliations and emails.
\author{RanDair Porter}

\affiliation{
  \institution{University of Washington}  
  \country{USA}                    %% \country is recommended
}
\email{randair@uw.edu}         %% \email is recommended

\author{Nathaniel Yazdani}

\affiliation{
  \institution{Northeastern University}  
  \country{USA}                    %% \country is recommended
}
\email{yazdani.n@husky.neu.edu}         %% \email is recommended

\author{John Leo}

\affiliation{
  \institution{Halfaya Research}  
  \country{USA}                    %% \country is recommended
}
\email{leo@halfaya.org}         %% \email is recommended

\author{Dan Grossman}

\affiliation{
  \institution{University of Washington}  
  \country{USA}                    %% \country is recommended
}
\email{djg@cs.washington.edu}         %% \email is recommended

%% Abstract
%% Note: \begin{abstract}...\end{abstract} environment must come
%% before \maketitle command
\begin{abstract}
We describe a new approach to automatically repairing broken proofs in response to changes in programs
and specifications in the Coq proof assistant.
Our approach combines a configurable proof term transformation with a proof term to tactic script decompiler.
The proof term transformation implements transport across certain equivalences in a way that is suitable for repair and does not rely on axioms beyond those Coq assumes.

We have implemented this approach in the form of \toolname, a Coq plugin for proof repair.
%The result is a flexible proof repair tool with workflow integration.
We have used \toolname to support a benchmark from a user study,
ease development with dependent types,
port functions and proofs between unary and binary natural numbers,
and support an industrial proof engineer to more easily interoperate between Coq and other verification tools.
\end{abstract}

%% 2012 ACM Computing Classification System (CSS) concepts
%% Generate at 'http://dl.acm.org/ccs/ccs.cfm'.
\begin{CCSXML}
<ccs2012>
<concept>
<concept_id>10011007.10011006.10011008</concept_id>
<concept_desc>Software and its engineering~General programming languages</concept_desc>
<concept_significance>500</concept_significance>
</concept>
<concept>
<concept_id>10003456.10003457.10003521.10003525</concept_id>
<concept_desc>Social and professional topics~History of programming languages</concept_desc>
<concept_significance>300</concept_significance>
</concept>
</ccs2012>
\end{CCSXML}

\ccsdesc[500]{Software and its engineering~General programming languages}
\ccsdesc[300]{Social and professional topics~History of programming languages}
%% End of generated code


%% Keywords
%% comma separated list
\keywords{proof engineering, proof repair, proof reuse, proof evolution}  %% \keywords are mandatory in final camera-ready submission


%% \maketitle
%% Note: \maketitle command must come after title commands, author
%% commands, abstract environment, Computing Classification System
%% environment and commands, and keywords command.
\maketitle

%% Body
\section{Introduction}

Program verification makes it possible to prove programs correct.
Proof engineering has made this a lot easier in the last couple of decades.
Unfortunately, maintaining proofs of programs as those programs change over
time is still a major challenge,
so much that even experts sometimes still just start from scratch.

The problem of refactoring or repairing proofs automatically has been explored in the past,
but mostly in an ad-hoc fashion, and without actually applying discovered patches.
We show a more principled approach to proof refactoring and repair. % functions are not enough!s
This approach combines a program transformation over proof terms (Section~\ref{sec:meat})
with search procedures to instantiate the program transformation to a particular change (Section~\ref{sec:search})
and a decompiler from proof terms back to tactics (Section~\ref{sec:decompiler}).

The key is to view every change from an old specification to new specification as corresponding to one or more equivalences,
possibly with additional information (equivalences between sigma types).
The program transformation then transports the proof term across the equivalence,
producing a term that no longer refers to the old specification.
Finally, the tactic decompiler takes those proof terms back to tactics that a proof engineer
can actually maintain.

We implement this in \toolname, a plugin for Coq 8.8. We use \toolname to do a lot of things:

\begin{enumerate}
  \item things
\end{enumerate}

We also contribute things:

\begin{enumerate}
\item things
\end{enumerate}

\section{Reuse, Refactoring, \& Repair}

Proof refactoring and proof repair are really just proof reuse over time.
That is, proof reuse asks how to repurpose a proof about some specification
to derive a proof about some similar specification.
Proof refactoring and repair ask the same thing, except over time,
between an old specification and a new specification.
There is just one key difference between this and reuse:
in refactoring and repair, after a change,
the old specification no longer exists, so the proof about the new specification
cannot refer to the proof about the old specification.

Consider this example:

\begin{lstlisting}
TODO
\end{lstlisting}
There is an equivalence between the two specifications:

\begin{lstlisting}
TODO
\end{lstlisting}
This means that we can use the old proofs to derive the new proofs:

\begin{lstlisting}
TODO
\end{lstlisting}
We can even avoid doing this by hand using \textit{transport} if our type theory is univalent,
or we can use something like the univalent parametricity framework.
But all of these have one thing in common, which is that the new proof refers to the old specification.
With repair, our old specification no longer exists, so the new proof does not type check.

With our tool we can transform this proof to a new proof that doesn't refer to the old specification at all:

\begin{lstlisting}
TODO
\end{lstlisting}
This works by a configurable program transformation over the proof term (generalizing the work from DEVOID),
followed by a decompilation step from the proof term back to tactics.
The tool is also sometimes able to infer the configuration for the program transformation automatically (also generalizing the work from DEVOID).

From these things, we get a tool that lets us do not just proof reuse, but also proof refactoring and proof repair
with the same approach. We walk through one example of each below.

\subsection{Reuse}

Taking all of the DEVOID stuff further---where they left off, didn't see it as the same problem,
but it is and we'll show you why and how.

\subsection{Refactoring}

Galois code.

\subsection{Repair}

REPLICA benchmark(s).

\section{Problem Definition}
\label{sec:key1}

\toolname is a tool for \textit{proof repair}.
Proof repair is the problem of updating a broken proof in response to a change in a program or specification~\cite{PGL-045, pumpkinpatch}---in the
case of \toolname, a change in a type definition that corresponds to an equivalence (Section~\ref{sec:scope}).

We can view proof repair as a form of 
\textit{proof reuse}~\cite{Ringer2019, felty1994generalization, caplan1995logical, pons2000generalization, johnsen2004theorem}, % TODO consider citation list
or reusing proofs about one specification (say, from another library, or from within the same proof development)
to derive proofs about another specification.
The difference is that in standard proof reuse, both of these specifications continue to exist.
In contrast, proof repair is the process of reusing proofs across \textit{two versions of a single specification},
only one of which---the new version---must continue to exist.
That is, the old version of the specification may be removed after updating proofs to use the new version.

\begin{quote}
\textbf{Insight 1}:
Proof repair is a form of proof reuse---reusing proofs about one specification to derive proofs about another specification---with 
the additional challenge that one of the specifications may cease to exist (Section~\ref{sec:repair}).
The key to supporting proof repair is to build a proof reuse
tool that can handle that additional challenge (Section~\ref{sec:time}).
\end{quote}

\subsection{Scope: Type Equivalences}
\label{sec:scope}

\begin{figure*}
\codeauto{%
\begin{minipage}{0.48\textwidth}
\lstinputlisting[firstline=1, lastline=14]{equivproof.tex}
\end{minipage}
\hfill
\begin{minipage}{0.48\textwidth}
\lstinputlisting[firstline=16, lastline=30]{equivproof.tex}
\end{minipage}}
\vspace{-0.3cm}
\caption{Two functions between \lstinline{Old.list} and \lstinline{New.list} (top) that form an equivalence (bottom).}
\label{fig:equivalence}
\end{figure*}

\toolname supports a particular kind of repair: proof repair in response to certain changes in type definitions.
In particular, these changes in type definitions must correspond to \textit{type equivalences}~\cite{univalent2013homotopy},
or pairs of functions that map between two types and are mutual inverses.
Figure~\ref{fig:equivalence} shows a type equivalence between the two versions of \lstinline{list}
from Section~\ref{sec:overview}, Figure~\ref{fig:listswap} that \toolname discovered and proved automatically.
When such a type equivalence between two types exists, we say those types are \textit{equivalent} (denoted $\simeq$), for example:

\begin{lstlisting}
Old.list $\simeq$ New.list
\end{lstlisting}

\toolname further requires that the equivalences it supports are expressed using something called a \textit{configuration}.
At a high level, a configuration is a deconstructed equivalence.
Its purpose is to help the proof term transformation update proof terms across the equivalence in a way that removes
references to the old type (like \lstinline{Old.list}), all the while abstracting away the details specific to any given change in type
(like permuting arguments to induction principles).

In Section~\ref{sec:configurable}, we will define the notion of a configuration more formally.
We will also show that every equivalence induces a configuration, % TODO one lemma missing from this though
though not necessarily one that is suitable for repair (removing references to the old type).
All of the changes from Table~\ref{fig:changes} can be described by configurations in a way that is suitable for repair.
To give some more intuition for what kinds of changes can be described this way, we elaborate on two such changes below.

\subsubsection{Factoring out Constructors}
\label{sec:ex1}

\begin{figure}
\begin{minipage}{0.48\columnwidth}
\lstinputlisting[firstline=1, lastline=3]{equiv2.tex}
\end{minipage}
\hfill
\begin{minipage}{0.48\columnwidth}
\lstinputlisting[firstline=5, lastline=7]{equiv2.tex}
\end{minipage}
\vspace{-0.3cm}
\caption{The type \lstinline{J} (right) is \lstinline{I} (left) with \lstinline{A} and \lstinline{B} factored out to \lstinline{bool} (Coq standard library).}
\label{fig:equivalence2}
\end{figure}

Consider using \toolname to port functions and proofs across the change from the type \lstinline{I} to the type \lstinline{J} 
in Figure~\ref{fig:equivalence2}.
\lstinline{J} can be viewed as \lstinline{I} with its two constructors \lstinline{A} and \lstinline{B} pulled out to a
new hypothesis of type \lstinline{bool} for a single constructor.

With \toolname, the proof engineer can repair functions and proofs about \lstinline{I} to instead use \lstinline{J},
as long as she first configures \toolname to describe which constructor 
of \lstinline{I} maps to \lstinline{true} and which maps to \lstinline{false}.
This information about constructor mappings induces an equivalence \lstinline{I }$\simeq$\lstinline{ J}
along which \toolname repairs functions and proofs.

The file \href{https://github.com/uwplse/pumpkin-pi/blob/master/plugin/coq/playground/constr_refactor.v}{constr_refactor.v}
shows an example of this, mapping \lstinline{A} to \lstinline{true} and \lstinline{B} to false.
It uses \toolname to automatically repair functions and proofs over \lstinline{I}, like:

\begin{lstlisting}
Theorem demorgan_1 : $\forall$ (i1 i2 : I),(@\vspace{-0.04cm}@)
  neg (and i1 i2) = or (neg i1) (neg i2).(@\vspace{-0.04cm}@)
Proof.(@\vspace{-0.04cm}@)
  intros i1 i2.(@\vspace{-0.04cm}@)
  induction i1; auto.(@\vspace{-0.04cm}@)
Qed.
\end{lstlisting}
to corresponding functions and proofs over \lstinline{J}, like:

\begin{lstlisting}[backgroundcolor=\color{cyan!30}]
Theorem demorgan_1 : $\forall$ (j1 j2 : J),(@\vspace{-0.04cm}@)
  neg (and j1 j2) = or (neg j1) (neg j2).(@\vspace{-0.04cm}@)
Proof.(@\vspace{-0.04cm}@)
  intros j1 j2.(@\vspace{-0.04cm}@)
  induction j1 (@\codediff{as [b]. induction b as [ | ]}@); auto.(@\vspace{-0.04cm}@)
Qed.
\end{lstlisting}
These repaired functions and proofs refer to \lstinline{J} in place of \lstinline{I}.
Otherwise, they behave the same way as the functions and proofs over \lstinline{I} up to the equivalence between
\lstinline{I} and \lstinline{J}---Section~\ref{sec:repair} explains this intuition more formally.

\subsubsection{Adding a Dependent Index}
\label{sec:ex2}

Despite the fact that \toolname requires changes to correspond to type equivalences,
\toolname can in fact handle some changes in which the proof engineer adds or removes information.
This is true when it is possible to express those changes as equivalences between \textit{refinements} ($\Sigma$ types).
It is up to the proof engineer to supply the additional information needed to construct proofs about the refinement
(the corresponding projection of the $\Sigma$ type).

%In general, for any change in type definition, there is always \href{https://github.com/uwplse/pumpkin-pi/blob/master/plugin/coq/playground/trivial.v}{a trivial equivalence} between $\Sigma$ types.
%But this trivial equivalence is not useful for proof engineers, since it requires as much information from the proof engineer as just
%writing the new proof from scratch.


\begin{figure*}
\begin{minipage}{0.40\textwidth}
   \lstinputlisting[firstline=1, lastline=4]{listtovect.tex}
\end{minipage}
\hfill
\begin{minipage}{0.58\textwidth}
   \lstinputlisting[firstline=6, lastline=9]{listtovect.tex}
\end{minipage}
\vspace{-0.3cm}
\caption{A vector (right) is a list (left) indexed by its length.}
\label{fig:listtovect}
\end{figure*}

Practically, the key to handling changes that add or remove information usefully is to separate the new information from redundant old information, 
and include only the new information in the correpsonding projection of the $\Sigma$ type.
Consider, for example, changing a list to a length-indexed vector (Figure~\ref{fig:listtovect}).
An early version of \toolname called \textsc{Devoid}~\cite{Ringer2019} could repair proofs about lists to proofs about \textit{vectors of some length}, since:

\begin{lstlisting}
list T $\simeq$ $\Sigma$(n : nat).vector T n.
\end{lstlisting}
This is enough to automatically repair a lemma about lists:

\begin{lstlisting}
$\forall$ {A B} (l1 : list A) (l2 : list B),(@\vspace{-0.04cm}@)
  zip_with pair l1 l2 = zip l1 l2.
\end{lstlisting}
to a lemma about vectors of some length:

\begin{lstlisting}
$\forall$ {A B} (l1 : (@\codediff{$\Sigma$(n : nat).vector A n}@)) (l2 : (@\codediff{$\Sigma$(n : nat).vector B n}@)),(@\vspace{-0.04cm}@)
  zip_with pair l1 l2 = zip l1 l2.
\end{lstlisting}
recursively updating dependencies \lstinline{zip} and \lstinline{zip_with}.
It is not enough, however, to help the proof engineer get from that to a proof about vectors \textit{of a particular length}:

\begin{lstlisting}
$\forall$ {A B} (@\codediff{n}@) (l1 : (@\codediff{vector A n}@)) (l2 : (@\codediff{vector B n}@)),(@\vspace{-0.04cm}@)
  zip_with pair (@\codediff{n}@) l1 l2 = zip (@\codediff{n}@) l1 l2.
\end{lstlisting}

\textsc{Devoid} leaves this step to the proof engineer.
\toolname, in contrast, can handle this step as well (\href{https://github.com/uwplse/pumpkin-pi/blob/master/plugin/coq/examples/Example.v}{\lstinline{Example.v}}).
The key is to repair functions and proofs across this equivalence:

\begin{lstlisting}
$\Sigma$(l : list T).length l = n $\simeq$ vector T n.
\end{lstlisting}
From the proof engineer's perspective, when the proof engineer changes specifications to refer to \lstinline{vector} instead of \lstinline{list},
to fix her functions and proofs, she must additionally prove invariants about the lengths of her lists.
\toolname makes it easy to separate out that proof obligation, then automates the rest.
Section~\ref{sec:search} shows this and other case studies using \toolname to repair real proofs
informed by the needs of proof engineers.

% TODO show an example when it is not possible to define this

\subsection{Transport with a Twist}
\label{sec:repair}

Proof repair across type equivalences corresponds to a particular kind of proof reuse called \textit{transport},
with the twist that the specification about which we are reusing proofs may cease to exist.
A transport method takes an input term $t$ and produces an output term $t'$ that is \textit{equal up to transport}
along an equivalence $A \simeq B$ (denoted $t \equiv_{A \simeq B} t'$).
Informally, equality up to transport means that if $t$ is a function, then $t'$ behaves the same way modulo the equivalence;
if $t$ is a proof, then $t'$ proves the same theorem the same way modulo the equivalence.
For example, in Section~\ref{sec:overview}, the original append function \lstinline{++} over \lstinline{Old.list}
and the updated append function \lstinline{++} over \lstinline{New.list} that \toolname produces are
equal up to transport along the equivalence from Figure~\ref{fig:equivalence}, since:

\begin{lstlisting}
$\forall$ T (l1 l2 : Old.list T), swap T (l1 ++ l2) = (swap T l1) ++ (swap T l2).
\end{lstlisting}
by induction and rewriting, and similarly in the opposite direction.
The original \lstinline{rev_app_distr} is equal to the transformed proof along the same equivalence,
since it proves the same thing the same way as the transformed proof up to the same equivalence, and up to the changes in \lstinline{++}
and \lstinline{rev}.

The formal details of equality up to transport in a univalent type theory can be found in \citet{univalent2013homotopy}, and an approximation in Coq without univalence can be found in \citet{tabareau2017equivalences}.
Note that for any equivalent \A and \B, there can be many equivalences $A \simeq B$.
Equality up to transport is along a \textit{particular} equivalence, though we erase this in the notation.

Transport methods typically work by explicitly applying the functions that make up the equivalence to convert
inputs and outputs back and forth between equivalent types.
This approach would not work for repair, since it does not make it possible to remove the old specification.
The goal of a proof repair tool like \toolname is to define a transport method that
can remove references to the old specification, %rather than converting back and forth like standard transport methods.
%That way, the proof repair tool can produce proofs that no longer refer in any way to the old specification,
since the old specification may no longer exist.

Section~\ref{sec:overview} showed a simple case of this: \toolname
reused the proof of \lstinline{rev_app_distr} defined over \lstinline{Old.list}
to generate a new proof of \lstinline{rev_app_distr} defined over equivalent \lstinline{New.list}.
Furthermore, it did so in a way that removed all references to \lstinline{Old.list}, both in the proof
and in its dependencies.
That way, after calling \lstinline{Repair}, \lstinline{Old.list} could be removed.

\subsection{\textsc{Carrot}: A Tool for Proof Repair Across Type Equivalences}
\label{sec:time}

\begin{figure*}
\begin{minipage}{0.52\textwidth}
\includegraphics[width=\linewidth]{workflowa.pdf}
\end{minipage}
\hfill
\begin{minipage}{0.45\textwidth}
\includegraphics[width=\linewidth]{workflowb.pdf}
\vspace{0.97cm}
\end{minipage}
\vspace{-0.4cm}
\caption{The two possible workflows for \toolname, using either automatic (left) or manual (right) configuration.}
\label{fig:system}
\end{figure*}

\toolname implements this transport with a twist using a proof term transformation.
The proof term transformation implements transport across equivalences,
but in a way that replaces references to the old specification (in Section~\ref{sec:overview}, the theorem that refers to \lstinline{Old.list})
with references to the new specification (in Section~\ref{sec:overview}, the theorem that refers to \lstinline{New.list}).
This proof term transformation is configurable to a particular equivalence:
it takes a \textit{configuration} (see Section~\ref{sec:configurable}) 
that tells \toolname how to transform certain constructors, eliminators, and equalities that 
correspond to the equivalence.
The configuration can be supplied manually, or discovered automatically by a search procedure.

TODO reash configurations, explain that configurations always induce equivalences, though not always in ways suitable for repair
\toolname further requires that the equivalences it supports are expressed using a particular configuration that we will define in Section~\ref{sec:configurable}. At a high level, this configuration helps the proof term transformation update constructors and eliminators
while maintaining some level of abstraction... etc. make this the focus, call it out as the sort of key trick.

Figure~\ref{fig:system} shows how this comes together when the proof engineer invokes \toolname:

\begin{enumerate}
\item \toolname configures itself, either:
\begin{enumerate}
\item automatically (left), using \textbf{Configure} to discover the configuration, or
\item manually (right), by taking the configuration as an argument.
\end{enumerate}
\item The configured \textbf{Transform} transforms the old proof term into the new proof term.
\item \textbf{Decompile} produces a new proof script from the new proof term.
\end{enumerate}

The example in Section~\ref{sec:overview} uses automatic configuration. When we run the \lstinline{Repair} command,
\textbf{Configure} invokes a search procedure that automatically proves the equivalence in Figure~\ref{fig:equivalence},
then configures \textbf{Transform} using that equivalence.
\textbf{Transform} then ports the proof term that inducted over \lstinline{Old.list}
to induct over \lstinline{New.list}, and finally
\textbf{Decompile} produces the tactic script in Figure~\ref{fig:auto}.

There are currently four search procedures for automatic configuration implemented in \toolname,
all informed by the needs of real proof engineers:

\begin{enumerate}
\item porting between tuples and records,
\item renaming and permuting constructors of inductive types,
\item porting along algebraic ornaments to types at \textit{some} index (from \textsc{Devoid}), and
\item unpacking types at \textit{some} index to a \textit{particular} index.
\end{enumerate}
All four search procedures generate equivalence proofs as in Figure~\ref{fig:equivalence} automatically (\href{https://github.com/uwplse/pumpkin-pi/blob/master/plugin/src/automation/search/search.ml}{search.ml} and \href{https://github.com/uwplse/pumpkin-pi/blob/master/plugin/src/automation/search/equivalence.ml}{equivalence.ml}),
then configure (\href{https://github.com/uwplse/pumpkin-pi/blob/master/plugin/src/automation/lift/liftconfig.ml}{liftconfig.ml}) the transformation to those equivalences.
Manual configuration makes it possible
for the proof engineer to directly configure the transformation to a particular equivalence
when a search procedure for that equivalence is not yet implemented.
Section~\ref{sec:search} shows examples of both workflows applied to real proof reuse and repair scenarios.





\section{A Configurable Proof Term Transformation}
\label{sec:key2}

At the heart of \toolname is a configurable proof term transformation for transporting
proofs across equivalences. This is based on the proof term transformation from 
\textsc{Devoid}~\cite{Ringer2019}, which solved this problem for particular class of equivalences.
The goal of \toolname is to implement something like \textsc{Devoid}, but over
a much broader set of changes, and with better workflow integration.
We were able to generalize the \textsc{Devoid} algorithm to do this.

\begin{quote}
\textbf{Insight 2}:
The proof term transformation from the \textsc{Devoid}~\cite{Ringer2019} proof reuse tool can be generalized (Section~\ref{sec:generic})
to build such a proof repair tool (Section~\ref{sec:implementation}), and the result can handle 
many different kinds of changes (Section~\ref{sec:configurable}).
\end{quote}

\begin{figure}
\small
\begin{grammar}
<i> $\in \mathbbm{N}$, <v> $\in$ Vars, <s> $\in$ \{ Prop, Set, Type<i> \}

<t> ::= <v> | <s> | $\Pi$ (<v> : <t>) . <t> | $\lambda$ (<v> : <t>) . <t> | <t> <t> | \\
Ind (<v> : <t>)\{<t>,\ldots,<t>\} | Constr (<i>, <t>) | Elim(<t>, <t>)\{<t>,\ldots,<t>\}
\end{grammar}
\caption{Syntax for CIC$_\omega$.} % TODO cite existing work, both DEVOID and the place it is from
\label{fig:syntax}
\end{figure}

\paragraph{Conventions}
All terms that we introduce in this section are in CIC$_{\omega}$ with primitive eliminators,
the syntax for which is in Figure~\ref{fig:syntax}.
The typing rules are standard.
Throughout, we use $\vec{i}$ and $\{t_1, \ldots, t_n\}$ to denote lists of terms.

\subsection{The Configuration}
\label{sec:configurable}

This configuration is the key to building a proof term transformation that can support many different classes of changes.
Before introducing the proof term transformation, we will describe the configuration, which in effect specifies the behavior
of the transformation.

The configuration instantiates the proof term transformation to two equivialent types \A and \B, so that the proof term transformation
can transform terms defined over \A to terms defined over \B instead.
%The behavior of the proof term transformation at a particular equivalence hinges on correct configuration.
At a high level, the configuration helps the transformation achieve two goals: % TODO is ``ensures'' too strong if we don't prove it?

\begin{enumerate}
\item preserve the equivalence between \A and \B, and
\item produce well-typed terms.
\end{enumerate}
This configuration is a pair of pairs:

\begin{lstlisting}
((DepConstr, DepElim), (Eta, Iota))
\end{lstlisting}
each of which corresponds to one of the two goals, namely:

\begin{enumerate}
\item \lstinline{DepConstr} and \lstinline{DepElim} define how to transform constructors and eliminators, thereby preserving the equivalence (Section~\ref{sec:equivalence}), and 
\item \lstinline{Eta} and \lstinline{Iota} define how to transform $\eta$-expansion and $\iota$-reduction of these constructors and eliminators, thereby producing well-typed terms (Section~\ref{sec:equality}).
\end{enumerate}
Each of these is defined in terms of other terms in CIC$_{\omega}$ for any given equivalence.

\textbf{Configure} passes this configuration to \textbf{Transform}.
The four parts of this configuration must be in relation to one another in a certain way in order for the proof
term transformation to work correctly (Section~\ref{sec:art}).

\subsubsection{Equivalence}
\label{sec:equivalence}

The two configuration parts responsible for ensuring that the program transformation preserves equivalence
are \lstinline{DepConstr} (\textit{dependent constructors}) and \lstinline{DepElim} (\textit{dependent eliminators}).
These describe how to construct and eliminate \A and \B, wrapping the two types with a common inductive structure.
There must be the same number of dependent constructors and inductive hypotheses in dependent eliminators for both \A and \B,
even if \A and \B are inductive types with different numbers of constructors.
The idea is to port functions and proofs over \A to functions and proofs over \B by viewing \B as if it is an \A.
This way, the rest of the transformation can replace constructions of \A with constructions of \B and
inductive proofs about \A with inductive proofs about \B, but otherwise just recursively lift
the other subterms without changing the order or number of arguments.

\begin{figure}
\begin{minipage}{0.48\textwidth}
\begin{lstlisting}
DepConstr(0, list T) : list T :=
  Constr((@\codediff{0}@), list T).
DepConstr(1, list T) t l : list T :=
  Constr ((@\codediff{1}@), list T) t l.

DepElim(l, P) { p$_{\mathtt{nil}}$, p$_{\mathtt{cons}}$ } : P l :=
  Elim(l, P) { (@\codediff{p$_{\mathtt{nil}}$}@), (@\codediff{p$_{\mathtt{cons}}$}@) }.
\end{lstlisting}
\end{minipage}
\hfill
\begin{minipage}{0.48\textwidth}
\begin{lstlisting}
DepConstr(0, list T) : list T :=
  Constr((@\codediff{1}@), list T).
DepConstr(1, list T) t l : list T :=
  Constr((@\codediff{0}@), list T) t l.

DepElim(l, P) { p$_{\mathtt{nil}}$, p$_{\mathtt{cons}}$ } : P l :=
  Elim(l, P) { (@\codediff{p$_{\mathtt{cons}}$}@), (@\codediff{p$_{\mathtt{nil}}$}@) }.
\end{lstlisting}
\end{minipage}
\caption{The dependent constructors and eliminators for old (left) and new (right) \lstinline{list}.}
\label{fig:listconfig}
\end{figure}

For the \lstinline{list} change from Figure~\ref{fig:listswap},
the configuration that \toolname discovers uses the the dependent constructors
and eliminators in Figure~\ref{fig:listconfig}. The dependent constructors for \lstinline{Old.list}
are just the normal constructors \lstinline{nil} and \lstinline{cons} with the order unchanged,
while the dependent constructors for \lstinline{New.list} swap \lstinline{nil} and \lstinline{cons}
back to the original order.
Similarly, the dependent eliminator for \lstinline{Old.list} is just the normal eliminator for \lstinline{Old.list},
while the dependent eliminator for \lstinline{New.list} swaps the \lstinline{nil} and \lstinline{cons} cases.

When our types are not both inductive types, the constructors and eliminators can be \textit{dependent}.
One example of this arises from integrating the change from lists and $\Sigma$\lstinline{(n : nat).vector T n} that
\textsc{Devoid} supports into the \toolname framework.
We can configure the \toolname transformation to perform this same repair
by configuring the dependent constructors to \textit{pack} the index into an existential, for example:

\begin{lstlisting}
DepConstr(0, $\Sigma$(n : nat).vector T n) : $\Sigma$(n : nat).vector T n :=
  $\exists$ (Constr(0, nat)) (Constr(0, vector T)).
\end{lstlisting}
and by configuring the eliminator to eliminate over the projections:

\begin{lstlisting}
DepElim(s, P) { f$_0$ f$_1$ } : P ($\exists$ ($\pi_l$ s) ($\pi_r$ s)) :=
  Elim($\pi_r$ s, $\lambda$ (n : nat) (v : vector T n) . P ($\exists$ n v)) {
    f$_0$
    ($\lambda$ (t : T) (n : nat) (v : vector T n) . f$_1$ t ($\exists$ n v))
  }. 
\end{lstlisting}

In both of these examples, the only interesting work moves into the configuration:
the configuration for the swap example takes care of swapping constructors and cases,
and the configuration for the \textsc{Devoid} example implements the constructor and eliminator rules from the \textsc{Devoid} transformation.
That way, that the rest of the \toolname transformation does not need to add, drop, or reorder arguments at any point.
In essence, all of the difficult work moves into the configuration, but once it is done, it is done.
Furthermore, for both of the examples above, the \textsc{Configure} component of \toolname is able to discover \lstinline{DepConstr}
and \lstinline{DepElim} from just the types \A and \B, taking care of even the difficult work.

\subsubsection{Equality}
\label{sec:equality}

The other configuration parts \lstinline{Eta} and \lstinline{Iota} deal with transporting equalities.
A naive proof term transformation in a non-univalent language, as noted in \citet{tabareau2019marriage},
may fail to generate well-typed terms if it does not consider the problem of transporting equalities.
Otherwise, if the transformation transforms a term \lstinline{t : T} to some \lstinline{t' : T'}, it does not necessarily
hold that it transforms \lstinline{T} to \lstinline{T'}.

The two rules that correspond to this in the program transformation are \lstinline{Eta} and \lstinline{Iota},
which use configuration parts with the same name.
These together describe identity and equality as they relate to \lstinline{DepConstr} and \lstinline{DepElim}.
More formally, they define $\eta$-expansion and $\iota$-reduction over \lstinline{DepConstr} and \lstinline{DepElim},
which may be propositional rather than definitional, and so must be represented explicitly in the transformation.

\lstinline{Eta} describes how to $\eta$-expand the body of the identity function in a way that preserves equalities
coherently with the definitions of \lstinline{DepConstr} and \lstinline{DepElim}.
Note that this refers to $\eta$ over a given type, like $\eta$-expansion for $\Sigma$ types, and not to $\eta$-expansion of functions.
For example, for the change from list to \lstinline{$\Sigma$(n : nat).vector T n}, we have:

\begin{lstlisting}
Eta ($\Sigma$(n : nat).vector T n) :=
  $\lambda$ (s : $\Sigma$(n : nat).vector T n).$\exists$ (\$pi_l$ s) (\$pi_r$ s).
\end{lstlisting}
which corresponds to the strategic packing in the \textsc{Devoid} implementation to deal with
non-primitive projections not handled by the transformation. Thanks to this, we can forego the assumption from the \textsc{Devoid} transformation
that our language has primitive projections (definitional $\eta$ for $\Sigma$ types).

\begin{figure}
\begin{minipage}{0.48\textwidth}
   \lstinputlisting[firstline=1, lastline=8]{nattobin.tex}
\end{minipage}
\hfill
\begin{minipage}{0.48\textwidth}
   \lstinputlisting[firstline=10, lastline=17]{nattobin.tex}
\end{minipage}
\caption{Unary (left) and binary (right) natural numbers.}
\label{fig:nattobin}
\end{figure}

Each \lstinline{Iota}---one per constructor---describes and proves the $\iota$-reduction behavior
of \lstinline{DepElim} on the corresponding case. For example, with unary natural numbers defined in the standard way,
and using the standard eliminator over the natural numbers, the $\iota$ rules are definitional, since:

\begin{lstlisting}
nat.refold_elim_S:
  $\forall$ P p$_\texttt{0}$ p$_\texttt{S}$ n,
    DepElim((@\codediff{DepConstr(1, nat) n}@), P) { p$_\texttt{0}$ p$_\texttt{S}$ } = (@\codediff{p$_\texttt{S}$}@) n (DepElim(n, P) { p$_\texttt{0}$ p$_\texttt{S}$ }).
\end{lstlisting}
goes through by reflexivity.
However, this is no longer true for binary numbers \lstinline{N} as defined in the Coq standard library (Figure~\ref{fig:nattobin}).
So while we can in fact define \lstinline{DepConstr} and \lstinline{DepElim} to induce an equivalence
between them (see Section~\ref{sec:bin}), we run into trouble reasoning about applications of \lstinline{DepElim},
since the corresponding $\iota$ rule:

\begin{lstlisting}
N.refold_elim_S:
  $\forall$ P p$_\texttt{0}$ p$_\texttt{S}$ n,
    DepElim((@\codediff{DepConstr(1, N) n}@), P) { p$_\texttt{0}$ p$_\texttt{S}$ } = (@\codediff{p$_\texttt{S}$}@) n (DepElim(n, P) { p$_\texttt{0}$ p$_\texttt{S}$ }).
\end{lstlisting}
no longer holds by reflexivity.

The result of this is that proofs about \lstinline{nat} that hold by reflexivity
do not necessarily hold by reflexivity over \lstinline{N}. For example, in Coq,
while \lstinline{S (n + m) = S n + m} holds by reflexivity over \lstinline{nat},
when we define \lstinline{+} with our new dependent eliminator over \lstinline{N},
this no longer holds by reflexivity.
To transform proofs about \lstinline{nat} to proofs about \lstinline{N}, we must transform \textit{definitional} $\iota$-reduction over \lstinline{nat}
to explicit \textit{propositional} $\iota$-reduction over \lstinline{N}.
For our choice of equivalence in Section~\ref{sec:bin} between \lstinline{nat} and \lstinline{N}, the \lstinline{Iota} rules are trivial
for the base case and are exactly the proofs of theorems above for the succesor case.

Taken together over both \A and \B, \lstinline{Iota} describes how the inductive structures of \A and \B differ from one another.
The structures of \lstinline{DepElim} over \A and \B are always the same, so if \A and \B have the same 
inductive structure (if they are \textit{ornaments}~\cite{mcbride}),
then if $\iota$-reduction is definitional over \lstinline{DepElim} on \A, it will also be definitional on \B.
Otherwise, if \A and \B have different inductive structures, as with \lstinline{nat} and \lstinline{N},
then definitional $\iota$ over one would lift to propositional $\iota$ over the other.
The \lstinline{Iota} rule of the transformation encodes this fact.

For the case of \lstinline{nat} and \lstinline{N},
the need for explicit $\iota$ was noted as far back as \citet{magaud2000changing}.
What \textsc{Iota} does in the proof term transformation is encode this more generally for any change in inductive type.

\subsubsection{Equivalence \& Equality}
\label{sec:art}

Of course, both when designing a search procedure for an automatic configuration and when
configuring \toolname manually, choosing correct and useful configuration is important,
and it is not always straightforward. This section specifies what it means for these
to be correct and gives some intuition to why.
Section~\ref{sec:search} shows some useful example configurations.

The configuration ((\lstinline{DepConstr}, \lstinline{DepElim}), (\lstinline{Eta}, \lstinline{Iota})) instantiates
the proof term transformation to a particular equivalence between \A and \B.
Choosing an equivalence is a bit of an art:
there can be infinitely many equivalences that correpond to a 
given change in specification, only some of which are useful.
Beyond that, even once we have chosen an equivalence, we could define many possible configurations that correspond
to the equivalence, some of which will produce functions and proofs that are more useful or efficient than others.

Thankfully, once the art is done, we at least understand what it means for it to be \textit{correct art}.
The correctness criteria for the configuration relate \lstinline{DepConstr}, \lstinline{DepElim}, \lstinline{Eta}, and \lstinline{Iota}
in a way that preserves equivalence (Section~\ref{sec:equivalence}) coherently with equality (Section~\ref{sec:equality}).

To preserve equivalence, we need that \lstinline{DepElim} and \lstinline{DepConstr} together induce an equivalence between \A and \B,
formed by one function that eliminates \A and constructs \B, and another function that eliminates \B and constructs \A:

\begin{lstlisting}
f : A -> B := DepElim(a, $\lambda$(a : A).B){ $\lambda$ ... DepConstr(0, B) ..., ... }
g : B -> A := DepElim(b, $\lambda$(b : B).A){ $\lambda$ ... DepConstr(0, A) ..., ... }
\end{lstlisting}
Each search procedure that \textbf{Configure} implements discovers these functions and produces a proof in Coq that these functions form an equivalence.
In addition, we need that:

\begin{enumerate} % TODO make more formal, and check if we need to say it induces particular equivalence
\item $\forall$ \lstinline{j}, \lstinline{DepConstr(j, A)} is equal to \lstinline{DepConstr(j, B)} up to transport along that equivalence, and
\item $\forall$ \lstinline{(a : A) (b : B) (P : A -> Type) (Q : B -> Type)}, if \lstinline{a} is equal to \lstinline{b} and \lstinline{P} is equal to \lstinline{Q} up to transport along that equivalence,
then \lstinline{DepElim(a, P)} is equal to \lstinline{DepElim(b, Q)} up to transport along that equivalence.
\end{enumerate}
The intuition for this is based on insights from \textsc{Devoid},
and is proven on an example from \textsc{Devoid} in the univalent parametricity framework.\footnote{\url{https://github.com/CoqHott/univalent_parametricity/commit/7dc14e69942e6b3302fadaf5356f9a7e724b0f3c}}
Essentially, that these are equal up to transport along the equivalence means that replacing dependent constructors (respectively eliminators) of \A
with dependent constructions (respectively eliminators) of \B will preserve equality up to transport for those particular subterms.
Furthermore, since CIC$_{\omega}$ is a constructive logic, the \textit{only} way to construct an \A (respectively \B) is to use its constructors,
and the \textit{only} way to match over an \A (respectively \B) is to apply its eliminator.
Finally, since these form an equivalence, all ways of constructing or eliminating \A and \B are covered by these dependent constructors and eliminators.
So, as long as we are able to identify and expand all implicit applications of \lstinline{DepConstr} and \lstinline{DepElim},
\textsc{Dep-Constr} and \textsc{Dep-Elim} preserve correctness of the transformation and cover all constructions and eliminations of \A and \B.

To ensure coherence with equality, we need \lstinline{Eta} and \lstinline{Iota} to correctly prove the $\eta$ and $\iota$ rules
over \lstinline{DepConstr} and \lstinline{DepElim}.
For \lstinline{Eta}, we need it to have the same definitional behavior as the
eliminator, in other words:

\begin{lstlisting}
DepElim(a, P) { f$_0$, ..., f$_n$ } : P (Eta(A) a)
\end{lstlisting}
and similarly for \B. % TODO do we need anything for DepConstr?

Each \lstinline{Iota} needs to prove and rewrite along the simplification or refolding behavior that corresponds to a case of the dependent eliminator, in other words: % TODO do we need eta here?

\begin{lstlisting}
Iota(A) :
  $\forall$ P $\vec{f}$ $\vec{x}$ (Q : P (DepConstr(j, A) $\vec{x}$) $\rightarrow$ Type),
    Q (DepElim((@\codediff{DepConstr(j, A) $\vec{x}$}@), P) $\vec{f}$ $\rightarrow$ 
    Q ((@\codediff{$\vec{f}$[j]}@) ... (DepElim(IH$_0$, P) $\vec{f}$) ... (DepElim(IH$_n$, P) $\vec{f}$) ...)
\end{lstlisting}
where each \lstinline{IH}$_i$ is each recursive occurrence of \A in the eliminator case,
and similarly for \B.
Together, these induce proofs of \lstinline{section} and \lstinline{retraction} (used in conjunction with
induction and rewriting).
The intuition here is that it should be enough to preserve the reduction behavior
of the eliminators and constructors, since again those are the only ways we can construct or eliminate our types.

It can be difficult to prove the correctness criteria for the configuration---proving equality of the dependent eliminators
up to transport, for example, requires either a special framework~\cite{tabareau2017equivalences}
or a univalent type theory~\cite{univalent2013homotopy}.
Thankfully, the user does not need to prove the correctness criteria for a configuration in order to use \toolname.
Rather, the correctness criteria simply need to hold in order for the proof term transformation to work correctly.

\subsection{The Proof Term Transformation}
\label{sec:generic}

\begin{figure}
\begin{mathpar}
\mprset{flushleft}
\small
\hfill\fbox{$\Gamma$ $\vdash$ $t$ $\Uparrow$ $t'$}\\

\inferrule[Dep-Elim]
  { \Gamma \vdash a \Uparrow b \\ \Gamma \vdash p_{a} \Uparrow p_b \\ \Gamma \vdash \vec{f_{a}}\phantom{l} \Uparrow \vec{f_{b}} }
  { \Gamma \vdash \mathrm{DepElim}(a,\ p_{a}) \vec{f_{a}} \Uparrow \mathrm{DepElim}(b,\ p_b) \vec{f_{b}} }

\inferrule[Dep-Constr]
{ \Gamma \vdash \vec{t}_{a} \Uparrow \vec{t}_{b} } %\\ TODO must we explicitly lift A to B if we want to handle parameters/indices?
{ \Gamma \vdash \mathrm{DepConstr}(j,\ A)\ \vec{t}_{a} \Uparrow \mathrm{DepConstr}(j,\ B)\ \vec{t}_{b}  }

\inferrule[Eta]
  { \\ }
  { \Gamma \vdash \mathrm{Eta}(A) \Uparrow \mathrm{Eta}(B) }

\inferrule[Iota]
  { \Gamma \vdash c_A \Uparrow c_B \\ \Gamma \vdash q_A \Uparrow q_B \\ \Gamma \vdash e_A \Uparrow e_B }
  { \Gamma \vdash \mathrm{Iota}(j,\ A,\ q_A,\ t_A) \Uparrow \mathrm{Iota}(j,\ B,\ q_B,\ t_B) }

\inferrule[Equivalence]
  { \\ }
  { \Gamma \vdash A\ \Uparrow B }

\inferrule[Constr]
{ \Gamma \vdash T \Uparrow T' \\ \Gamma \vdash \vec{t} \Uparrow \vec{t'} }
{ \Gamma \vdash \mathrm{Constr}(j,\ T)\ \vec{t} \Uparrow \mathrm{Constr}(j,\ T')\ \vec{t'} }

\inferrule[Ind]
  { \Gamma \vdash T \Uparrow T' \\ \Gamma \vdash \vec{C} \Uparrow \vec{C'}  }
  { \Gamma \vdash \mathrm{Ind} (\mathit{Ty} : T) \vec{C} \Uparrow \mathrm{Ind} (\mathit{Ty} : T') \vec{C'} }

\inferrule[Elim] % TODO wait why do we have c here when it clearly refers to the term we eliminate over? um
  { \Gamma \vdash c \Uparrow c' \\ \Gamma \vdash Q \Uparrow Q' \\ \Gamma \vdash \vec{f} \Uparrow \vec{f'}}
  { \Gamma \vdash \mathrm{Elim}(c, Q) \vec{f} \Uparrow \mathrm{Elim}(c', Q') \vec{f'}  }

%% Application
\inferrule[App]
 { \Gamma \vdash f \Uparrow f' \\ \Gamma \vdash t \Uparrow t'}
 { \Gamma \vdash f t \Uparrow f' t' }

% Lamda
\inferrule[Lam]
  { \Gamma \vdash T \Uparrow T' \\ \Gamma,\ t : T \vdash b \Uparrow b' }
  {\Gamma \vdash \lambda (t : T).b \Uparrow \lambda (t : T').b'}

% Product
\inferrule[Prod]
  { \Gamma \vdash T \Uparrow T' \\ \Gamma,\ t : T \vdash b \Uparrow b' }
  {\Gamma \vdash \Pi (t : T).b \Uparrow \Pi (t : T').b'}
\end{mathpar}
\caption{Proof term transformation.}
\label{fig:final}
\end{figure}

Figure~\ref{fig:final} shows the proof term transformation that forms the core of \toolname.
Like the transformation from \textsc{Devoid}, this transformation is parameterized over
two equivalent types \A and \B (\textsc{Equivalence}) and assumes fully expanded terms.
In addition, it is parameterized over the configuration (\lstinline{DepConstr}, \lstinline{DepElim}, \lstinline{Eta}, and \lstinline{Iota}),
which appear in the proof term transformation as metavariables.

The goal of the proof term transformation is to preserve that equivalence in some way, while no longer referring to the old specification.
That is, for equivalent types \A and \B, the transformation takes as input functions and proofs
that refer to \A and returns functions and proofs that refer to \B.
Furthermore, the transformed functions behave the same way as the input functions,
and the transformed proofs talk about the same things as the input proofs.

\begin{figure}
\begin{minipage}{0.48\textwidth}
\begin{lstlisting}
$\lambda$ (T : Type) (l m : list T) .
 Elim (l, $\lambda$(l: list T).list T $\rightarrow$ list T))
 {
   (@\codediff{($\lambda$ m . m)}@),
   ($\lambda$ t _ IHl m.
      Constr((@\codediff{1}@), list T) t (IHl m))
 } m.
\end{lstlisting}
\end{minipage}
\hfill
\begin{minipage}{0.48\textwidth}
\begin{lstlisting}
$\lambda$ (T : Type) (l m : list T) .
 Elim (l, $\lambda$(l: list T).list T $\rightarrow$ list T))
 {
   ($\lambda$ t _ IHl m.
      Constr((@\codediff{0}@), list T) t (IHl m)),
   (@\codediff{($\lambda$ m . m)}@)
 } m.
\end{lstlisting}
\end{minipage}
\caption{The list append function before (left) and after (right) repair.}
\label{fig:appswap1}
\end{figure}

\begin{figure}
\begin{minipage}{0.48\textwidth}
\begin{lstlisting}
$\lambda$ (T : Type) (l m : list T) .
 Elim
   (l, $\lambda$(l: list T).list T $\rightarrow$ list T))
 {
   ($\lambda$ m . m)
   ($\lambda$ t _ IHl m.
      Constr(1, list T) t (IHl m))
 } m.
\end{lstlisting}
\end{minipage}
\hfill
\begin{minipage}{0.48\textwidth}
\begin{lstlisting}
$\lambda$ (T : Type) (l m : list T) .
 (@\codediff{DepElim}@)
   (l, $\lambda$(l: list T).list T $\rightarrow$ list T))
 {
   ($\lambda$ m . m)
   ($\lambda$ t _ IHl m.
      (@\codediff{DepConstr}@)(1, list T) t (IHl m))
 } m.
\end{lstlisting}
\end{minipage}
\caption{The original list append function (left) and the same function rewritten to use \lstinline{DepConstr} (right).}
\label{fig:appswap2}
\end{figure}

To update the append function \lstinline{++} from the left of Figure~\ref{fig:appswap1}, \toolname
identifies implicit applications of \lstinline{DepConstr} and \lstinline{DepElim} and expands them (Figure~\ref{fig:appswap2}).
The transformation then just recursively substitutes in the updated \lstinline{list} type
for the original \lstinline{list} type, which moves \lstinline{DepConstr} and \lstinline{DepElim}
to construct and eliminate over the updated type.
Finally, this reduces to the term on the right of Figure~\ref{fig:appswap1}.
Transforming proofs works the same way.

Both versions of append behave identically up to the equivalence from Figure~\ref{fig:equivalence}, since:

\begin{lstlisting}
$\forall$ T (l1 l2 : Old.list T), f T (l1 ++ l2) = (f T l1) ++ (f T l2).
\end{lstlisting}
by induction and rewriting, and similarly in the opposite direction.
Our transformed proof \lstinline{rev_app_distr} then proves the same thing the same way
as the original proof up to the same equivalence---and up to the corresponding changes in \lstinline{++}
and \lstinline{rev}.

More formally, the output of the proof term transformation ought to be equal to the input of the program transformation
\textit{up to transport} along the equivalence between \A and \B that the configuration instantiates it to.
This is the same as the correctness criterion for the program transformation from \textsc{Devoid} that this is based on,
with the transformation generalized to handle other equivalences beyond the class that \textsc{Devoid} supports.
The key steps in this transformation that make this possible are porting functions and proofs along the configuration corresponding
to a particular equivalence ((\textsc{Dep-Constr}, \textsc{Dep-Elim}), (\textsc{Eta}, and \textsc{Iota})).
From there, rest of the transformation is straightforward.

Note that this correctness criterion is metatheoretical:
stating and proving that two terms are in equal up to transport is in general not possible in a language like Coq
without additional axioms, though it is in some cases realizable with an external tool~\cite{tabareau2017equivalences}.
\toolname does not yet generate these proofs as we were focused on building a usable tool with axiomatic freedom and few dependencies.
Coq ensures that all terms that plugins produce are well-typed; for now, as with \textsc{Devoid}, the proof engineer must vet the transformed
specifications herself.

\subsection{The Tool}
\label{sec:implementation}

The configurable proof term transformation helped us build a flexible proof repair and reuse tool.
However, it alone was not enough to build a tool that reaches real users.
This section describes a sample of the implementation challenges that we encountered and how we solved them.
Section~\ref{sec:discussion} elaborates on the remaining challenges and our plans to address them in the future.

\paragraph{From CIC$_{\omega}$ to Coq}

Like \textsc{Devoid}, we also had to handle language differences to scale from CIC$_{\omega}$ to Coq.
We use the same \lstinline{Preprocess} command that \textsc{Devoid} uses to turn pattern matching and fixpoints into applications of eliminators.
We generalize the work from \textsc{Devoid} to handle Coq's non-primitive projections into our \lstinline{Eta} rule.
We move the work of refolding constants into the configuration for each search procedure, so that it does not
cloud the proof term transformation itself.

\paragraph{Matching Against Preconditions}

It is easy to \textit{describe} the proof term transformation, but it is much more difficult to implement it.
This is because the proof term transformation only describes what transformation rules are applicable when,
but it does not describe how to actually check that the precondition holds.
In many cases, this check is not purely syntactic---we really want to know if a term \textit{unifies}
with an application of \lstinline{DepConstr}, for example, not whether it applies the term exactly.
This is especially pronounced with \lstinline{Eta} and \lstinline{Iota},
which typically show up contracted in real code.
This problem is exactly why \citet{tabareau2019marriage} speculated that converting definitional to propositional equalities
like we do with \lstinline{Iota} may, in general, be intractable.

In practice, we find that unification is often not enough to identify an implicit application of one of the configuration terms.
Each of our search procedures for automatic configuration in turn implements special rewrite rules that tell \toolname
how to identify and expand these implicit applications before applying the transformation.
There is not yet a way for proof engineers themselves to supply these custom rewrite rules,
so sometimes in order to use \toolname with manual configuration, proof engineers must manually expand
input terms to explicitly apply parts of the configuration like \lstinline{Iota}.
This can be challenging, so we plan to give proof engineers the opportunity to write
these custom rewrite rules themselves in the future.

\paragraph{Termination \& Intent}

Another challenge with implementing the proof term transformation is deciding whether to run a rule that matches at all.
That is, when the correctness criteria for a configuration hold and a subterm matches a rule, this suggests that \toolname \textit{can}
run the transformation rule, but it does not necessarily mean that it \textit{should}.
In some cases, repeatedly running a matching transformation rule would result in nontermination.
For example, if our type \B is a refinement of our type \A, then we can always run \textsc{Equivalence}
over and over again, forever.
\textsc{Devoid} ruled out this case by simply prohibiting the case where \B refers to \A, but we found it sometimes
useful in practice to support that case.
To support this, we include some simple termination checks in our code.

More generally, even when termination is guaranteed, whether to run a matching transformation rule
depends on the intent of the user.
For example, our industrial proof engineer sometimes wished to port only some occurrences of \A,
especially when \A was a tuple like \lstinline{(nat * bool)} that could feasibly appear elsewhere in the term
but have a different meaning.
We helped the proof engineer do this by interacting with \toolname using a particular workflow.
We plan to support this automatically using type-directed search in the future.

\paragraph{Reaching Real Users}
Many of our design decisions in implementing \toolname were informed by our partnership with
an industrial user.
For example, we found that our industrial user rarely had the patience to wait more than ten seconds
for \toolname to port a function or proof.
In response, we implemented very aggressive caching (with an option to disable the cache), even caching intermediate subterms that
we encounter in the course of running our program transformation.
We also added the option to set certain terms or even entire modules as opaque to \toolname, to prevent
unecessary $\delta$-reduction of constants.
We included many other optimizations, including lazy $\eta$-expansion of terms like \textsc{Devoid}.

User experiences also informed features that we exposed to users.
For example, all of our search procedures for configurations generate proofs that the discovered
equivalence actually is an equivalence, using functionality that expands on the same functionality from \textsc{Devoid}.
We also implemented special search procedures to generate custom eliminators to make it easier to reason about
types that we found common for certain search procedures.
For example, \toolname generates a custom eliminator automatically to reason about \{\lstinline{l : list T & length l = n}\}
and other similar types by breaking it into parts and reasoning seperately about the two projections.
These features along with our tactic decompiler helped with integration into proof engineering workflows.

%First we need that \lstinline{DepElim} over $A$ into \lstinline{DepConstr} over $B$ and \lstinline{DepElim} over $B$ into
%\lstinline{DepConstr} over $A$ form an equivalence between $A$ and $B$. When that's true, I think it should hold that \lstinline{DepElim} over $A$
%and \lstinline{DepElim} over $B$ are in univalent relation with one another. If not, then that's an extra condition.
%Finally, we need the transformation to preserve definitional equalities. Not sure about the general case, but for vectors and lists,
%we need:

%\begin{lstlisting}
%  $\forall$ A l (f : $\forall$ (l : sigT (Vector.t A)), l = l),
%    vect_dep_elim A (fun l => l = l) (f nil) (fun t s _ => f (cons t s)) l = f (id_eta l).
%\end{lstlisting}
%and:

%\begin{lstlisting}
%Definition elim_id (A : Type) (s : {H : nat & t A H}) :=
%  vect_dep_elim
%    A
%    (fun _ => {H : nat & t A H})
%    nil
%    (fun (h : A) _ IH =>
%      cons h IH)
%    s.

% $\forall$ A h s,
%    exists (H : cons h (elim_id A s) = elim_id A (cons h s)),
%      H = eq_refl.
%\end{lstlisting}
%More generally, for each constructor index $j$, define:

%\begin{lstlisting}
%  eqc (j, B) (f : $\forall$ b : B, b = b) :=
%    fun ... (* TODO get the hypos from the type of the eliminator *) =>
%      f (DepConstr (j, B)) (* TODO args *)%%

  %elim_id := (* TODO *)
%\end{lstlisting}
%Then we need:

%\begin{enumerate}
%\item $\forall b f, \mathrm{DepElim}(b,\ p_{b}) \{\mathrm{eqc} (1, B) f, \ldots, \mathrm{eqc} (n, B) f\} = f (\mathrm{Eta}(A) a) $
%\item Something relating the constructors and \lstinline{elim_id} to reflexivity
%\end{enumerate}
%and similarly for $A$.

%Really the point of these conditions is that from them, with some restrictions on input terms, we can get
%that lifting terms gives us the same type that we'd get from lifting the type. But there are still
%some restrictions (see the few that fail).

%It's probably not always possible to define these three things for every equivalence.
%Could generalize by rewriting. But this lets us avoid the rewriting problem from Nicolas' paper.

% TODO how does this get us something like primitive projections? Just makes Eta definitionally equal to regular Id?

% TODO so we can probably just frame search in terms of DepConstr and DepElim and then generate proofs about this on an ad-hoc basis
% and get away with not including the specific details of our instantiations. We can give examples instead, give intuition, and say we generate
% the proofs in Coq

%For the second one we need not just an eliminator rule but also an identity rule.
%DEVOID assumed primitive projections which let them get away without thinking of this,
%but then had this weirdly ad-hoc ``repacking'' thing in their implementation.
%It turns out this is just a more general identity rule, which basically says what
%the identity function should lift to so that the transformation preserves definitional equalities.
%Actually deciding when to run this rule is one of the biggest challenges in practice,
%so we'll talk about that more in the implementation section.


\section{Decompiling Proof Terms to Tactics}
\label{sec:decompiler}

\textbf{Transform} produces a proof term,
while the proof engineer typically writes and maintains proof scripts made up of tactics.
We improve user experience thanks the realization that, since Coq's proof term language Gallina is very structured,
we can decompile these Gallina terms to Ltac proof scripts for the proof engineer to maintain.

\begin{quote}
\textbf{Insight 3}: The transformed proof terms can then be translated back to tactics.
\end{quote}

The \textbf{Decompile} component implements a prototype of this translation:
it accepts a proof term and generates a candidate proof script that attempts to prove the same theorem.
Of course, this problem is not very well defined: there is a proof script that always 
works (applying the entire proof term with the \lstinline{apply} tactic), but is often unreadable.
This is the baseline for success of the decompiler, and the decompiler defaults to this behavior.
From there, the goal of the decompiler is to improve on that baseline as much as possible,
or else produce a candidate proof script that is close enough that the proof engineer can step through it
and manually massage it into something that both works and is maintainable.

The output language for the implementation of \textbf{Decompile} is Ltac, the proof script language for Coq.
Ltac can be confusing to reason about, since Ltac tactics can refer to Gallina terms, and the semantics of Ltac depends both on the
semantics of Gallina and on the implementation of proof search procedures written in OCaml.
To give a sense of how the decompiler works without the clutter of these proof search details, we start by defining a toy
decompiler from CIC$_{\omega}$ to a simple subset of Ltac containing just a few predefined tactics (Section~\ref{sec:first}).
We then explain how we scale that up to the actual implementation (Section~\ref{sec:second}).

\subsection{A Toy Decompiler}
\label{sec:first}

\begin{figure}
\small
\begin{grammar}
<v> $\in$ Vars, <t> $\in$ CIC$_{\omega}$

<p> ::= intro <v> |  rewrite <t> <t> | symmetry | apply <t> | induction <t> <t> \{ <p>, \ldots, <p> \} | split \{ <p>, <p> \} | left | right | <p> . <p>
\end{grammar}
\caption{Qtac syntax.}
\label{fig:ltacsyntax1}
\end{figure}

The toy decompiler takes CIC$_{\omega}$ terms and produces tactics in a toy version of Ltac which we call Qtac.\footnote{Pronounced \textit{cute-tac}.}
The syntax for Qtac is in Figure~\ref{fig:ltacsyntax1}.
Qtac includes hypothesis introduction (\lstinline{intro},
rewriting by equalities (\lstinline{rewrite}), symmetry (\lstinline{symmetry}) of equality,
application of a term to prove the goal (\lstinline{apply}), induction over terms (\lstinline{induction}),
case splitting of conjunctions (\lstinline{split}),
constructors of disjunctions (\lstinline{left} and \lstinline{right}), and
composition (\lstinline{.}).

Unlike in Ltac, in Qtac, \lstinline{induction} and \lstinline{rewrite} always take a motive explicitly, rather than relying on a unification engine.
Smilarly, \lstinline{apply} applies only the function without inferring any arguments, and leaves those arguments to proof obligations.
The implementation reasons about Ltac and so does not make these assumptions.

\begin{figure}
\begin{mathpar}
\mprset{flushleft}
\small
\hfill\fbox{$\Gamma$ $\vdash$ $t$ $\Rightarrow$ $p$}\\

\inferrule[Intro]
  { \Gamma,\ n : T \vdash b \Rightarrow p }
  { \Gamma \vdash \lambda (n : T) . b \Rightarrow \mathrm{intro}\ n.\ p }

\inferrule[Symmetry]
  { \Gamma \vdash H \Rightarrow p }
  { \Gamma \vdash \mathtt{eq\_sym}\ H \Rightarrow \mathrm{symmetry}.\ p } \\

\inferrule[Split]
  { \Gamma \vdash l \Rightarrow p \\ \Gamma \vdash r \Rightarrow q }
  { \Gamma \vdash \mathrm{Constr}(0,\ \wedge)\ l r \Rightarrow \mathrm{split} \{ p, q \}.\ }

\inferrule[Left]
  { \Gamma \vdash H \Rightarrow p }
  { \Gamma \vdash \mathrm{Constr}(0,\ \vee)\ H \Rightarrow \mathrm{left}.\ p }

\inferrule[Right]
  { \Gamma \vdash H \Rightarrow p }
  { \Gamma \vdash \mathrm{Constr}(1,\ \vee)\ H \Rightarrow \mathrm{right}.\ p } \\

\inferrule[Rewrite]
  { \Gamma \vdash H_1 : x = y \\ \Gamma \vdash H_2 \Rightarrow p }
  { \Gamma \vdash \mathrm{Elim}(H_1,\ P) \{ x,\ H_2,\ y \} \Rightarrow \mathrm{symmetry}.\ \mathrm{rewrite}\ P\ H_1.\ p }

\inferrule[Induction]
  { \Gamma \vdash \vec{f} \Rightarrow \vec{p} }
  { \Gamma \vdash \mathrm{Elim}(t,\ P)\ \vec{f} \Rightarrow \mathrm{induction}\ P\ t\ \vec{p} }

\inferrule[Apply]
  { \Gamma \vdash t \Rightarrow p }
  { \Gamma \vdash f t \Rightarrow \mathrm{apply}\ f.\ p }

\inferrule[Base]
  { \\ }
  { \Gamma \vdash t \Rightarrow \mathrm{apply}\ t }
\end{mathpar}
\caption{Qtac decompiler semantics.}
\label{fig:someantics}
\end{figure}

The semantics for the toy decompiler are in Figure~\ref{fig:someantics} (assuming $=$, \lstinline{eq_sym}, $\wedge$, and $\vee$ are defined as in Coq).
As with the real decompiler, the toy decompiler defaults to the naive proof script
that applies the entire proof term with the \lstinline{apply} tactic (\textsc{Base}).
Otherwise, it improves on that behavior by recursing over the proof term and constructing a proof script using a predefined set of tactics.

For the toy decompiler, this is fairly straightforward: Lambda terms become introduction of hypotheses (\textsc{Intro}), since they introduce new bindings
in the environment of the body. Applications of \lstinline{eq_sym} become symmetry of equality (\textsc{Symmetry}).
Constructors of conjunction and disjunction become map to the respective tactics (\textsc{Split}, \textsc{Left}, and \textsc{Right}).
Applications of equality eliminators compose symmetry (to orient the rewrite direction with the goal) with rewrites (\textsc{Rewrite}),
and all other applications of eliminators become induction (\textsc{Induction}).
The remaining applications become apply tactics (\textsc{Apply}).
In all cases, the decompiler recurses on the remaining body, breaking into cases when relevant, until no other preconditions match.
At that point the \textsc{Base} case holds, and we are done.

While the toy compiler is very simple, only a few simple changes are needed
to move this from CIC$_{\omega}$ to Coq.
Furthermore, the result can already handle some of the example proofs \toolname has produced.
The generated proof term of \lstinline{rev_app_distr} with swapped list constructors from Section~\ref{sec:overview},
for example, consists only of induction, rewriting, simplification, and reflexivity.
The proof term for the base case:

\begin{lstlisting}
fun (@\codesimb{(y0 : list A)}@) =>
  (@\codesima{list_rect}@) _ _
    (fun (@\codesima{a l IHl}@) =>
      (@\codesimc{eq_ind_r}@) _ (@\codesimd{eq_refl}@) (@\codesimc{(app_nil_r (rev l) (a::[]))}@))
    (@\codesime{eq_refl}@)
    (@\codesima{y0}@)
\end{lstlisting}
decompiles to this proof script:

\begin{lstlisting}
- (@\codesimb{intro y0.}@) (@\codesima{induction y0 as [a l IHl|]}.@)
  + (@\codesimc{simpl. rewrite (app_nil_r (rev l) (a::[])).}@) (@\codesimd{reflexivity.}@)
  + (@\codesime{reflexivity.}@)
\end{lstlisting}
where corresponding terms and tactics are highlighted with the same color, and nothing else is highlighted for clarity.
Of course, this script is fairly low-level and close to the proof term, but it is already something that the proof engineer
can step through piece by piece to understand and maintain the proof.
There are very few differences from the toy decompiler needed to produce this,
for example handling of rewrites in both directions (\lstinline{eq_ind_r} as opposed to \lstinline{eq_ind}),
simplifying to handle motive inference for rewrites,
and turning applications of \lstinline{eq_refl} into reflexivity.

In fact, since \toolname uses the \lstinline{Preprocess} command from \textsc{Devoid}, \textit{all} of the proof terms that \toolname
produces will use induction and rewriting rather than fixpoints and pattern matching.
Because we have control over output terms, even a toy decompiler gets us pretty far.

% TODO add any new things RanDair implements, like exists

\subsection{Scaling Up}
\label{sec:second}

The toy decompiler abstracts a lot of the details that make Ltac so useful to proof engineers---and so painful to 
reason about automatically.
This section discusses how we scale up from this toy decompiler to a prototype Gallina to Ltac decompiler,
and how we imagine the decompiler continuing to evolve from there.

\paragraph{Second Pass}
The toy decompiler reasons about tactics one subterm at a time, and produces tactics only from a predefined set of tactics.
This does not always match the thought processes of proof engineers.
To produce a more natural set of tactics, \textbf{Decompile} operates in two passes: first it runs something that looks a lot
like the toy decompiler, and then it modifies those tactics to produce a more natural proof script.
For example, the first pass, like the toy decompiler, produces a single \lstinline{intro} per hypothesis in a lambda abstraction;
the second pass combines each sequence of \lstinline{intro} tactics into an \lstinline{intros} tactic.

The prototype decompiler still has no special reasoning for decision procedures and custom tactics, which are common
in some proof engineering styles.
This second pass will be the natural integration point for these.
Our current plan is to take additional input in this pass, either directly from the proof engineer
or by feeding in the original proof script from before \textbf{Transform}.
We can then iteratively replace tactics with custom tactics and decision procedures, and check the result see if it works.
We also plan to support tacticals like \lstinline{;} and \lstinline{try}.
For now, massaging the output to use these is left to the proof engineer.

\paragraph{Induction and Rewriting}
The toy decopmiler includes simpler and more predictable versions of \lstinline{rewrite} and \lstinline{induction}
than those found in Coq. The implementation of \textbf{Decompile} includes additional logic to reason about these tactics.
For example, it assumes that there is only one \lstinline{rewrite} direction. Coq has two rewrite directions,
and so the decompiler infers the right direction based on the motive used.

It also assumes that both tactics take the inductive motive explicitly.
In Coq, however, both tactics infer the motive automatically.
Consequentially, Coq will sometimes infer the wrong motive without manipulation of goals and hypotheses,
or will fail to infer a motive at all.
This is especially common for the \lstinline{rewrite} tactic, which is purely syntactic.
To handle induction, the decompiler strategically use the \lstinline{revert} tactic to manipulate the goal
so that Coq can better infer the motive.
To handle rewrites, it uses the \lstinline{simpl} tactic to refold the goal before rewriting.
Neither of these approaches are guaranteed to work, so the proof engineer may sometimes need to tweak the output proof script appropriately.
We have found that even if we pass Coq's induction principle an explicit motive, Coq still sometimes fails due
to unrepresented assumptions.
Long term, using another tactic like \lstinline{change} to manipulate hypotheses and goals before applying these tactics
may help with cases for which Coq cannot infer the correct motive.

\paragraph{Manipulating Hypotheses}
Changing from Qtac to Ltac is not the only challenge in writing the decompiler---we also scale from CIC$_{\omega}$ to Coq.
This introduces let bindings, which are generated by tactics like \lstinline{rewrite in}, \lstinline{apply in}, and \lstinline{pose}.
\textbf{Decompile} implements support for \lstinline{rewrite in} and \lstinline{apply in} similarly to how it implements support for
\lstinline{rewrite} and \lstinline{apply}, but with two differences:

\begin{enumerate}
\item it ensures that the unmanipulated hypothesis does not occur in body of the let expression,
\item it swaps the direction of the rewrite, and
\item it checks for generated subgoals and recurses into those subgoals.
\end{enumerate}
In all other cases, the implementation uses the \lstinline{pose} tactic, a catch-all for let bindings.

\paragraph{Pretty Printing}
After decompiling proof terms, there is one final step to present the information to the proof engineer: pretty printing.
Like the toy decompiler, the implementation of \textbf{Decompile} represents its output language using a predefined grammar of Ltac tactics,
albeit one that is larger than Qtac.
It maintains the recursive proof structure as it goes, then uses that proof structure to print proofs of subgoals using bullet points.
It displays the resulting proof script to the proof engineer, who can then modify it as needed to ensure that it works correctly
and is maintainable.
For convenience, it includes scripts that automate the process of printing all of these tactic proofs to a Coq file,
in case the proof engineer does not want such an interactive workflow.
\toolname keeps all output proof terms from the proof term transformation in the Coq environment as a fallback in case the decompiler does not succeed.
Once the proof engineer has this new proof, she can remove the old specifications, functions, and proofs, using the repaired
versions from then on.



\section{Example Configurations}
\label{sec:search}

combining the case study and configuration ones: guide by case studies, then explain configuration

This is basically the ``common interface'' the journal paper of Univalent Parametricity mentions, except they say it
doesn't scale to automation. In fact it can if you automate entire classes of relations between types rather than
force the user to do it on an ad-hoc basis or try to write a tool that solves the general case. So that is what
we are doing here.

\subsection{Industrial Use}

\subsection{Free Dependent Types}

\subsection{Unary and Binary Numbers}

\subsection{REPLICA Benchmarks}

Next, for the four classes of equivalences we support (maybe more if we have time), we define the constructor, eliminator, and identity rules.
We prove that each of these satisfies the correctness criteria.
Our implementation also (or alternatively, depending on how much time we have) generates these proofs on an ad-hoc basis because
the general proof isn't possible within Coq.

TODO bin to nat


\subsection{Four Transformations}

% TODO oracles, naming, consider just doing code and generating proofs in Coq

Here are the constructor, eliminator, and identity rules for our four implemented transformations (very informal WIP).
The equivalence between $A$ and $B$ can be constructed in terms of these.
This shows just one direction---the opposite is similar.

\subsubsection{Swap}

Let $A$ and $B$ be inductive types:

\begin{lstlisting}
$A$ := $\mathrm{Ind} (\mathit{Ty}_A : \Pi (\vec{i_A} : \vec{\mathrm{X}_A}) . \mathrm{s}_A)\{\mathrm{C}_{A_1}, \ldots, \mathrm{C}_{A_n}\}$
$B$ := $\mathrm{Ind} (\mathit{Ty}_B : \Pi (\vec{i_B} : \vec{\mathrm{X}_B}) . \mathrm{s}_B)\{\mathrm{C}_{B_1}, \ldots, \mathrm{C}_{B_n}\}$
\end{lstlisting}		
Assume there is some invertible swap map $m$ such that for any index $j$,
\lstinline{C}$_{B_{m(j)}}$ is exactly \lstinline{C}$_{A_j}[B / A]$.
Then:

\begin{lstlisting}
DepConstr(j, A) : C$_{A_{j}}$ := Constr(j, A) 
DepConstr(j, B) : C$_{A_{j}}$[B / A] := Constr(m(j), B)

DepElim(a, p){f$_{1}$, $\ldots$, f$_{n}$} : p a := Elim(a, p){f$_{1}$, $\ldots$, f$_{n}$}
DepElim(b, p){f$_{1}$, $\ldots$, f$_{n}$} : p b := Elim(b, p){f$_{m(1)}$, $\ldots$, f$_{m(n)}$}

IdEta(A) := $\lambda$ ($\vec{t}$ : $\vec{T}$) (a : A $\vec{t}$).a
IdEta(B) := $\lambda$ ($\vec{t}$ : $\vec{T}$) (b : B $\vec{t}$).b
\end{lstlisting}

\subsubsection{Algebraic}

It is straightforward to fit the search algorithm from DEVOID into this framework, and in fact
we can loosen the restriction that the language has primitive projections.
Let $A$ be $A$ from DEVOID, let $B_{ind}$ be $B$ from DEVOID, let $I_B$ be $I_B$ from DEVOID,
and let \lstinline{index} be \lstinline{index} from DEVOID.
Let $B$ wrap $B_{ind}$ packed into a sigma type:

\begin{lstlisting}
B := $\lambda$ ($\vec{t}$ : $\vec{T}$) . ($\Sigma$ (i : I$_B$ $\vec{t}$) . B$_{ind}$ (index i $\vec{t}$))
\end{lstlisting}
Let $\vec{T_{B_j}}$ be the arguments of constructor type $C_{B_j}$ (type of constructor of $B_{\mathrm{ind}}$).
Define \lstinline{DepConstr(j, B)} recursively using the following derivation (based on and same fall-through convention as the DEVOID paper for now,
and I'd prefer to move this away from a derivation but not sure how to do so and maintain formality): % TODO check

\begin{mathpar}
\mprset{flushleft}
\small
\hfill\fbox{$\Gamma$ $\vdash$ $(T_A, T_B)$ $\Downarrow_{C}$ $t$}\\

\inferrule[Dep-Constr-Conclusion]
  { \Gamma \vdash \vec{t_{B_j}} : \vec{T_{B_j}} \\ \Gamma \vdash Constr(j, B)\ \vec{t_{B_j}} : B_{\mathrm{ind}} \vec{i_B}  }
  { \Gamma \vdash (A\ \vec{i_A},\ B_{\mathrm{ind}}\ \vec{i_B}) \Downarrow_{p_{c}} \exists\ (\vec{i_B}[\mathrm{off}\ A\ B]) (Constr(j, B)\ \vec{t_{B_j}}) }

\inferrule[Dep-Constr-Index] % new hypothesis for index
  { \mathrm{new}\ n_B\ b_B \\ \Gamma,\ n_B : t_B \vdash (\Pi (n_A : t_A) . b_A,\ b_B) \Downarrow_{i_{c}} t }
  {  \Gamma \vdash (\Pi (n_A : t_A) . b_A,\ \Pi (n_B : t_B) . b_B) \Downarrow_{C} t}

\inferrule[Dep-Constr-IH] % inductive hypothesis
  { \Gamma,\ n_B : B\ \vec{i_B} \vdash (b_A [n_B / n_A], b_B [\pi_l\ n_B / \vec{i_B}[\mathrm{off}\ A\ B]]) \Downarrow_{C} t }
  { \Gamma \vdash (\Pi (n_A : A\ \vec{i_A}) . b_A, \Pi (n_B : B\ \vec{i_B}) \Downarrow_{C} \lambda (n_B : B\ \vec{i_B}) . t }

\inferrule[Dep-Constr-Prod] % otherwise, unchanged (when we get rid of the gross fall-through thing, needs not new, and needs to check t_A and t_B not IHs)
  { \Gamma,\ n_B : t_B \vdash (b_A [n_B / n_A], b_B) \Downarrow_{C} t }
  { \Gamma \vdash (\Pi (n_A : t_A) . b_A, \Pi (n_B, t_B) . b_B) \Downarrow_{C} \lambda (n_B : t_B) . t }\\

\inferrule[Dep-Constr]
{ \Gamma \vdash Constr(j, A) : C_{A_j} \\ \Gamma \vdash (C_{A_j}, C_{B_j}) \Downarrow_{C} t }
{ \Gamma \vdash (Constr(j, A), Constr(j, B_{\mathrm{ind}}) \Downarrow_{C} t }
\end{mathpar}
and \lstinline{DepElim(b, p)} similarly:

\begin{mathpar}
TODO
\end{mathpar}

Then:

\begin{lstlisting}
DepConstr(j, A) : C$_{A_{j}}$ := Constr(j, A)
DepConstr(j, B) : C$_{A_{j}}$[B / A] := DepConstr(j, B)

DepElim(a, p){f$_{1}$, $\ldots$, f$_{n}$} : p a := Elim(a, p){f$_{1}$, $\ldots$, f$_{n}$}
DepElim(b, p){f$_{1}$, $\ldots$, f$_{n}$} : p b := DepElim(b, p)

IdEta(A) := $\lambda$(a : A).a
IdEta(B) := $\lambda$(b : B).$\exists$ ($\pi_l$ b) ($\pi_r$ b)
\end{lstlisting}

% TODO investigate below projection thing, and write in when you finish
%For now assume we have some \lstinline{pack} function to pack into an existential;
%this is just for convenience.
%The indexer is just the first projection of this lifted across the eliminator rule, AFAIK---note this isn't exactly $\Pi_{l}$ like we use
%in the tool, but is really an eliminated $\Pi_{l}$? I will need to check on this, it's the only weird part.
%Also assume some \lstinline{index_args} function to add the new index to the appropriate arguments---I'll
%elaborate on this later but it's also something search needs to find and it's determined in terms of the \lstinline{indexer} that search finds.
%Also now, we no longer assume primitive projections.

\subsubsection{Unpack sigma}

This one is kind of weird but it gets us user-friendly types. I'll explain later.

\begin{lstlisting}
DepConstr(j, A) := (* TODO pack into existential, deal with equality *)
DepConstr(j, B) : C$_{B_{j}}$ := Constr(j, B)

DepElim(a, p){f$_{1}$, $\ldots$, f$_{n}$} : p a := (* TODO *)
DepElim(b, p){f$_{1}$, $\ldots$, f$_{n}$} : p b := Elim(b, p){f$_{1}$, $\ldots$, f$_{n}$}

IdEta(A) := $\lambda$(a : A).$\exists$ ($\exists$ ($\pi_l$ ($\pi_l$ a)) ($\pi_r$ ($\pi_l$ a))) ($\pi_r$ a)
IdEta(B) := $\lambda$(b : B).b
\end{lstlisting}

\subsubsection{Records and tuples}

This one should be easier. We'll play a similar trick with $B$ and $B_{ind}$ like we do for algebraic,
and give things similar names.
Then:

\begin{lstlisting}
DepConstr(j, A) : C$_{A_{j}}$ := Constr(j, A)
DepConstr(j, B) : C$_{A_{j}}$[B / A] := $\lambda$ ($\vec{t_{A_j}}$ : $\vec{T_{A_j}}$) . (* TODO recursively pack into pair *)

DepElim(a, p){f$_{1}$, $\ldots$, f$_{n}$} : p a := Elim(a, p){f$_{1}$, $\ldots$, f$_{n}$}
DepElim(b, p){f$_{1}$, $\ldots$, f$_{n}$} : p b := (* TODO recursively eliminate product *)

IdEta(A) := $\lambda$(a : A).a
IdEta(B) := (* TODO recursive eta *)
\end{lstlisting}

\section{Related Work}

\paragraph{Proof Reuse}

esp. theorem reuse by proof term transformation, DEVOID, univalent parametricity, transport. \cite{magaud2000changing} have a transformation.
leo congruence tactic and how it relates to Bas' tactic, and how that can be used for automatic transport.

\paragraph{Proof Evolution}

including good design

\paragraph{Program Evolution}

including a discussion about program repair

\section{Conclusions \& Future Work}
\label{sec:discussion}

We showed how to combine a configurable proof term transformation with a tactic decompiler to build \toolname,
a proof repair and reuse tool that is flexible and useful for real proof engineering scenarios.
\toolname has helped an industrial proof engineer integrate Coq with a company workflow,
and has supported benchmarks common in the proof engineering community.

Moving forward, our goal is to make proofs easier to repair and reuse regardless of proof engineering expertise.
We want to reach more proof engineers, and we want \toolname to integrate seamlessly with Coq.
We conclude with a discussion of some of the challenges that remain.

% encountered in scaling up the \toolname proof term 
%transformation (Section~\ref{sec:problems}), and how we believe ideas from the rewrite system and constraint
%solver communities can address those challenges and improve the state of the art in proof engineering (Section~\ref{sec:egraph}).
%Our hope is to inspire research communities to come together and open the door to better tools for proof reuse and repair.

\paragraph{Future Work}

We encountered three challenges in scaling up the \toolname transformation:

\begin{enumerate}
\item \textbf{Multiple Equivalences}: Deciding when to run the transformation rules is left to the implementation.
\toolname automates the most basic case of this: changing \textit{every} occurrence of \A to \B.
This can lead to confusing or undesired behavior, especially when there are multiple matching equivalences for a subterm.
The ideal would be a type-directed search procedure.
\item \textbf{Nontermination}: Naively applying the transformation can result in nontermination when the output type refers to the input type.
\toolname includes termination checks, but these termination checks are ad-hoc and do not capture every potentially nonterminating use case.
\item \textbf{Custom Rewrite Rules}: \toolname with manual configuration is not always smart enough to match the appropriate rule in the proof term
transformation without the proof engineer manually expanding the input term.
Proof engineers would benefit from the ability to add custom rewrite rules to \toolname
and expand the input terms automatically.
\end{enumerate}
We hope to solve all three of these challenges elegantly and efficiently using \textit{e-graphs}~\cite{egraph1},
a data structure that is used in the constraint solver and rewrite system communities for managing equivalences.
E-graphs are built specifically to deal with multiple equivalences,
remove the burden of ad-hoc reasoning about termination,
and make it simple for anyone to extend a system with new
rewrite rules. These new rewrite rules can even call out to external procedures~\cite{egraph5}.
The one hurdle is to adapt them to use a univalent definition of equality.
In cubical type theory, this adaptation is simple~\cite{egraph6}; we hope to repurpose this insight.

Beyond that, we believe that the biggest gains will come from continuing to improve the prototype decompiler.
Two particularly helpful features would be support for common search procedures and support for custom tactics.
We hope to use user input to guide this process, as analyzing Ltac directly is unlikely to be fruitful.
Some improvements could come from better tactics themselves---like better handling of explicitly passed 
motives in the \lstinline{induction} tactic, or a more structured tactic language.
With that, we believe that the future of seamless and powerful proof repair and reuse for all is within reach.
We hope you will join us in bringing it to life.





%% Acknowledgments
\begin{acks}                            %% acks environment is optional
                                        %% contents suppressed with 'anonymous'
  %% Commands \grantsponsor{<sponsorID>}{<name>}{<url>} and
  %% \grantnum[<url>]{<sponsorID>}{<number>} should be used to
  %% acknowledge financial support and will be used by metadata
  %% extraction tools.
  This material is based upon work supported by the
  \grantsponsor{GS100000001}{National Science
    Foundation}{http://dx.doi.org/10.13039/100000001} under Grant
  No.~\grantnum{GS100000001}{nnnnnnn} and Grant
  No.~\grantnum{GS100000001}{mmmmmmm}.  Any opinions, findings, and
  conclusions or recommendations expressed in this material are those
  of the author and do not necessarily reflect the views of the
  National Science Foundation.
\end{acks}


%% Bibliography
\bibliography{paper.bib}


%% Appendix
%\appendix
%\section{Appendix}

%Text of appendix \ldots

\end{document}
