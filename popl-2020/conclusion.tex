\section{Conclusions \& Future Work}
\label{sec:discussion}

We showed how to combine a configurable proof term transformation with a tactic decompiler to build \toolname,
a proof repair and reuse tool that is flexible and useful for real proof engineering scenarios.
\toolname has helped an industrial proof engineer integrate Coq with a company workflow,
and has supported benchmarks common in the proof engineering community.

Moving forward, our goal is to make proofs easier to repair and reuse regardless of proof engineering expertise.
We want to reach more proof engineers, and we want \toolname to integrate seamlessly with Coq.
We conclude with a discussion of some of the challenges that remain.

% encountered in scaling up the \toolname proof term 
%transformation (Section~\ref{sec:problems}), and how we believe ideas from the rewrite system and constraint
%solver communities can address those challenges and improve the state of the art in proof engineering (Section~\ref{sec:egraph}).
%Our hope is to inspire research communities to come together and open the door to better tools for proof reuse and repair.

\paragraph{Future Work}

We encountered three challenges in scaling up the \toolname transformation:

\begin{enumerate}
\item \textbf{Multiple Equivalences}: Deciding when to run the transformation rules is left to the implementation.
\toolname automates the most basic case of this: changing \textit{every} occurrence of \A to \B.
This can lead to confusing or undesired behavior, especially when there are multiple matching equivalences for a subterm.
The ideal would be a type-directed search procedure.
\item \textbf{Nontermination}: Naively applying the transformation can result in nontermination when the output type refers to the input type.
\toolname includes termination checks, but these termination checks are ad-hoc and do not capture every potentially nonterminating use case.
\item \textbf{Custom Unification Heuristics}: \toolname with manual configuration is not always smart enough to match the appropriate rule in the proof term
transformation without the proof engineer manually expanding the input term.
Proof engineers would benefit from the ability to add custom unification heuristics to \toolname
and expand the input terms automatically.
\end{enumerate}
We hope to solve all three of these challenges elegantly and efficiently using \textit{e-graphs}~\cite{egraph1},
a data structure that is used in the constraint solver and rewrite system communities for managing equivalences.
E-graphs are built specifically to deal with multiple equivalences,
remove the burden of ad-hoc reasoning about termination,
and make it simple for anyone to extend a system with new
rewrite rules---even ones that can call out to external procedures~\cite{egraph5} 
like unification heuristics implemented in OCaml.
The one hurdle is to adapt them to use a univalent definition of equality.
In cubical type theory, this adaptation is simple~\cite{egraph6}; we hope to repurpose this insight.

Beyond that, we believe that the biggest gains will come from continuing to improve the prototype decompiler.
Two particularly helpful features would be support for common search procedures and support for custom tactics.
We hope to use user input to guide this process, as analyzing Ltac directly is unlikely to be fruitful.
Some improvements could come from better tactics themselves---like better handling of explicitly passed 
motives in the \lstinline{induction} tactic, or a more structured tactic language.
With that, we believe that the future of seamless and powerful proof repair and reuse for all is within reach.
We hope you will join us in bringing it to life.



