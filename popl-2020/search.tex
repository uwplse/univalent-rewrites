\section{Case Studies: \textsc{Carrot} Four Ways}
\label{sec:search}

This section explans four different case studies in using \toolname:

\begin{enumerate}
\item Refactoring automatically generated proofs for an industrial user (automatic, Section~\ref{sec:industry})
\item Generating dependently typed vector functions and proofs from functions and proofs about lists and their lengths (automatic, Section~\ref{sec:dep})
\item Supporting a benchmark from a user study of Coq proof engineers (automatic, Section~\ref{sec:replica})
\item Porting functions and proofs about unary numbers to functions and proofs about binary numbers (manual, Section~\ref{sec:bin})
\end{enumerate}
For each case study, we explain the configuration used, walk through an example, and describe takeaways.
In all, we found the following:

\begin{enumerate}
\item \toolname handled examples of both proof refactoring and repair. In practice, the line between these was sometimes
blurred, as was the line with proof reuse.
\item \toolname was configurable to different classes of changes. Each search procedure for automatic configuration
took between two days and two weeks to write, and handled an entire class of equivalences corresponding to a real use case.
Manual configuration was possible for an interesting use case, but remained challenging.
\item \toolname had good enough workflow integration to support real users.
The tactic decompiler was far from perfect, but showed promising early results with clear paths to improvements.
Some user workflows were unanticipated and informed changes in the design of \toolname.
\end{enumerate}
% TODO is it a contribution in itself to have real observation of users in a 3P proof refactoring/repair tool?

%\begin{enumerate} TODO
%\item LOC for each equiv/structure
%\item Choosing the swap equivalence
%\item Fun eliminators
%\end{enumerate}

\subsection{Industrial Use}
\label{sec:industry}

An industrial user at \company\footnote{Name withheld for double-blind review.} has been using \toolname in proving
correct an implementation of the TLS handshake protocol.
While this is ongoing work, thus far,
\toolname has helped \company integrate Coq with their existing verification workflow.

Before contact, \company had been using a custom solver-aided verification language to prove correct C programs.
They had found that at times, those constraint solvers got stuck, and they could not
progress on proofs about those programs.
They had built a compiler that translates their solver-aided language into Coq's specification language Gallina,
that way their proof engineers could finish stuck proofs interactively using Coq.
However, they had found that the generated Gallina programs and specifications were sometimes too difficult to work with.

A proof engineer at \company has used \toolname to work with those automatically generated programs and specifications
with the following workflow:

\begin{enumerate}
\item First, the proof engineer uses \toolname to refactor the automatically generated programs and specifications into more
human-readable programs and specifications.
\item Next, the proof engineer writes Coq proofs about the more human-readable programs and specifications.
\item Finally, the proof engineer uses \toolname again to refactor those proofs about human-readable programs and specifications back to
proofs about the original automatically generated programs and specifications.
\end{enumerate}
This workflow has allowed for industrial integration with Coq and has helped the user write functions and proofs
that would have otherwise been difficult.

% TODO will add better proofs here once Val sends them

\subsubsection{Configuration}

\begin{figure}
\begin{minipage}{0.25\textwidth}
   \lstinputlisting[firstline=1, lastline=4]{records.tex}
\end{minipage}
\hfill
\begin{minipage}{0.74\textwidth}
   \lstinputlisting[firstline=6, lastline=9]{records.tex}
\end{minipage}
\caption{Two unnamed tuples (left) and corresponding named records (right).}
\label{fig:records}
\end{figure}

Minimal examples corresponding to the proofs that the proof engineer used \toolname for
can be found in \lstinline{minimal_records.v}.
The proof engineer used \toolname to port anonymous tuples produced by \company's compiler
to named records, as shown in the example in Figure~\ref{fig:records}.

We implemented a search procedure for the proof engineer to automatically configure the proof term transformation to an equivalence
between nested tuples and named records.
The search procedure triggered automatically when the proof engineer called the \lstinline{Repair} or \lstinline{Repair module} command
with the tuple and record as arguments.
It set \A to be the record type and \B to be the tuple.
There were no \lstinline{RewEta} since there were no inductive hypotheses.
For the record \A, it set \lstinline{IdEta} to identity, \lstinline{DepConstr} to the single
constructor corresponding to the record constructor, and \lstinline{DepElim} to the standard record eliminator.
For the tuple \B, on the other hand, it expanded identity to deal with non-primitive projections,
in our example:
\begin{lstlisting}
Definition id_eta_B (H : Tuple.input) : Tuple.input :=
  (fst H, (fst (snd H), snd (snd H))).
\end{lstlisting}
it constructed a nested tuple by recursively applying the pair constructor:

\begin{lstlisting}
Definition dep_constr_0_B (firstBool : bool) (numberI : nat) (secondBool : bool) : Tuple.input :=
  (firstBool, (numberI, secondBool)).
\end{lstlisting}
and it recursively eliminated over the pair:

\begin{lstlisting}
Definition dep_elim_B (P : Tuple.input -> Type) (f : $\forall$ f n s, P (dep_constr_0_B f n s)) (i : Tuple.input) : P (id_eta_B i) :=
  prod_rect
    (fun p : bool * (nat * bool) => P (id_eta_b p))
    (fun (a : bool) (b : nat * bool) =>
      f a (fst b) (snd b))
    i.
\end{lstlisting}
This induced an equivalence between the nested tuple and record,
which \toolname generated and proved automatically.

\subsubsection{Example}
Using this configuration, the proof engineer automatically ported this compiler-generated function:

\begin{lstlisting}
Definition op (r : bool * (nat * bool)) : nat * bool :=
  (fst (snd r), andb (firstBool r) (secondBool r)).
\end{lstlisting}
to this function:

\begin{lstlisting}
Definition (@\codeauto{op}@) (r : Record.input) : Record.output :=
  {|
     numberO := numberI r;
     andBools := andb (fistBool r) (secondBool r)
  |}.
\end{lstlisting}
The proof engineer then wrote this proof about the new specification: % TODO did val write this or did I?

\begin{lstlisting}
Theorem and_spec_true_true :
  forall (r : Record.input),
    firstBool r = true ->
    secondBool r = true ->
    andBools (Record.op r) = true.
Proof.
  destruct r as [f n s].
  unfold Record.op.
  simpl in *.
  apply andb_true_intro.
  intuition.
Qed.
\end{lstlisting}
and then used \toolname to automatically port this proof back to a proof of the original function
(with a bit of massaging to clean up the tactic output): % TODO induction & preprocess though

\begin{lstlisting}
Theorem (@\codeauto{and_spec_true_true}@) :
  forall (r : Tuple.input),
    fst r = true ->
    snd (snd r) = true ->
    andb (Tuple.op r) = true.
Proof.
  destruct r as [ a b].
  apply andb_true_intro.
  intuition.
Qed.
\end{lstlisting}

\subsubsection{Takeaways}

\paragraph{Refactoring \& Repair}:
The proof engineer used \toolname for both refactoring and repair.
Refactorings like the changes from tuples to records were the most common use cases.
The proof engineer has also used \toolname to repair some non-dependently-typed
functions and proofs to instead use dependent types.

\paragraph{Configuration \& Flexibility}:
The proof engineer used at least three automatic configurations and no manual configurations.
One of these configurations was backed by a search procedured that we implemented
specifically to support this user: the search procedure to
configure the equivalence between nested tuples and records.
This search procedure took about two weeks for us to implement using techniques from
the repair and reuse tools \textsc{Pumpkin Patch} and \textsc{Devoid}.
Flexibility could be further improved by exposing an interface to users to allow them to
write these search procedures themselves.

\paragraph{Workflow Integration}:
The industrial proof engineer was able to use \toolname to integrate Coq with existing proof engineering
workflows using solver-aided tools at \company.
The workflow for using \toolname itself, however, was a bit nonstandard,
and there was little need for tactic proofs about the compiler-generated functions and specifications.
In the initial days, we worked closely with the proof engineer;
later, the proof engineer worked independently and reached out occasional by email.
\toolname was usable enough for this to work.
However, we found two challenges with workflow integration:
the proof engineer sometimes could not distinguish between user errors and bugs in our code,
and the proof engineer typically waited only about 10 seconds at most for \toolname
to port a function or proof.
Both of these observations informed significant changes to \toolname, like better error messages
and caching of transformed subterms.
% TODO how long did compiling the file take?

\subsection{Vectors from Lists and their Lengths}
\label{sec:dep}

The proof term transformation in \toolname is based on the proof term transformation from
the \textsc{Devoid} proof reuse tool.
\textsc{Devoid} is a proof reuse tool for \textit{algebraic ornaments}~\cite{mcbride}. Algebraic ornaments describe relations
between two inductive types, where one inductive type is exactly the old inductive type indexed by a fold
over the original type.
The running example of this in the \textsc{Devoid} paper is the relation between a list and a
length-indexed vector, like we saw in Figure~\ref{fig:listtovect} in Section~\ref{sec:key1}.

We configured the generalized algorithm in \toolname to support algebraic ornaments like those found in \textsc{Devoid},
and passed all of the regression tests from \textsc{Devoid}.
We found that this simplified the algorithm from \textsc{Devoid}.
In addition, we added one more configuration to automate effort that had been manual in \textsc{Devoid}.
Several proof engineers including our industrial proof engineer, a Reddit user,
and someone on the \lstinline{coq-club} message board contacted us expressing interest in using this functionality,
though we do not yet know if the latter two ended up using \toolname. % TODO honestly ask

\subsubsection{Configuration}

We used two configurations to easy development with dependent types using algebraic ornaments.
The first fits the proof term transformation from \textsc{Devoid} into the \toolname framework:
it sets \A to the unornamented type and \B to the ornamented type at \textit{some} index---\textsc{Devoid}
calls the resulting type \textit{packed}.
In the case of \lstinline{list} and \lstinline{vector}, that configuration
transports proofs across this equivalence:

\begin{lstlisting}
list T $\simeq$ $\Sigma$ (n : nat) . vector T n
\end{lstlisting}
The second configuration goes beyond the functionality from \textsc{Devoid}, and provides
the missing link to get proofs about \B at a \textit{particular} index:

\begin{lstlisting}
{ l : list T & length l = n } $\simeq$ vector T n
\end{lstlisting}
producing something \textit{unpacked},
a step that \textsc{Devoid} had left to the user.

The search procedure for the first configuration is based heavily on the search procedure from \textsc{Devoid}.
Since \textsc{Devoid} supports ornaments which by definition represent types with the same inductive structure,
\lstinline{RewEta} for both \A and \B was always reflexivity.
For the inductive type \A, \lstinline{IdEta}, \lstinline{DepConstr}, and \lstinline{DepElim}
correspond to the identity function, the constructors of \A, and the eliminator for \A, respectively.
For \B, \lstinline{IdEta} expands the input to avoid reliance on primitive projections,
in our example:

\begin{lstlisting}
Definition id_eta_B (T : Type) (s : $\Sigma$ (n : nat) . vector T n) : $\Sigma$ (n : nat) . vector T n  :=
  $\exists$ ($\pi_l$ s) ($\pi_r$ s).
\end{lstlisting}
This implements \textit{repacking} from \textsc{Devoid}.
Like in \textsc{Devoid},
\lstinline{DepConstr} packs constructors of \B: % TODO shrink this now that some of it is in the other section

\begin{lstlisting}
Definition dep_constr_B_0 (T : Type) : $\Sigma$ (n : nat) . vector T n :=
  $\exists$ 0 (Vector.nil A).

Definition dep_constr_B_1 (T : Type) (t : T) (s : $\Sigma$ (n : nat) . vector T n) : $\Sigma$ (n : nat) . vector T n :=
  $\exists$ (S ($\pi_l$ s)) (Vector.cons ($\pi_l$ s) t ($\pi_r$ s)).
\end{lstlisting}
and \lstinline{DepElim} eliminates its projections:

\begin{lstlisting}
Definition dep_elim_B_0 (T : Type) (P : $\Sigma$ (n : nat) . vector T n -> Type) (pnil : P (dep_constr_B_0 T)) (pcons : $\forall$ t s, P (id_eta_B T s) -> P (dep_constr_B_1 T t s)) (s : $\Sigma$ (n : nat) . vector T n) : P (id_eta_B T s) :=
  vector_rect
    T
    (fun (n : nat) (v : vector T n) => P ($\exists$ n v))
    pnil
    (fun (t : T) (n : nat) (v : vector T n) => pcons t ($\exists$ n v))
    ($\pi_l$ s)
    ($\pi_r$ s).
\end{lstlisting}
Together this configuration is enough to capture all of the functionality from \textsc{Devoid} for both search and lifting,
save for the optional proof that these induce an equivalence, which \toolname borrows from \textsc{Devoid}.

To get from lists to vectors \textit{at a particular length}, \toolname implements one additional configuration.
This configuration corresponds to the equivalence between packed types at a particular projection
and unpacked types, in our example:

\begin{lstlisting}
{ s : $\Sigma$ (n : nat) . vector T n & $\pi_l$ s = n } $\simeq$ vector T n
\end{lstlisting}
By composition with the initial equivalence, this lets us transport proofs
across the equivalence we eventually want:

\begin{lstlisting}
{ l : list T & length l = n } $\simeq$ vector T n
\end{lstlisting}
since the left-hand sides of these two equivalences are equal up to transport along the first equivalence.

The second configuration is similar to the first configuration, except that it also carries equality proofs over the indices.
That is, it views the type \B like \lstinline{vector T n} as implicitly representing \{\lstinline{ v : vector T n' & n' = n }\} for some \lstinline{n'}.
This is seen, for example, in the identity rule for \B, here: 
% TODO both should take the same number of arguments, even if id_eta_A takes fewer. also does this belong in rew_eta?

\begin{lstlisting}
Definition id_eta_B (T : Type) (n n' : nat) (v : vector T n') (H : n = n') : vector T n :=
  eq_rect n' (vector T) v n H.
\end{lstlisting}
which is the identity function generalized over any equal index.

\subsubsection{Example}

The expanded example from the \textsc{Devoid} paper is in \lstinline{Example.v}.
The \textsc{Devoid} example ported a list \lstinline{zip} function,
a \lstinline{zip_with} function, and a proof \lstinline{zip_with_is_zip} relating the two
functions from lists to packed vectors.
It then manually ports those proofs to proofs over unpacked vectors at a particular index.
The updated \toolname example automates this last step.

The workflow for this was a bit different than it was with \textsc{Devoid}.
First, we used a custom eliminator \toolname generated to combine the list functions
with the length invariants from \textsc{Devoid}, and to combine the list proofs
with the proofs about those length invariants from \textsc{Devoid}.
This gave us a proof of this lemma:

\begin{lstlisting}
Lemma zip_with_is_zip :
  forall A B n (v1 : { l1 : list A & length l1 = n }) (v2 : { l2 : list B & length l2 = n }),
    zip_with pair v1 v2 = zip v1 v2.
\end{lstlisting}
with functions \lstinline{zip_with} and \lstinline{zip} operating over lists at given lengths.
We then ran \lstinline{Repair module} to transport those functions and proofs using the first
configuration, which proved this lemma:

\begin{lstlisting}
Lemma zip_with_is_zip :
  forall A B n (v1 : { l1 : $\Sigma$(n : nat).vector T n & $\pi_l$ l1 = n }) (v2 : { l2 : $\Sigma$(n : nat).vector T n & $\pi_l$ l2 = n }),
    zip_with pair v1 v2 = zip v1 v2.
\end{lstlisting}
with functions \lstinline{zip_with} and \lstinline{zip} operating over vectors at given projections.
We composed this with \lstinline{Repair module} on the second configuration,
which proved this lemma:

\begin{lstlisting}
Lemma zip_with_is_zip :
  forall A B n (v1 : vector A n) (v2 : vector B n),
    zip_with pair v1 v2 = zip v1 v2.
\end{lstlisting}
with functions \lstinline{zip_with} and \lstinline{zip} operating directly over vectors.

\subsubsection{Takeaways}

\paragraph{Refactoring \& Repair}
With the additional information corresponding to the proofs about the indices, this can be viewed
as a form of proof repair. In the example above, this imposes on the user the additional
obligation of showing that the length is equal to a particular index.
\toolname implements some additional automation to generate eliminators that help separate out this additional information
for repair. The line between repair and reuse is blurry here, since users have used this
functionality to get functions and proofs over vectors from functions and proofs defined in the list library.

\paragraph{Configuration \& Flexibility}
Implementing the configuration from \textsc{Devoid} in this framework was straightforward,
took a couple of days, and simplified the code from \textsc{Devoid} significantly.
For example, it removed the need for special rules for handling eliminators and constructors
by adding and removing arguments,
as that work is done in the configuration.
Implementing the second configuration was less straightforward,
especially when it came to understanding the behaviors of equalities.
There are still some open challenges in correctly supporting \lstinline{RewEta} to port any arbitrary
proof backward along the second configuration in the opposite direction.

\paragraph{Workflow Integration}
This simplified the workflow from \textsc{Devoid} and automated steps that had been manual.
Furthermore, the tactic decompiler successfully generated tactic scripts to write dependently typed
functions that carry proofs about length invariants.
However, the decompiler struggled significantly at generating tactic scripts for proofs relating those invariants.
This is to some degree unsurprising, since these tactic proofs
were prohibitively difficult for us to write by hand.
Still, additional effort is needed to improve tactic integration with dependent types.
% TODO how long did compiling the file take?

\subsection{\textsc{Replica} Benchmark}
\label{sec:replica}

\begin{figure}
\begin{minipage}{0.48\textwidth}
   \lstinputlisting[firstline=1, lastline=8]{replica.tex}
\end{minipage}
\hfill
\begin{minipage}{0.48\textwidth}
   \lstinputlisting[firstline=10, lastline=17]{replica.tex}
\end{minipage}
\caption{A simple language (left) and the same language with two swapped constructors (right).}
\label{fig:replica}
\end{figure}

Section~\ref{sec:overview} showed an example of swapping constructors.
This example was inspired by two benchmarks from the \textsc{Replica} user study of proof engineers~\cite{replica}.
An isolated change corresponding to part of the benchmark is shown in Figure~\ref{fig:replica}.
In that change, the proof engineer had a simple language represented by an inductive type \lstinline{Term},
as well as some definitions and proofs about the language.
As part of the change, the proof engineer swapped two constructors in the term language.

Using the search procedure for the swap configuration, we were able to use \toolname
to automatically configure the program transformation to move this constructor,
then transform all of the functions and proofs about the language.
We also succeeded at more difficult variants of this,
like swapping two constructors with the same type, or renaming all of the constructors,
or swapping and renaming at the same time.
In all cases, with just a bit of human guidance, \toolname was able to repair the functions and proofs.

\subsubsection{Configuration}

The configuration this used handles swapping and renaming constructors of inductive types,
inducing an equivalence directly between \A and \B.
This is one of the simplest configurations.
The only nontrivial part is that \lstinline{DepConstr} over \B swaps constructors, in our example,
maintaining the original order of the two swapped constructors:

\begin{lstlisting}
dep_constr_B_1 (H : Z) : Term := Int H.
dep_constr_B_2 (H H0 : Term) : Term := Eq H H0.
\end{lstlisting}
so that they align perfectly with the constructors of \A.
The eliminator similarly swaps cases:

\begin{lstlisting}
Definition dep_elim_B P f0 f1 f2 f3 f4 f5 f6 t : P t :=
  Old.Term_rect P f0 (@\codediff{f2}@) (@\codediff{f1}@) f3 f4 f5 f6 t.
\end{lstlisting}
where \lstinline{Old.Term_rect} is the eliminator over the old version of \lstinline{Term}.

Despite the configuration being simple, the search procedure beneath it is flexible.
For example, it does not make any assumptions about the name of the inductive type or constructor,
so it can handle arbitrary swapping and renaming of constructors at the same time.
It can also support multiple swapped constructors at the same time.
When a swap is ambiguous and there are many possible mappings between the constructors of the old
and new version, \toolname prompts the proof engineer to choose a configuration from a ranked list.
It also allows the user to provide a custom mapping between constructors if the desired mapping
does not show up high enough on the ranked list.
It proves the equivalence from the swap map automatically, even in the case that the user provides
the swap map directly.
Then it transports functions and proofs along the equivalence, swapping and renaming cases of functions
and proofs.

\subsubsection{Example}

We used \toolname to automatically refactor the functions and proofs in \lstinline{Swap.v} from the \textsc{Replica} benchmark.
This included functions about \lstinline{Term}, as well as a large record \lstinline{EpsilonLogic} that encoded the semantics of the language
and a proof of the record:

\begin{lstlisting}
Theorem eval_eq_true_or_false :
  $\forall$ (L : EpsilonLogic) env (t1 t2 : Term),
    L.eval env (Eq t1 t2) = L.vTrue \/ L.eval env (Eq t1 t2) = L.vFalse.
\end{lstlisting}
\toolname was able to update all of these. For the inductive proof, it produced
a new inductive proof that used the induction principle with the swapped constructors.
It also discovered all other 23 type-correct permutations of constructors, all available for selection to 
prove the appropriate equivalence and use to transport functions and proofs along that equivalence.
It presented the desired transformation as the first option in the list, so that all we had to do
was pass the argument \lstinline{mapping 0} to \lstinline{Repair module} for it to handle this refactoring.
It was also able to handle more advance changes like renaming constructors, swapping multiple constructors at once,
and doing both at the same time.

\subsubsection{Takeaways}

\paragraph{Refactoring \& Repair}
This was a simple refactoring.
It was only a single snapshot in the \textsc{Replica} benchmark, which included
other changes that qualify as repairs, like adding constructors to inductive types.
We have not yet implemented search procedures for those repairs.

\paragraph{Configuration \& Flexibility}
Implementing the search procedure for constructor swapping and renaming was very simple and took about three days total.
The biggest challenge was implementing an interactive interface to choose between mappings when there are multiple possible mappings,
or to allow the user to write a single custom mapping function and derive the rest of the configuration from there.
One ongoing challenge is in defining a useful configuration to support other changes from the same benchmark.
For example, we do not yet know what configuration would be most useful for adding new constructors to inductive types,
or even if there should be just one configuration corresponding to that change for a given proof development.

\paragraph{Workflow Integration}
The isolated change was simple enough that \toolname would not have been necessary
for refactoring the proofs for the initial change---keeping the tactics the same would have also worked.
This is mostly a property of the particular proofs that we had access to;
as we saw in Section~\ref{sec:overview}, even for simple changes, this is not always true.
In addition, more advanced variants of this benchmark that involved also renaming constructors did necessitate \toolname,
and \toolname worked well for those. % TODO what happens with the tactic decompiler for these?
The entire \lstinline{Swap.v} file, which includes swapping constructors of every function in the \lstinline{list} module and
its dependencies, refactoring four variants of the \textsc{Replica} benchmark,
and testing a large and ambiguous swap of a type \lstinline{Enum} with 30 constructors of identical types,
took \toolname less than 90 seconds total. % TODO specs, reproduction
Each variant of the \textsc{Replica} benchmark took \toolname less than 5 seconds. % TODO specs, reproduction

\subsection{Unary and Binary Numbers}
\label{sec:bin}

All of the changes that we have seen so far have used automatic search procedures
to find equivialences between types with the same inductive structure, or \textit{ornaments}.
Some of the oldest problems in the transport literature deal with \textit{changing} inductive
structure.
A classic example of this is translating unary natural numbers to binary natural numbers~\cite{magaud2000changing}.
We have realized this functionality with \toolname using a manual configuration. % TODO change now that redundant with earlier thing

The binary numbers (Figure~\ref{fig:nattobin} in Section~\ref{sec:key2}) from the Coq standard library
allow for a fast addition function, also found in the Coq standard library.
In the style of \citet{magaud2000changing}, we use \toolname to derive a slow binary
addition function that does not refer to \lstinline{nat} at all.
From that, we are able to port our theorems over unary addition to binary addition,
removing all references to \lstinline{nat}, and show that they hold over fast binary addition too.

\subsubsection{Configuration}
We supplied this configuration manually using the \lstinline{Configure} command,
which takes the configuration parts as arguments.
The result is in \lstinline{nonorn.v}.
The configuration for \lstinline{nat} was straightforward.
For \lstinline{N}, we used functions from the Coq standard library that
behaved like the \lstinline{nat} constructors:

\begin{lstlisting}
dep_constr_0_B : N := 0%N.
dep_constr_1_B : N -> N := N.succ.
\end{lstlisting}
Similarly, the Coq standard library included precisely the eliminator we wanted, one that behaves
like the eliminator over unary natural numbers:

\begin{lstlisting}
Definition dep_elim_B :
  forall (P : N -> Type),
    P 0%N ->
    (forall (n : N), P n -> P (N.succ n)) ->
    forall (n : N), P n
:=
  N.peano_rect.
\end{lstlisting}
\lstinline{RewEta} was almost written for us.
The Coq standard library included a lemma that showed that the dependent eliminator preserved the successor case:

\begin{lstlisting}
N.peano_rect_succ :
  $\forall$ (P : N -> Type) (a : P 0%N) (f : $\forall$ (n : N), P n -> P (N.succ n)) (n : N),
    N.peano_rect P a f (N.succ n) = f n (N.peano_rect P a f n).
\end{lstlisting}
From there, \lstinline{RewEta} over the successor case is a simple rewrite, as we saw in Section~\ref{sec:equality}:

\begin{lstlisting}
Lemma rew_eta_1_B :
  $\forall$ (P : N -> Type) (a : P 0%N) (PS : $\forall$ (n : N), P n -> P (N.succ n)) (n : N) (Q : P (N.succ n) -> Type),
     Q (PS n (N.peano_rect P PO PS n)) ->
     Q (N.peano_rect P PO PS (N.succ n)).
Proof.
  intros. rewrite N.peano_rect_succ. auto.
Defined.
\end{lstlisting}

The need for a nontrivial \lstinline{RewEta} comes from the fact that \lstinline{N} has a different
inductive structure from \lstinline{nat}.
The need for it is noted as far back as \citet{magaud2000changing}.
This corresponds to a broader pattern---it captures the essence of the change in inductive structure.
\toolname's configurable proof term transformation captures that intuition.

\subsubsection{Example}

We ported unary addition from \lstinline{nat} to \lstinline{N} fully automatically:

\begin{lstlisting}
Repair nat N in add as slow_add.
\end{lstlisting}
The result (tellingly named) has the same slow behavior as the \lstinline{add} function over \lstinline{nat}.
However, it no longer refers to \lstinline{nat} in any way.
Like \citet{magaud2000changing}, we found it easy to manually prove that
this has the same behavior as fast binary addition:

\begin{lstlisting}
Lemma add_fast_add:
  forall (n m : Bin.nat),
    slow_add n m = N.add n m.
Proof.
  induction n using N.peano_rect; intros m; auto. unfold slow_add.
  rewrite N.peano_rect_succ. (* <- rew_eta_1_B *)
  unfold slow_add in IHn. rewrite IHn.
  rewrite N.add_succ_l.
  reflexivity.
Qed.
\end{lstlisting}

We then used \toolname again to transform a proof:
\begin{lstlisting}
plus_n_Sm : $\forall$ (n m : nat), S (add n m) = add n (S m).
\end{lstlisting}
from \lstinline{add} to \lstinline{slow_add}:

\begin{lstlisting}
slow_plus_n_Sm : $\forall$ (n m : N), N.succ (slow_add n m) = slow_add n (N.succ m).
\end{lstlisting}
This was not quite as push-button.
It involved a manual expansion step, turning implicit casts in the inductive case
into explicit applications of \lstinline{RewEta} over \A.
These applications were formulaic, but tricky to write.
Once we had that, though, we could run the same \lstinline{Repair} command
to get \lstinline{slow_plus_n_Sm}.

Showing that the same theorem held over fast binary addition was then
straightforward:

\begin{lstlisting}
Lemma add_n_Sm :
  forall n m,
    Bin.succ (N.add n m) = N.add n (Bin.succ m).
Proof.
  intros. repeat rewrite <- add_fast_add. apply slow_plus_n_Sm.
Qed.
\end{lstlisting}

\subsubsection{Takeaways}

\paragraph{Refactoring \& Repair}
The change from \lstinline{nat} to \lstinline{N} using \toolname was a refactoring,
but allowed for simple compatibility with functions with better performance.

\paragraph{Configuration \& Flexibility}
\lstinline{RewEta} was the key to supporting this case,
and it was enough to implement this transformation that had previously been its own tool
just by writing a configuration with \lstinline{RewEta}. 

\paragraph{Workflow Integration}
This was another case when we did not need tactic integration, since we
ultimately wanted compatibility with a faster version of the function.
The most difficult part was manually expanding proofs about \lstinline{nat}
to explicitly apply \lstinline{RewEta}.
This is because there is not yet any way for manual configuration to supply custom matching functions,
and unification was not enough.
We discuss some ideas for this in Section~\ref{sec:discussion}. % TODO actually do
The entire file took under a second for us to compile using \toolname.

% TODO what does the tactic decompiler do for this? It's broken. Why?

%\subsubsection{Algebraic}

%It is straightforward to fit the search algorithm from DEVOID into this framework, and in fact
%we can loosen the restriction that the language has primitive projections.
%Let $A$ be $A$ from DEVOID, let $B_{ind}$ be $B$ from DEVOID, let $I_B$ be $I_B$ from DEVOID,
%and let \lstinline{index} be \lstinline{index} from DEVOID.
%Let $B$ wrap $B_{ind}$ packed into a sigma type:

%\begin{lstlisting}
%B := $\lambda$ ($\vec{t}$ : $\vec{T}$) . ($\Sigma$ (i : I$_B$ $\vec{t}$) . B$_{ind}$ (index i $\vec{t}$))
%\end{lstlisting}
%Let $\vec{T_{B_j}}$ be the arguments of constructor type $C_{B_j}$ (type of constructor of $B_{\mathrm{ind}}$).
%Define \lstinline{DepConstr(j, B)} recursively using the following derivation (based on and same fall-through convention as the DEVOID paper %for now,
%and I'd prefer to move this away from a derivation but not sure how to do so and maintain formality): % TODO check

%\begin{mathpar}
%\mprset{flushleft}
%\small
%\hfill\fbox{$\Gamma$ $\vdash$ $(T_A, T_B)$ $\Downarrow_{C}$ $t$}\\%

%\inferrule[Dep-Constr-Conclusion]
%  { \Gamma \vdash \vec{t_{B_j}} : \vec{T_{B_j}} \\ \Gamma \vdash Constr(j, B)\ \vec{t_{B_j}} : B_{\mathrm{ind}} \vec{i_B}  }
%  { \Gamma \vdash (A\ \vec{i_A},\ B_{\mathrm{ind}}\ \vec{i_B}) \Downarrow_{p_{c}} \exists\ (\vec{i_B}[\mathrm{off}\ A\ B]) (Constr(j, B)\ \vec{t_{B_j}}) }

%\inferrule[Dep-Constr-Index] % new hypothesis for index
%  { \mathrm{new}\ n_B\ b_B \\ \Gamma,\ n_B : t_B \vdash (\Pi (n_A : t_A) . b_A,\ b_B) \Downarrow_{i_{c}} t }
%  {  \Gamma \vdash (\Pi (n_A : t_A) . b_A,\ \Pi (n_B : t_B) . b_B) \Downarrow_{C} t}

%\inferrule[Dep-Constr-IH] % inductive hypothesis
%  { \Gamma,\ n_B : B\ \vec{i_B} \vdash (b_A [n_B / n_A], b_B [\pi_l\ n_B / \vec{i_B}[\mathrm{off}\ A\ B]]) \Downarrow_{C} t }
%  { \Gamma \vdash (\Pi (n_A : A\ \vec{i_A}) . b_A, \Pi (n_B : B\ \vec{i_B}) \Downarrow_{C} \lambda (n_B : B\ \vec{i_B}) . t }

%\inferrule[Dep-Constr-Prod] % otherwise, unchanged (when we get rid of the gross fall-through thing, needs not new, and needs to check t_A and t_B not IHs)
%  { \Gamma,\ n_B : t_B \vdash (b_A [n_B / n_A], b_B) \Downarrow_{C} t }
%  { \Gamma \vdash (\Pi (n_A : t_A) . b_A, \Pi (n_B, t_B) . b_B) \Downarrow_{C} \lambda (n_B : t_B) . t }\\

%\inferrule[Dep-Constr]
%{ \Gamma \vdash Constr(j, A) : C_{A_j} \\ \Gamma \vdash (C_{A_j}, C_{B_j}) \Downarrow_{C} t }
%{ \Gamma \vdash (Constr(j, A), Constr(j, B_{\mathrm{ind}}) \Downarrow_{C} t }
%\end{mathpar}
%and \lstinline{DepElim(b, p)} similarly:

%\begin{mathpar}
%TODO
%\end{mathpar}

%Then:

%\begin{lstlisting}
%DepConstr(j, A) : C$_{A_{j}}$ := Constr(j, A)
%DepConstr(j, B) : C$_{A_{j}}$[B / A] := DepConstr(j, B)

%DepElim(a, p){f$_{1}$, $\ldots$, f$_{n}$} : p a := Elim(a, p){f$_{1}$, $\ldots$, f$_{n}$}
%DepElim(b, p){f$_{1}$, $\ldots$, f$_{n}$} : p b := DepElim(b, p)

%IdEta(A) := $\lambda$(a : A).a
%IdEta(B) := $\lambda$(b : B).$\exists$ ($\pi_l$ b) ($\pi_r$ b)
%\end{lstlisting}

% TODO investigate below projection thing, and write in when you finish
%For now assume we have some \lstinline{pack} function to pack into an existential;
%this is just for convenience.
%The indexer is just the first projection of this lifted across the eliminator rule, AFAIK---note this isn't exactly $\Pi_{l}$ like we use
%in the tool, but is really an eliminated $\Pi_{l}$? I will need to check on this, it's the only weird part.
%Also assume some \lstinline{index_args} function to add the new index to the appropriate arguments---I'll
%elaborate on this later but it's also something search needs to find and it's determined in terms of the \lstinline{indexer} that search finds.
%Also now, we no longer assume primitive projections.

%\subsubsection{Unpack sigma}

%This one is kind of weird but it gets us user-friendly types. I'll explain later.

%\begin{lstlisting}
%DepConstr(j, A) := (* TODO pack into existential, deal with equality *)
%DepConstr(j, B) : C$_{B_{j}}$ := Constr(j, B)

%DepElim(a, p){f$_{1}$, $\ldots$, f$_{n}$} : p a := (* TODO *)
%DepElim(b, p){f$_{1}$, $\ldots$, f$_{n}$} : p b := Elim(b, p){f$_{1}$, $\ldots$, f$_{n}$}

%IdEta(A) := $\lambda$(a : A).$\exists$ ($\exists$ ($\pi_l$ ($\pi_l$ a)) ($\pi_r$ ($\pi_l$ a))) ($\pi_r$ a)
%IdEta(B) := $\lambda$(b : B).b
%\end{lstlisting}

%\subsubsection{Records and tuples}

%This one should be easier. We'll play a similar trick with $B$ and $B_{ind}$ like we do for algebraic,
%and give things similar names.
%Then:

%\begin{lstlisting}
%DepConstr(j, A) : C$_{A_{j}}$ := Constr(j, A)
%DepConstr(j, B) : C$_{A_{j}}$[B / A] := $\lambda$ ($\vec{t_{A_j}}$ : $\vec{T_{A_j}}$) . (* TODO recursively pack into pair *)

%DepElim(a, p){f$_{1}$, $\ldots$, f$_{n}$} : p a := Elim(a, p){f$_{1}$, $\ldots$, f$_{n}$}
%DepElim(b, p){f$_{1}$, $\ldots$, f$_{n}$} : p b := (* TODO recursively eliminate product *)

%IdEta(A) := $\lambda$(a : A).a
%IdEta(B) := (* TODO recursive eta *)
%\end{lstlisting}
