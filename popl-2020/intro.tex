\section{Introduction}

Program verification with interactive theorem provers has come a long way since its inception,
especially when it comes to the scale of programs that can be verified.
The seL4~\cite{Klein2009} verified operating system kernel, for example,
is the effort of a team of proof engineers over more than twenty years and spanning more than
a million lines of proof.
Given historical critique of verification~\cite{DeMillo1977} (emphasis ours):

\begin{quote}
A \textit{sufficiently fanatical researcher}
might be willing to devote \textit{two or 
three years} to verifying a significant 
piece of software if he could be 
assured that the software would remain stable.
\end{quote}
we can conclude that, since 1977, either verification has become much easier,
or our researchers have become much more fanatical. Unfortunately, not all has changed (emphasis still ours):

\begin{quote}
But real-life programs need to 
be maintained and modified. 
There is \textit{no reason to believe} that verifying a modified program is any 
easier than verifying the original the 
first time around.
\end{quote}
Tools that can automatically refactor or repair proofs~\cite{wibergh2019, WhitesidePhD, Dietrich2013, adams2015, Bourke12, Roe2016, robert2018, pumpkinpatch}
give us reason to believe that verifying a modified program can sometimes be easier than verifying the original the first time
around, even when proof engineers do not follow good development processes,
or when change occurs outside of proof engineers' control~\cite{PGL-045}.
Still, maintaining verified programs can be challenging: it means keeping not just the programs, but also the
specifications and proofs about those programs up-to-date.
This remains so difficult that sometimes, even experts give up in the face of change~\cite{replica}.

We make progress on two open challenges in \textit{proof repair}, the problem of automatically updating proofs in response
to changes in programs or specifications:

\begin{enumerate}
\item Proof repair tools support very limited classes of changes like non-structural changes~\cite{pumpkinpatch} or a predefined set
of changes~\cite{robert2018, wibergh2019}, and these are not informed by the needs of proof engineers~\cite{replica}.
\item Proof repair tools are not yet integrated with typical proof engineering workflows like tactics~\cite{PGL-045, pumpkinpatch, robert2018}.
\end{enumerate}
Our progress towards these challenges leverages three key insights:

\begin{enumerate}
\item Proof repair is a form of proof reuse---reusing proofs about one specification to derive proofs about another specification---with 
the additional challenge that one of the specifications may cease to exist.
The key to supporting proof repair is to build a proof reuse
tool that can handle that additional challenge (Section~\ref{sec:key1}). 
\item A configurable proof term transformation can be used to build such a proof repair tool,
and the result can handle many different kinds of changes (Section~\ref{sec:key2}).
\item The transformed proof terms can then be translated back to tactic scripts (Section~\ref{sec:decompiler}).
\end{enumerate}

These insights informed our design of 
\toolname\footnote{Real name withheld for double-blind review.} (Configurable Approach to Repairing and Refactoring Outdated Tactics), a plugin for Coq 8.8.
\toolname combines a configurable proof term transformation,
search procedures to configure the proof term transformation,
and a prototype decompiler from proof terms back to tactics.
The result is a flexible proof repair tool that: 

\begin{enumerate}
\item supports changes informed by proof engineers and not supported by other repair tools, and
\item produces tactic scripts as part of better workflow integration.
\end{enumerate}
\toolname can support certain structural changes like porting non-dependent types to certain dependent types
and changing inductive structure---anything described by an equivalence or refinement with a certain form
detailed in Section~\ref{sec:key2}.

Our main technical advances are techniques for transforming proof terms directly to repair broken proofs, 
while our decompiler up to the tactic level is important for usability in Coq.
We demonstrate flexibility and usability with four case studies (Section~\ref{sec:search}), which show that \toolname:

\begin{enumerate}
\item can support variants of a benchmark from a user study of Coq proof engineers,
\item can simplify dependently-typed programming, %automating manual steps from previous work,
\item can help proof engineers port functions and proofs from unary to binary numbers, and
\item has helped an industrial proof engineer integrate Coq with a company workflow and write proofs about an implementation of the TLS Handshake Protocol.
\end{enumerate}
%Our experiences drive ideas (Section~\ref{sec:discussion}) that open the door to better proof engineering tools.

