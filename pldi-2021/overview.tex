\section{A Simple Motivating Example}
\label{sec:overview}

\begin{figure*}
\begin{minipage}{0.46\textwidth}
   \lstinputlisting[firstline=1, lastline=3]{listswap.tex}
\end{minipage}
\hfill
\begin{minipage}{0.46\textwidth}
   \lstinputlisting[firstline=5, lastline=7]{listswap.tex}
\end{minipage}
\vspace{-0.4cm}
\caption{A change from the old version of \lstinline{list} (left) to the new version of \lstinline{list} (right).
The old version of \lstinline{list} is an inductive datatype that is either empty (the \lstinline{nil} constructor), or the result
of placing an element in front of another \lstinline{list} (the \lstinline{cons} constructor). The change swaps these
two constructors (\codediff{orange}).}
\label{fig:listswap}
\end{figure*}

\begin{figure*}
\codeauto{%
\begin{minipage}{0.48\textwidth}
\lstinputlisting[firstline=1, lastline=13]{equivproof.tex}
\end{minipage}
\hfill
\begin{minipage}{0.48\textwidth}
\lstinputlisting[firstline=15, lastline=28]{equivproof.tex}
\end{minipage}}
\vspace{-0.3cm}
\caption{Two functions between \lstinline{Old.list} and \lstinline{New.list} (top) that form an equivalence (bottom).}
\label{fig:equivalence}
\end{figure*}

Consider a simple example of using \toolname: repairing proofs after swapping the two constructors of the \lstinline{list} datatype (Figure~\ref{fig:listswap}).
This is inspired by a similar change from a user study of proof engineers (Section~\ref{sec:search}).
Even such a simple change can cause trouble, as in this proof from the Coq standard library (comments ours for clarity):\footnote{We use induction instead of pattern matching.}

\begin{lstlisting}
Lemma rev_app_distr {A} :(@\vspace{-0.04cm}@)
  $\forall$ (x y : list A), rev (x ++ y) = rev y ++ rev x.(@\vspace{-0.04cm}@)
Proof. (@\color{gray}{(* by induction over x and y *)}@)(@\vspace{-0.04cm}@)
  induction x as [| a l IHl].(@\vspace{-0.04cm}@)
  (@\color{gray}{(* x nil: *)}@) induction y as [| a l IHl].(@\vspace{-0.04cm}@)
  (@\color{gray}{(* y nil: *)}@) simpl. auto.(@\vspace{-0.04cm}@)
  (@\color{gray}{(* y cons *)}@) simpl. rewrite app_nil_r; auto.(@\vspace{-0.04cm}@)
  (@\color{gray}{(* both cons: *)}@) intro y. simpl.(@\vspace{-0.04cm}@)
  rewrite (IHl y). rewrite app_assoc; trivial.(@\vspace{-0.04cm}@)
Qed.(@\vspace{-0.05cm}@)
\end{lstlisting}
This lemma says that appending (\lstinline{++}) two lists and reversing (\lstinline{rev}) the result behaves the same as appending
the reverse of the second list onto the reverse of the first list.
The proof script works by induction over the input lists \lstinline{x} and \lstinline{y}:
In the base case for both \lstinline{x} and \lstinline{y}, the result holds by reflexivity.
In the base case for \lstinline{x} and the inductive case for \lstinline{y}, the result follows from the existing lemma \lstinline{app_nil_r}.
Finally, in the inductive case for both \lstinline{x} and \lstinline{y}, the result follows by the inductive hypothesis
and the existing lemma \lstinline{app_assoc}.

When we change the \lstinline{list} type, this proof no longer works.
%This theorem statement \lstinline{rev_app_distr} defined over the old version of \lstinline{list} is our \textit{old specification}.
%When we change the \lstinline{list} type, we get the \textit{new specification}.
%But the \textit{old proof} or tactic script no longer works with this new specification.
To repair this proof with \toolname, we run this command:

\iffalse
\begin{figure}
\begin{minipage}{0.46\textwidth}
   \lstinputlisting[firstline=1, lastline=3]{listswap.tex}
\end{minipage}
\hfill
\begin{minipage}{0.46\textwidth}
   \lstinputlisting[firstline=5, lastline=7]{listswap.tex}
\end{minipage}
\vspace{-0.4cm}
\caption{The updated \lstinline{list} (bottom) is the old \lstinline{list} (top) with its two constructors swapped (\codediff{orange}).}
\label{fig:listswap}
\end{figure}
\fi

\begin{lstlisting}
Repair Old.list New.list in rev_app_distr.(@\vspace{-0.05cm}@)
\end{lstlisting}
assuming the old and new list types from Figure~\ref{fig:listswap} are in modules \lstinline{Old} and \lstinline{New}.
This suggests a proof script that succeeds (in $\codeauto{light blue}$ to denote \toolname produces it automatically):

\begin{lstlisting}[backgroundcolor=\color{cyan!30}]
Proof. (@\color{gray}{(* by induction over x and y *)}@)(@\vspace{-0.04cm}@)
  intros x. induction x as [a l IHl| ]; intro y0.(@\vspace{-0.04cm}@)
  - (@\color{gray}{(* both cons: *)}@) simpl. rewrite IHl. simpl.(@\vspace{-0.04cm}@)
    rewrite app_assoc. auto.(@\vspace{-0.04cm}@)
  - (@\color{gray}{(* x nil: *)}@) induction y0 as [a l H| ].(@\vspace{-0.04cm}@)
    + (@\color{gray}{(* y cons: *)}@) simpl. rewrite app_nil_r. auto.(@\vspace{-0.04cm}@)
    + (@\color{gray}{(* y nil: *)}@) auto.(@\vspace{-0.04cm}@)
Qed.(@\vspace{-0.05cm}@)
\end{lstlisting}
where the dependencies (\lstinline{rev}, \lstinline{++}, \lstinline{app_assoc}, and \lstinline{app_nil_r}) have
also been updated automatically~\href{https://github.com/uwplse/pumpkin-pi/blob/silent/plugin/coq/Swap.v}{\circled{1}}. % Swap.v
If we would like, we can manually modify this to something that more closely matches the style of the original proof sript:

\begin{lstlisting}
Proof. (@\color{gray}{(* by induction over x and y *)}@)(@\vspace{-0.04cm}@)
  induction x as [a l IHl|].(@\vspace{-0.04cm}@)
  (@\color{gray}{(* both cons: *)}@) intro y. simpl.(@\vspace{-0.04cm}@)
  rewrite (IHl y). rewrite app_assoc; trivial.(@\vspace{-0.04cm}@)
  (@\color{gray}{(* x nil: *)}@) induction y as [a l IHl|].(@\vspace{-0.04cm}@)
  (@\color{gray}{(* y cons: *)}@) simpl. rewrite app_nil_r; auto.(@\vspace{-0.04cm}@)
  (@\color{gray}{(* y nil: *)}@) simpl. auto.(@\vspace{-0.04cm}@)
Qed.(@\vspace{-0.04cm}@)
\end{lstlisting}
We can even repair the entire list module from the Coq standard library all at once by running the \lstinline{Repair module}
command~\href{https://github.com/uwplse/pumpkin-pi/blob/silent/plugin/coq/Swap.v}{\circled{1}}. % Swap.v
When we are done, we can get rid of \lstinline{Old.list}. % entirely.

The key to success is taking advantage of Coq's structured proof term language:
Coq compiles every proof script to a proof term in a rich functional programming language called 
%Gallina that is based on the calculus of inductive  constructions---\toolname repairs that term.
Gallina---\toolname repairs that term.
\toolname then decompiles the repaired proof term (with optional hints from the original proof script) back 
to a proof script that the proof engineer can maintain.
%Here, \toolname transforms the proof term Coq compiles \lstinline{rev_app_distr} to,
%and then decompiles that transformed proof term to the proof script in light blue.

In contrast, updating the poorly structured proof script directly would not be straightforward.
Even for the simple proof script above, grouping tactics by line, there are $6! = 720$ permutations of this proof script.
It is not clear which lines to swap since these tactics do not have a semantics beyond the searches their evaluation performs.
Furthermore, just swapping lines is not enough: even for such a simple change, we must also swap
arguments, so \lstinline{induction x as [| a l IHl]} becomes \lstinline{induction x as [a l IHl|]}.
%Handling even swapping constructors this way would require a search procedure that would not generalize to other changes.
\citet{robert2018} describes the challenges of repairing tactics in detail.
\toolname's approach circumvents this challenge.

%\toolname's approach circumvents this challenge. % by transforming proof terms, then decompiling the transformed proof term to a tactic script.
%By decompiling the transformed proof term, \toolname is able to suggest a tactic script in the end.
%As later sections show, this approach is much more general than just permuting constructors.



