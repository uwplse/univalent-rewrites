\section{The Transformation}
\label{sec:key2}

\begin{figure*}
\begin{minipage}{0.48\textwidth}
\begin{lstlisting}
DepConstr(0, list T) : list T := Constr((@\codediff{0}@), list T).(@\vspace{-0.04cm}@)
DepConstr(1, list T) t l : list T :=(@\vspace{-0.04cm}@)
  Constr ((@\codediff{1}@), list T) t l.(@\vspace{-0.04cm}@)
(@\vspace{-0.14cm}@)
DepElim(l, P) { p$_{\mathtt{nil}}$, p$_{\mathtt{cons}}$ } : P l :=(@\vspace{-0.04cm}@)
  Elim(l, P) { (@\codediff{p$_{\mathtt{nil}}$}@), (@\codediff{p$_{\mathtt{cons}}$}@) }.(@\vspace{-0.04cm}@)
\end{lstlisting}
\end{minipage}
\hfill
\begin{minipage}{0.48\textwidth}
\begin{lstlisting}
DepConstr(0, list T) : list T := Constr((@\codediff{1}@), list T).(@\vspace{-0.04cm}@)
DepConstr(1, list T) t l : list T :=(@\vspace{-0.04cm}@)
  Constr((@\codediff{0}@), list T) t l.(@\vspace{-0.04cm}@)
(@\vspace{-0.14cm}@)
DepElim(l, P) { p$_{\mathtt{nil}}$, p$_{\mathtt{cons}}$ } : P l :=(@\vspace{-0.04cm}@)
  Elim(l, P) { (@\codediff{p$_{\mathtt{cons}}$}@), (@\codediff{p$_{\mathtt{nil}}$}@) }.(@\vspace{-0.04cm}@)
\end{lstlisting}
\end{minipage}
\vspace{-0.3cm}
\caption{The dependent constructors and eliminators for old (left) and new (right) \lstinline{list}, with the difference in \codediff{orange}.}
\vspace{-0.1cm}
\label{fig:listconfig}
\end{figure*}

At the heart of \toolname is a configurable proof term transformation for transporting
proofs across equivalences~\href{https://github.com/uwplse/pumpkin-pi/blob/v2.0.0/plugin/src/automation/lift/lift.ml}{\circled{4}}. % lift.ml
It is a generalization of the transformation from an earlier version of \toolname called
\textsc{Devoid}~\cite{Ringer2019}, which solved this problem a particular class of equivalences.
%\toolname moves the reasoning specific to that class of equivalences into the configuration. 

%\begin{quote}
%\textbf{Insight 2}:
%A configurable proof term transformation can be used to build such a proof repair tool,
%and the result can handle many different kinds of changes.
%\end{quote}

The transformation takes as input a deconstructed equivalence that we call a \textit{configuration}.
This section introduces the configuration (Section~\ref{sec:configurable}),
defines the transformation that builds on that (Section~\ref{sec:generic}),
then specifies correctness criteria for the configuration (Section~\ref{sec:art}).
Section~\ref{sec:implementation} describes the additional work needed to implement this transformation.

\mysubsubsec{Conventions}
All terms that we introduce in this section are in the Calculus of Inductive Constructions (CIC$_{\omega}$), the type theory
that Coq's proof term language Gallina implements.
CIC$_{\omega}$ is based on the Calculus of Constructions (CoC), a variant of the lambda calculus with polymorphism (types that dependent on types) and dependent types (types that depend on terms)~\cite{coquand:inria-00076024}. CIC$_{\omega}$ extends CoC with
inductive types~\cite{inductive}.
Inductive types are defined solely by their constructors (like \lstinline{nil} and \lstinline{cons} for \lstinline{list}) and eliminators (like the induction principle for \lstinline{list}); this section assumes that these eliminators are primitive.

The syntax for CIC$_{\omega}$ with primitive eliminators is in Figure~\ref{fig:syntax}.
The typing rules are standard.
We assume inductive types $\Sigma$ with constructor $\exists$ and projections $\pi_l$ and $\pi_r$,
and an equality type \lstinline{=} with constructor \lstinline{eq_refl}.
We use $\vec{t}$ and $\{t_1, \ldots, t_n\}$ to denote lists of terms.

\subsection{The Configuration}
\label{sec:configurable}

The configuration is the key to building a proof term transformation that implements transport in a way that is suitable for repair.
Each configuration corresponds to an equivalence \A $\simeq$ \B.
It deconstructs the equivalence into things that talk about \A, and things that talk about \B.
It does so in a way that hides details
specific to the equivalence, like the order or number of arguments to an induction principle or type.

At a high level, the configuration helps the transformation achieve two goals: preserve equality up to transport across the equivalence 
between \A and \B, and produce well-typed terms.
This configuration is a pair of pairs:

\begin{lstlisting}
((DepConstr, DepElim), (Eta, Iota))(@\vspace{-0.05cm}@)
\end{lstlisting}
each of which corresponds to one of the two goals:
\lstinline{DepConstr} and \lstinline{DepElim} define how to transform constructors and eliminators, thereby preserving the equivalence, and 
\lstinline{Eta} and \lstinline{Iota} define how to transform $\eta$-expansion and $\iota$-reduction of constructors and eliminators, thereby producing well-typed terms.
Each of these is defined in CIC$_{\omega}$ for each equivalence.
%\textbf{Configure} passes this configuration to \textbf{Transform}.

%Section~\ref{sec:art} describes how the four parts of this configuration must relate to one another in order for the proof
%term transformation to work correctly, and proves that every equivalence induces a configuration.

\mysubsubsec{Preserving the Equivalence}
To preserve the equivalence, the configuration ports terms over \A to terms over \B by viewing each
term of type \B as if it is an \A.
This way, the rest of the transformation can replace values of \A with values of \B, and
inductive proofs about \A with inductive proofs about \B, %, then recursively transform
%subterms 
all without changing the order or number of arguments.

The two configuration parts responsible for this are \lstinline{DepConstr}
and \lstinline{DepElim} (\textit{dependent constructors} and \textit{eliminators}).
These describe how to construct and eliminate \A and \B, wrapping the types with a common inductive structure.
The transformation requires the same number of dependent constructors and cases in dependent eliminators for \A and \B,
even if \A and \B are types with different numbers of constructors
(\A and \B need not even be inductive; see Sections~\ref{sec:art} and~\ref{sec:search}).

For the \lstinline{list} change from Section~\ref{sec:overview},
the configuration that \toolname discovers uses the dependent constructors
and eliminators in Figure~\ref{fig:listconfig}. The dependent constructors for \lstinline{Old.list}
are the normal constructors with the order unchanged,
while the dependent constructors for \lstinline{New.list} swap constructors
back to the original order.
Similarly, the dependent eliminator for \lstinline{Old.list} is the normal eliminator for \lstinline{Old.list},
while the dependent eliminator for \lstinline{New.list} swaps cases.

As the name hints, these constructors and eliminators can be dependent.
Consider the type of vectors of some length:

\begin{lstlisting}
packed_vect T := $\Sigma$(n : nat).vector T n.(@\vspace{-0.05cm}@)
\end{lstlisting}
\toolname can port proofs across the equivalence between this type and \lstinline{list T}~\href{https://github.com/uwplse/pumpkin-pi/blob/v2.0.0/plugin/coq/examples/Example.v}{\circled{3}}. % Example.v
The dependent constructors \toolname discovers pack the index into an existential, like:

\begin{lstlisting}
DepConstr(0, packed_vect) : packed_vect T :=(@\vspace{-0.04cm}@)
  $\exists$ (Constr(0, nat)) (Constr(0, vector T)).(@\vspace{-0.05cm}@)
\end{lstlisting}
and the eliminator it discovers eliminates the projections:

\begin{lstlisting}
DepElim(s, P) { f$_0$ f$_1$ } : P ($\exists$ ($\pi_l$ s) ($\pi_r$ s)) :=(@\vspace{-0.04cm}@)
  Elim($\pi_r$ s, $\lambda$(n : nat)(v : vector T n).P ($\exists$ n v)) {(@\vspace{-0.04cm}@)
    f$_0$,(@\vspace{-0.04cm}@)
    ($\lambda$(t : T)(n : nat)(v : vector T n).f$_1$ t ($\exists$ n v))(@\vspace{-0.04cm}@)
  }.(@\vspace{-0.05cm}@) 
\end{lstlisting}

In both these examples, the interesting work moves into the configuration:
the configuration for the first swaps constructors and cases,
and the configuration for the second maps constructors and cases over \lstinline{list} to constructors and cases over \lstinline{packed_vect}. %packs constructors and eliminates projections.
That way, the transformation need not add, drop, or reorder arguments.
%In essence, all of the difficult work moves into the configuration.
Furthermore, both examples use automatic configuration, so \toolname's \textbf{Configure} component 
discovers \lstinline{DepConstr} and \lstinline{DepElim} from just the types \A and \B, taking care of even the difficult work.

\mysubsubsec{Producing Well-Typed Terms}
The other configuration parts \lstinline{Eta} and \lstinline{Iota} deal with producing well-typed terms,
in particular by transporting equalities.
CIC$_{\omega}$ distinguishes between two important kinds of equality: those that hold by reduction (\textit{definitional} equality), and those that hold by proof (\textit{propositional} equality).
That is, two terms \lstinline{t} and \lstinline{t'} of type \lstinline{T} are definitionally equal if they reduce to the same normal form,
and propositionally equal if there is a proof that \lstinline{t = t'} using the inductive
equality type \lstinline{=} at type \lstinline{T}. Definitionally equal terms are necessarily propositionally equal, but 
the converse is not in general true.

When a datatype changes, sometimes, definitional equalities defined over the old version of that type must become propositional.
A naive proof term transformation may fail to generate well-typed terms if it does not account for this.
Otherwise, if the transformation transforms a term \lstinline{t : T} to some \lstinline{t' : T'}, it does not necessarily
transform \lstinline{T} to \lstinline{T'}~\cite{tabareau2019marriage}.

\lstinline{Eta} and \lstinline{Iota} describe how to transport equalities.
More formally, they define $\eta$-expansion and $\iota$-reduction of \A and \B,
which may be propositional rather than definitional,
and so must be explicit in the transformation.
$\eta$-expansion describes how to expand a term to apply a constructor to an eliminator in a way that preserves propositional equality,
and is important for defining dependent eliminators~\cite{nlab:eta-conversion}.
$\iota$-reduction ($\beta$-reduction for inductive types) describes how to reduce an elimination of a constructor~\cite{nlab:beta-reduction}.

The configuration for the change from \lstinline{list} to \lstinline{packed_vect} has propositional \lstinline{Eta}.
It uses $\eta$-expansion for $\Sigma$:

\begin{lstlisting}
Eta(packed_vect) := $\lambda$(s:packed_vect).$\exists$ ($\pi_l$ s) ($\pi_r$ s).(@\vspace{-0.05cm}@)
\end{lstlisting}
which is propositional and not definitional in Coq.
Thanks to this, we can forego the assumption that our language has primitive projections (definitional $\eta$ for $\Sigma$).

\begin{figure}
\begin{minipage}{0.44\columnwidth}
   \lstinputlisting[firstline=1, lastline=8]{nattobin.tex}
\end{minipage}
\hfill
\begin{minipage}{0.54\columnwidth}
   \lstinputlisting[firstline=10, lastline=17]{nattobin.tex}
\end{minipage}
\vspace{-0.4cm}
\caption{A unary natural number \lstinline{nat} (left) is either zero (\lstinline{0}) or the successor of some other natural number (\lstinline{S}).
A binary natural number \lstinline{N} (right) is either zero (\lstinline{N0}) or a positive binary number (\lstinline{Npos}), where a positive binary number is either 1 (\lstinline{xH}), or the result of shifting left and adding 1 (\lstinline{xI}) or
0 (\lstinline{xO}). Unary and binary natural numbers are equivalent, but have different inductive structures.
Consequentially, definitional equalities over \lstinline{nat} may become propositional over \lstinline{N}.}
\vspace{-0.2cm}
\label{fig:nattobin}
\end{figure}

Each \lstinline{Iota}---one per constructor---describes and proves the $\iota$-reduction behavior
of \lstinline{DepElim} on the corresponding case.
This is needed, for example, to port proofs about unary numbers \lstinline{nat} to
proofs about binary numbers \lstinline{N} (Figure~\ref{fig:nattobin}).
While we can define \lstinline{DepConstr} and \lstinline{DepElim} to induce an equivalence
between them~\href{https://github.com/uwplse/pumpkin-pi/blob/v2.0.0/plugin/coq/nonorn.v}{\circled{5}}, % nonorn.v
we run into trouble reasoning about applications of \lstinline{DepElim},
since proofs about \lstinline{nat} that hold by reflexivity do not necessarily hold by reflexivity over \lstinline{N}. 
For example, in Coq, while \lstinline{S (n + m)  = S n + m} holds by reflexivity over \lstinline{nat},
when we define \lstinline{+} with \lstinline{DepElim} over \lstinline{N},
the corresponding theorem over \lstinline{N} does not hold by reflexivity.

To transform proofs about \lstinline{nat} to proofs about \lstinline{N}, we must transform \textit{definitional} $\iota$-reduction over \lstinline{nat} to \textit{propositional} $\iota$-reduction over \lstinline{N}.
For our choice of \lstinline{DepConstr} and \lstinline{DepElim},
$\iota$-reduction is definitional over \lstinline{nat}, since a proof of:

\begin{lstlisting}
$\forall$ P p$_\texttt{0}$ p$_\texttt{S}$ n,(@\vspace{-0.04cm}@)
  DepElim((@\codediff{DepConstr(1, nat) n}@), P) { p$_\texttt{0}$, p$_\texttt{S}$ } =(@\vspace{-0.04cm}@)
  (@\codediff{p$_\texttt{S}$}@) n (DepElim(n, P) { p$_\texttt{0}$, p$_\texttt{S}$ }).(@\vspace{-0.05cm}@)
\end{lstlisting}
holds by reflexivity.
\lstinline{Iota} for \lstinline{nat} in the \lstinline{S} case is a rewrite by that proof by reflexivity~\href{https://github.com/uwplse/pumpkin-pi/blob/v2.0.0/plugin/coq/nonorn.v}{\circled{5}},
with type:

\begin{lstlisting}
$\forall$ P p$_\texttt{0}$ p$_\texttt{S}$ n (Q: P (DepConstr(1, nat) n) $\rightarrow$ s),(@\vspace{-0.04cm}@)
  Iota(1, nat, Q) :(@\vspace{-0.04cm}@)
    Q ((@\codediff{p$_\texttt{S}$}@) n (DepElim(n, P) { p$_\texttt{0}$, p$_\texttt{S}$ })) $\rightarrow$(@\vspace{-0.04cm}@)
    Q (DepElim((@\codediff{DepConstr(1, nat) n}@), P) { p$_\texttt{0}$, p$_\texttt{S}$ }).(@\vspace{-0.05cm}@)
\end{lstlisting}

In contrast, $\iota$ for \lstinline{N} is propositional, since the corresponding 
theorem over \lstinline{N}:

\begin{lstlisting}
$\forall$ P p$_\texttt{0}$ p$_\texttt{S}$ n,(@\vspace{-0.04cm}@)
  DepElim((@\codediff{DepConstr(1, N) n}@), P) { p$_\texttt{0}$, p$_\texttt{S}$ } =(@\vspace{-0.04cm}@)
  (@\codediff{p$_\texttt{S}$}@) n (DepElim(n, P) { p$_\texttt{0}$, p$_\texttt{S}$ }).(@\vspace{-0.05cm}@)
\end{lstlisting}
no longer holds by reflexivity.
\lstinline{Iota} for \lstinline{N} is a rewrite by the propositional equality that proves this theorem~\href{https://github.com/uwplse/pumpkin-pi/blob/v2.0.0/plugin/coq/nonorn.v}{\circled{5}},
with type:

\begin{lstlisting}
$\forall$ P p$_\texttt{0}$ p$_\texttt{S}$ n (Q: P (DepConstr(1, N) n) $\rightarrow$ s),(@\vspace{-0.04cm}@)
  Iota(1, N, Q) :(@\vspace{-0.04cm}@)
    Q ((@\codediff{p$_\texttt{S}$}@) n (DepElim(n, P) { p$_\texttt{0}$, p$_\texttt{S}$ })) $\rightarrow$(@\vspace{-0.04cm}@)
    Q (DepElim((@\codediff{DepConstr(1, N) n}@), P) { p$_\texttt{0}$, p$_\texttt{S}$ }).(@\vspace{-0.05cm}@)
\end{lstlisting}

By replacing \lstinline{Iota} over \lstinline{nat} with \lstinline{Iota} over \lstinline{N},
the transformation replaces rewrites by reflexivity over \lstinline{nat} to rewrites by propositional equalities over \lstinline{N}.
That way, \lstinline{DepElim} behaves the same over \lstinline{nat} and \lstinline{N}.

Taken together over both \A and \B, \lstinline{Iota} describes how the inductive structures of \A and \B differ.
The transformation requires that \lstinline{DepElim} over \A and over \B have the same structure
as each other, so if \A and \B \textit{themselves} have the same 
inductive structure (if they are \textit{ornaments}~\cite{mcbride}),
then if $\iota$ is definitional for \A, it will be possible to choose
\lstinline{DepElim} with definitional $\iota$ for \B.
Otherwise, if \A and \B (like \lstinline{nat} and \lstinline{N}) have different inductive structures,
then definitional $\iota$ over one would become propositional $\iota$ over the other.
%For the case of \lstinline{nat} and \lstinline{N},
%the need for propositional $\iota$ was noted as far back as \citet{magaud2000changing}.
%\lstinline{Iota} in the configuration encodes this more generally.


\subsection{The Proof Term Transformation}
\label{sec:generic}

\begin{figure*}
\begin{mathpar}
\mprset{flushleft}
\small
\hfill\fbox{$\Gamma$ $\vdash$ $t$ $\Uparrow$ $t'$}\vspace{-0.3cm}\\

\inferrule[Dep-Elim]
  { \Gamma \vdash a \Uparrow b \\ \Gamma \vdash p_{a} \Uparrow p_b \\ \Gamma \vdash \vec{f_{a}}\phantom{l} \Uparrow \vec{f_{b}} }
  { \Gamma \vdash \mathrm{DepElim}(a,\ p_{a}) \vec{f_{a}} \Uparrow \mathrm{DepElim}(b,\ p_b) \vec{f_{b}} }

\inferrule[Dep-Constr]
{ \Gamma \vdash \vec{t}_{a} \Uparrow \vec{t}_{b} } %\\ TODO must we explicitly lift A to B if we want to handle parameters/indices?
{ \Gamma \vdash \mathrm{DepConstr}(j,\ A)\ \vec{t}_{a} \Uparrow \mathrm{DepConstr}(j,\ B)\ \vec{t}_{b}  }

\inferrule[Eta]
  { \\ }
  { \Gamma \vdash \mathrm{Eta}(A) \Uparrow \mathrm{Eta}(B) }

\inferrule[Iota]
  { \Gamma \vdash q_A \Uparrow q_B \\ \Gamma \vdash \vec{t_A} \Uparrow \vec{t_B} }
  { \Gamma \vdash \mathrm{Iota}(j,\ A,\ q_A)\ \vec{t_A} \Uparrow \mathrm{Iota}(j,\ B,\ q_B)\ \vec{t_B} }

\inferrule[Equivalence]
  { \\ }
  { \Gamma \vdash A\ \Uparrow B }

\inferrule[Constr]
{ \Gamma \vdash T \Uparrow T' \\ \Gamma \vdash \vec{t} \Uparrow \vec{t'} }
{ \Gamma \vdash \mathrm{Constr}(j,\ T)\ \vec{t} \Uparrow \mathrm{Constr}(j,\ T')\ \vec{t'} }

\inferrule[Ind]
  { \Gamma \vdash T \Uparrow T' \\ \Gamma \vdash \vec{C} \Uparrow \vec{C'}  }
  { \Gamma \vdash \mathrm{Ind} (\mathit{Ty} : T) \vec{C} \Uparrow \mathrm{Ind} (\mathit{Ty} : T') \vec{C'} }

%% Application
\inferrule[App]
 { \Gamma \vdash f \Uparrow f' \\ \Gamma \vdash t \Uparrow t'}
 { \Gamma \vdash f t \Uparrow f' t' }

\inferrule[Elim] % TODO wait why do we have c here when it clearly refers to the term we eliminate over? um
  { \Gamma \vdash c \Uparrow c' \\ \Gamma \vdash Q \Uparrow Q' \\ \Gamma \vdash \vec{f} \Uparrow \vec{f'}}
  { \Gamma \vdash \mathrm{Elim}(c, Q) \vec{f} \Uparrow \mathrm{Elim}(c', Q') \vec{f'}  }

% Lamda
\inferrule[Lam]
  { \Gamma \vdash t \Uparrow t' \\ \Gamma \vdash T \Uparrow T' \\ \Gamma,\ t : T \vdash b \Uparrow b' }
  {\Gamma \vdash \lambda (t : T).b \Uparrow \lambda (t' : T').b'}

% Product
\inferrule[Prod]
  { \Gamma \vdash t \Uparrow t' \\ \Gamma \vdash T \Uparrow T' \\ \Gamma,\ t : T \vdash b \Uparrow b' }
  {\Gamma \vdash \Pi (t : T).b \Uparrow \Pi (t' : T').b'}

\inferrule[Var]
  { v \in \mathrm{Vars} }
  {\Gamma \vdash v \Uparrow v}

%\inferrule[Sort]
%  { \\ }
%  {\Gamma \vdash s \Uparrow s}
\end{mathpar}
\vspace{-0.3cm}
\caption{Transformation for transporting terms across $A \simeq B$ with configuration \lstinline{((DepConstr, DepElim), (Eta, Iota))}.}
\label{fig:final}
\end{figure*}

\begin{figure*}
\begin{minipage}{0.49\textwidth}
\begin{lstlisting}
(* 1: original term *)(@\vspace{-0.04cm}@)
$\lambda$ (T : Type) (l m : Old.list T) .(@\vspace{-0.04cm}@)
 Elim(l, $\lambda$(l: Old.list T).Old.list T $\rightarrow$ Old.list T)) {(@\vspace{-0.04cm}@)
   ($\lambda$ m . m),(@\vspace{-0.04cm}@)
   ($\lambda$ t _ IHl m . Constr(1, Old.list T) t (IHl m))(@\vspace{-0.04cm}@)
 } m.(@\vspace{-0.04cm}@)
(@\vspace{-0.18cm}@)
(* 2: after unifying (@\texttt{with}@) configuration *)(@\vspace{-0.04cm}@)
$\lambda$ (T : Type) (l m : (@\codediff{A}@)) .(@\vspace{-0.04cm}@)
 (@\codediff{DepElim}@)(l, $\lambda$(l: (@\codediff{A}@)).(@\codediff{A}@) $\rightarrow$ (@\codediff{A}@))) {(@\vspace{-0.04cm}@)
   ($\lambda$ m . m)(@\vspace{-0.04cm}@)
   ($\lambda$ t _ IHl m . (@\codediff{DepConstr}@)(1, (@\codediff{A}@)) t (IHl m))(@\vspace{-0.04cm}@)
 } m.(@\vspace{-0.04cm}@)
\end{lstlisting}
\end{minipage}
\hfill
\begin{minipage}{0.49\textwidth}
\begin{lstlisting}
(* 4: reduced to final term *)(@\vspace{-0.04cm}@)
$\lambda$ (T : Type) (l m : New.list T) .(@\vspace{-0.04cm}@)
 Elim(l, $\lambda$(l: New.list T).New.list T $\rightarrow$ New.list T)) {(@\vspace{-0.04cm}@)
   ($\lambda$ t _ IHl m . Constr(0, New.list T) t (IHl m)),(@\vspace{-0.04cm}@)
   ($\lambda$ m . m)(@\vspace{-0.04cm}@)
 } m.(@\vspace{-0.04cm}@)
(@\vspace{-0.18cm}@)
(* 3: after transforming *)(@\vspace{-0.04cm}@)
$\lambda$ (T : Type) (l m : (@\codediff{B}@)) .(@\vspace{-0.04cm}@)
 (@\codediff{DepElim}@)(l, $\lambda$(l: (@\codediff{B}@)).(@\codediff{B}@) $\rightarrow$ (@\codediff{B}@))) {(@\vspace{-0.04cm}@)
   ($\lambda$ m . m)(@\vspace{-0.04cm}@)
   ($\lambda$ t _ IHl m . (@\codediff{DepConstr}@)(1, (@\codediff{B}@)) t (IHl m))(@\vspace{-0.04cm}@)
 } m.(@\vspace{-0.04cm}@)
\end{lstlisting}
\end{minipage}
\vspace{-0.3cm}
\caption{Swapping cases of the append function, counterclockwise, the input term: 1) unmodified, 2) unified with the configuration, 3) ported to the updated type, and 4) reduced to the output.}
\label{fig:appswap1}
\end{figure*}

Figure~\ref{fig:final} shows the proof term transformation $\Gamma \vdash t \Uparrow t'$ that forms the core of \toolname.
%Like the transformation from \textsc{Devoid},
The transformation is parameterized over equivalent types \A and \B (\textsc{Equivalence})
as well as the configuration. %terms, which appear in the transformation explicitly.
It assumes $\eta$-expanded functions.
It implicitly constructs an updated context $\Gamma'$ in which to interpret $t'$, but this is not needed for computation.

The proof term transformation is (perhaps deceptively) simple by design:
it moves the bulk of the work into the configuration,
and represents the configuration explicitly.
Of course, typical proof terms in Coq do not apply these configuration
terms explicitly.
\toolname does some additional work using \textit{unification heuristics} to get real proof terms into this format before running the transformation.
It then runs the proof term transformation, which transports proofs across the equivalence that corresponds to the configuration.

\mysubsubsec{Unification Heuristics}
The transformation does not fully describe the search procedure for transforming terms that \toolname implements.
Before running the transformation, \toolname \textit{unifies} subterms with particular \A (fixing parameters and indices),
and with applications of configuration terms over \A. 
The transformation then transforms configuration terms over \A
to configuration terms over \B.
Reducing the result produces the output term defined over \B.

Figure~\ref{fig:appswap1} shows this with the list append function \lstinline{++} from Section~\ref{sec:overview}.
To update \lstinline{++} (top left), \toolname unifies \lstinline{Old.list T} with \A, and \lstinline{Constr} and \lstinline{Elim}
with \lstinline{DepConstr} and \lstinline{DepElim} (bottom left).
After unification, the transformation recursively substitutes \B
for \A, which moves \lstinline{DepConstr} and \lstinline{DepElim}
to construct and eliminate over the updated type (bottom right).
This reduces to a term with swapped constructors and cases over \lstinline{New.list T} (top right).

In this case, unification is straightforward. % since \lstinline{DepConstr} and \lstinline{DepElim} correspond to
%\lstinline{Constr} and \lstinline{Elim} directly.
This can be more challenging when configuration terms are dependent.
This is especially pronounced with definitional \lstinline{Eta} and \lstinline{Iota},
which typically are implicit (reduced) in real code.
%This problem is exactly why \citet{tabareau2019marriage} speculated that converting definitional to propositional equalities
%like we do with \lstinline{Iota} may, in general, be intractable.
To handle this, \toolname implements custom \textit{unification heuristics} for each search procedure
that unify subterms with applications of configuration terms, and that instantiate parameters and dependent indices in those subterms~\href{https://github.com/uwplse/pumpkin-pi/blob/v2.0.0/plugin/src/automation/lift/liftconfig.ml}{\circled{6}}. % liftconfig.ml
The transformation in turn assumes that all existing parameters and indices are determined and instantiated
by the time it runs.

\toolname falls back to Coq's unification for manual configuration and when these custom heuristics fail.
When even Coq's unification is not enough, \toolname relies on proof engineers to provide hints
in the form of annotations~\href{https://github.com/uwplse/pumpkin-pi/blob/v2.0.0/plugin/coq/nonorn.v}{\circled{5}}.

\begin{figure*}
\begin{minipage}{0.43\textwidth}
\begin{lstlisting}
section: $\forall$ (a : A), g (f a) = a.(@\vspace{-0.04cm}@)
retraction: $\forall$ (b : B), f (g b) = b.(@\vspace{-0.04cm}@)
(@\vspace{-0.14cm}@)
constr_ok: $\forall$ $j$ $\vec{x_A}$ $\vec{x_B}$, $\vec{x_A}$ $\equiv_{A \simeq B}$ $\vec{x_B}$ $\rightarrow$(@\vspace{-0.04cm}@)
  DepConstr($j$, A) $\vec{x_A}$ $\equiv_{A \simeq B}$ DepConstr(j, B) $\vec{x_B}$.(@\vspace{-0.04cm}@)
(@\vspace{-0.14cm}@)
elim_ok: $\forall$ a b P$_A$ P$_B$ $\vec{f_A}$ $\vec{f_B}$,(@\vspace{-0.04cm}@)
  a $\equiv_{A \simeq B}$ b $\rightarrow$(@\vspace{-0.04cm}@)
  P$_A$ $\equiv_{(A \rightarrow s) \simeq (B \rightarrow s)}$ P$_B$ $\rightarrow$(@\vspace{-0.04cm}@)
  $\forall$ $j$, $\vec{f_A}$[j] $\equiv_{\xi (A, P_A, j) \simeq \xi (B, P_B, j)}$ $\vec{f_B}$[j]$\rightarrow$(@\vspace{-0.04cm}@)
  DepElim(a, P$_A$) $\vec{f_A}$ $\equiv_{(P a) \simeq (P b)}$ DepElim(b, P$_B$) $\vec{f_A}$.(@\vspace{-0.04cm}@)
\end{lstlisting}
\end{minipage}
\hfill
\begin{minipage}{0.56\textwidth}
\begin{lstlisting}
elim_eta(A): $\forall$ a P $\vec{f}$, DepElim(a, P) $\vec{f}$ : P (Eta(A) a).(@\vspace{-0.04cm}@)
eta_ok(A): $\forall$ (a : A), Eta(A) a = a.(@\vspace{-0.04cm}@)
(@\vspace{-0.14cm}@)
(@\phantom{constr_ok: $\forall$ $j$ $\vec{x_A}$ $\vec{x_B}$,}@)(@\vspace{-0.04cm}@)
(@\phantom{  DepConstr($j$, A) $\vec{x_A}$ $\equiv_{A \simeq B}$ DepConstr(j, B) $\vec{x_B}$.}@)(@\vspace{-0.04cm}@)
(@\vspace{-0.14cm}@)
iota_ok(A): $\forall$ $j$ P $\vec{f}$ $\vec{x}$ (Q: P(Eta(A) (DepConstr($j$, A) $\vec{x}$)) $\rightarrow$ s),(@\vspace{-0.04cm}@)
  Iota(A, j, Q) : (@\vspace{-0.04cm}@)
    Q (DepElim(DepConstr(j, A) $\vec{x}$, P) $\vec{f}$) $\rightarrow$ (@\vspace{-0.04cm}@)
    Q (rew $\leftarrow$ eta_ok(A) (DepConstr(j, A) $\vec{x}$) in(@\vspace{-0.04cm}@)
      ($\vec{f}$[j]$\ldots$(DepElim(IH$_0$, P) $\vec{f}$)$\ldots$(DepElim(IH$_n$, P) $\vec{f}$)$\ldots$)).(@\vspace{-0.04cm}@)
\end{lstlisting}
\end{minipage}
% Q (eq_rect (dep_constr_A_0 b) (fun H : A => P H) (f0 b) (eta_A (dep_constr_A_0 b)) (eq_sym (eta_OK_A (dep_constr_A_0 b)))).
\iffalse
\begin{minipage}{0.44\textwidth}
\smallmath{$f := \lambda(a : A).\mathrm{DepElim}(a, \lambda(a : A).B) { \lambda \ldots \mathrm{DepConstr}(0, B) \ldots, \ldots }.$}
%[\
%g := \lambda(b : B).\mathrm{DepElim}(b, \lambda(b : B).A)\\
%  { \lambda \ldots \mathrm{DepConstr}(0, A) \ldots, \ldots }.\\
%section : \forall (a : A), g (f a) = a.\\
%retraction : \forall (b : B), f (g b) = b.\\
%\\
%constr_ok :\\
%  \forall j, DepConstr(j, A) \equiv_{A \simeq B} DepConstr(j, B).\\
%\\
%elim_ok : \forall a b (P : A \rightarrow s) (Q : B \rightarrow s),\\
%  a \equiv_{A \simeq B} b \rightarrow\\
%  P \equiv_{A \simeq B} Q \rightarrow\\
%  DepElim(a, P) \equiv_{A \simeq B} DepElim(b, Q).\\
%\]
\end{minipage}
\hfill
\begin{minipage}{0.55\textwidth}
\begin{lstlisting}
elim_eta(A) : $\forall$ (a : A) (P : A $\rightarrow$ Type) $\vec{f}$, DepElim(a, P) $\vec{f}$ : P (Eta(A) a).(@\vspace{-0.04cm}@)
eta_ok(A) : $\forall$ (a : A), Eta(A) a = a.(@\vspace{-0.04cm}@)
(@\vspace{-0.14cm}@)
iota_ok(A) : $\forall$ j (P : A $\rightarrow$ Type) $\vec{f}$ $\vec{x}$(@\vspace{-0.04cm}@)
    (Q : P (Eta(A) (DepConstr(j, A) $\vec{x}$)) $\rightarrow$ Type),(@\vspace{-0.04cm}@)
  Q (DepElim(DepConstr(j, A) $\vec{x}$, P) $\vec{f}$) $\rightarrow$ (@\vspace{-0.04cm}@)
  Q (rew $\leftarrow$ eta_ok(A) (DepConstr(j, A) $\vec{x}$) (@\vspace{-0.04cm}@)
     in ($\vec{f}$[j] $\ldots$ (DepElim(IH$_0$, P) $\vec{f}$) $\ldots$ (DepElim(IH$_n$, P) $\vec{f}$) $\ldots$))(@\vspace{-0.04cm}@)
:= Iota(A, j, Q).(@\vspace{-0.04cm}@)
\end{lstlisting}
% Q (eq_rect (dep_constr_A_0 b) (fun H : A => P H) (f0 b) (eta_A (dep_constr_A_0 b)) (eq_sym (eta_OK_A (dep_constr_A_0 b)))).
\end{minipage}
\fi
\iffalse
\begin{mathpar}
\mprset{flushleft}
\small

%f := $\lambda$(a : A).DepElim(a, $\lambda$(a : A).B)(@\vspace{-0.04cm}@)
%  { $\lambda$ $\ldots$ DepConstr(0, B) $\ldots$, $\ldots$ }.(@\vspace{-0.04cm}@)
%g := $\lambda$(b : B).DepElim(b, $\lambda$(b : B).A)(@\vspace{-0.04cm}@)
%  { $\lambda$ $\ldots$ DepConstr(0, A) $\ldots$, $\ldots$ }.(@\vspace{-0.04cm}@)

\inferrule[Is-Equivalence]
 {  \Gamma \vdash \mathrm{section} : \Pi (a : A) . \mathrm{g}\ (\mathrm{f}\ a)\ =\ a \\\\ \Gamma \vdash \mathrm{retraction} : \Pi (b : B) . \mathrm{f}\ (\mathrm{g}\ b)\ =\ b }
 { \Gamma \vdash \mathrm{is\_equivalence}(\mathrm{f}, \mathrm{g}, \mathrm{section}, \mathrm{retraction}) }

\inferrule[Constr-OK]
  { \Gamma \vdash \mathrm{DepConstr}(j, A)\ \vec{x_A} \equiv_{A \simeq B} \mathrm{DepConstr}(j, B)\ \vec{x_B} }
  { \Gamma \vdash \mathrm{constr\_ok}(j) }

\inferrule[Elim-OK]
  { \Gamma \vdash a \equiv_{A \simeq B} b \\\\ \Gamma \vdash P_A \equiv_{A \simeq B} P_B \\\\ \Gamma \vdash \vec{f_A} \equiv_{A \simeq B} \vec{f_B} \\\ \Gamma \vdash \mathrm{DepElim}(a,\ P_A)\ \vec{f_A} \equiv_{A \simeq B} \mathrm{DepElim}(b,\ P_B)\ \vec{f_B}  }
  { \Gamma \vdash \mathrm{elim\_ok}(j) }

\inferrule[Elim-Eta]
  { \Gamma \vdash a : A \\ \Gamma \vdash P : \Pi (a : A) . s }
  { \Gamma \vdash \mathrm{DepElim}(a, P)\ \vec{f} : P\ (\mathrm{Eta}(A)\ a) }

\inferrule[Eta-OK]
  { \Gamma \vdash T \in \{A, B\} }
  { \Gamma \vdash \mathrm{eta\_ok}(T) : \Pi (t : T) . \mathrm{Eta}(T)\ t\ =\ t }

%iota_ok(A) : $\forall$ j (P : A $\rightarrow$ Type) $\vec{f}$ $\vec{x}$(@\vspace{-0.04cm}@)
%    (Q : P (Eta(A) (DepConstr(j, A) $\vec{x}$)) $\rightarrow$ Type),(@\vspace{-0.04cm}@)
%  Q (DepElim(DepConstr(j, A) $\vec{x}$, P) $\vec{f}$) $\rightarrow$ (@\vspace{-0.04cm}@)
%  Q (rew $\leftarrow$ eta_ok(A) (DepConstr(j, A) $\vec{x}$) in ($\vec{f}$[j] $\ldots$ (DepElim(IH$_0$, P) $\vec{f}$) $\ldots$ (DepElim(IH$_n$, P) $\vec{f}$) $\ldots$))(@\vspace{-0.04cm}@)
%:= Iota(A, j, Q).(@\vspace{-0.04cm}@)
\end{mathpar}
\fi
\vspace{-0.4cm}
\caption{Correctness criteria for a configuration to ensure that the transformation
preserves equivalence (left) coherently with equality (right, shown for \A; \B is similar). \lstinline{f} and \lstinline{g} are defined in text. $s$, $\vec{f}$, $\vec{x}$, and $\vec{\mathtt{IH}}$ represent
sorts, eliminator cases, constructor arguments, and inductive hypotheses. $\xi$ $(A,$ $P,$ $j)$ is the type 
of \lstinline{DepElim(A, P)} at \lstinline{DepConstr(j, A)} (similarly for \B).} %, respectively.}
\label{fig:spec}
\end{figure*}
 % TODO sigs for Iota here are not quite correct---Q is not bound. Also need to tweak to deal w/ eta, and to use eta_OK, and to relate both eta
% TODO (!!) Define f and g with some schema like:
% ж(A, B) := λ(a : A).DepElim(a, $\lambda$(a : A).B){ $\lambda$ ... DepConstr(0, B) ..., ... }
% ж(A, B) := λ(b : B).DepElim(b, $\lambda$(b : B).A){ $\lambda$ ... DepConstr(0, A) ..., ... }

% $\mathrm{E}_{A_i}\ (p_A : \mathrm{P}_A)$ := $\xi(A,\ p_A,\ \mathrm{Constr}(i,\ A),\ C_{A_i})$

\mysubsubsec{Specifying a Correct Transformation}
The implementation of this transformation in \toolname produces a term that Coq type checks, and so does not
add to the trusted computing base.
As \toolname is an engineering tool, there is no need to formally prove the transformation correct, though doing so would be satisfying.
The goal of such a proof would be to show that % the transformation preserves equality up to transport along the equivalence $A \simeq B$,
%while no longer referring to the old specification.
%That is, we need that 
if $\Gamma \vdash t \Uparrow t'$,
then $t$ and $t'$ are equal up to transport, and $t'$ refers to \B in place of \A.
%This is the same as the correctness criterion for the program transformation from \textsc{Devoid} that this is based on,
%with the transformation generalized to handle other equivalences beyond the class that \textsc{Devoid} supports.
The key steps in this transformation that make this possible are porting terms along the configuration % corresponding
%to a particular equivalence 
(\textsc{Dep-Constr}, \textsc{Dep-Elim}, \textsc{Eta}, and \textsc{Iota}).
%The rest is straightforward.
For metatheoretical reasons, without additional axioms, a proof of this theorem in Coq can only be approximated~\cite{tabareau2017equivalences}.
It would be possible to generate per-transformation proofs of correctness, but this does not serve an engineering need.

\subsection{Specifying Correct Configurations}
\label{sec:art}

%Both when designing a search procedure for an automatic configuration and when
%configuring \toolname manually, choosing a configuration is important,
%and it is not always straightforward.
%This section specifies what it means for a configuration to be correct. % and gives some intuition as to why.
%Section~\ref{sec:search} shows some useful example configurations.
%The configuration instantiates the proof term transformation to a particular equivalence between \A and \B.

Choosing a configuration necessarily depends in some way on the proof engineer's intentions:
there can be infinitely many equivalences that correspond to a 
change, only some of which are useful (for example~\href{https://github.com/uwplse/pumpkin-pi/blob/v2.0.0/plugin/coq/playground/refine_unit.v}{\circled{7}}, any \A is equivalent to \lstinline{unit} refined by \A). % refine_unit.v
And there can be many configurations that correspond
to an equivalence, some of which will produce terms that are more useful or efficient than others
(consider \lstinline{DepElim} converting through several intermediate types).

While we cannot control for intentions, we \textit{can} specify what it means for a chosen configuration to be correct:
Fix a configuration. Let \lstinline{f} be the function that uses \lstinline{DepElim} to eliminate \A and \lstinline{DepConstr} to construct \B (\lstinline{g} similar). %Assume a univalent metatheory in which equality up to transport is defined.
Figure~\ref{fig:spec} specifies the correctness criteria for the configuration.
These criteria relate \lstinline{DepConstr}, \lstinline{DepElim}, \lstinline{Eta}, and \lstinline{Iota}
in a way that preserves equivalence coherently with equality.

\mysubsubsec{Equivalence}
To preserve the equivalence (Figure~\ref{fig:spec}, left), \lstinline{DepConstr} and \lstinline{DepElim} must form an equivalence
% between \A and \B.
(\lstinline{section} and \lstinline{retraction} must hold for \lstinline{f} and \lstinline{g}).
%uses  \lstinline{DepElim} to eliminate \B and \lstinline{DepConstr} to construct \A.
\lstinline{DepConstr} over \A and \B must be equal up to transport across that equivalence (\lstinline{constr_ok}), 
and similarly for \lstinline{DepElim} (\lstinline{elim_ok}).
%An example proves this on the change from \lstinline{list T} to \lstinline{packed_vect T} in the 
%univalent parametricity framework.\footnote{\url{https://github.com/CoqHott/univalent_parametricity/commit/7dc14e69942e6b3302fadaf5356f9a7e724b0f3c}}
Intuitively, \lstinline{constr_ok} and \lstinline{elim_ok} guarantee that the transformation
correctly transports dependent constructors and dependent eliminators,
as doing so will preserve equality up to transport for those subterms.
This makes it possible %for the transformation
to avoid applying \lstinline{f} and \lstinline{g}, instead porting terms from \A directly to \B.

%Furthermore, since CIC$_{\omega}$ is constructive, the \textit{only} way to construct an \A (respectively \B) is to use its constructors,
%and the \textit{only} way to eliminate an \A (respectively \B) is to apply its eliminator.
%Finally, since these form an equivalence, all ways of constructing or eliminating \A and \B are covered by these dependent constructors and %eliminators.
%So, as long as we are able to unify subterms with applications of \lstinline{DepConstr} and \lstinline{DepElim},
%\textsc{Dep-Constr} and \textsc{Dep-Elim} should preserve correctness of the transformation and cover all values and eliminations of \A and \B.

\begin{figure*}
\small
\begin{grammar}
<v> $\in$ Vars, <t> $\in$ CIC$_{\omega}$

<p> ::= intro <v> \hspace{0.05cm} | \hspace{0.05cm} rewrite <t> <t> \hspace{0.05cm} | \hspace{0.05cm} symmetry \hspace{0.05cm} | \hspace{0.05cm} apply <t> \hspace{0.05cm} | \hspace{0.05cm} induction <t> <t> \{ <p>, \ldots, <p> \} \hspace{0.05cm} | \hspace{0.05cm} split \{ <p>, <p> \} \hspace{0.05cm} | \hspace{0.05cm} left \hspace{0.05cm} | \hspace{0.05cm} right \hspace{0.05cm} | \hspace{0.05cm} <p> . <p>
\end{grammar}
\vspace{-0.4cm}
\caption{Qtac syntax.}
\vspace{-0.4cm}
\label{fig:ltacsyntax1}
\end{figure*}

\begin{figure*}
\begin{mathpar}
\mprset{flushleft}
\small
\hfill\fbox{$\Gamma$ $\vdash$ $t$ $\Rightarrow$ $p$}\vspace{-0.5cm}\\

\inferrule[Intro]
  { \Gamma,\ n : T \vdash b \Rightarrow p }
  { \Gamma \vdash \lambda (n : T) . b \Rightarrow \mathrm{intro}\ n.\ p }

\inferrule[Symmetry]
  { \Gamma \vdash H \Rightarrow p }
  { \Gamma \vdash \mathtt{eq\_sym}\ H \Rightarrow \mathrm{symmetry}.\ p }

\inferrule[Split]
  { \Gamma \vdash l \Rightarrow p \\ \Gamma \vdash r \Rightarrow q }
  { \Gamma \vdash \mathrm{Constr}(0,\ \wedge)\ l\ r \Rightarrow \mathrm{split} \{ p, q \}.\ }\\

\inferrule[Left]
  { \Gamma \vdash H \Rightarrow p }
  { \Gamma \vdash \mathrm{Constr}(0,\ \vee)\ H \Rightarrow \mathrm{left}.\ p }

\inferrule[Right]
  { \Gamma \vdash H \Rightarrow p }
  { \Gamma \vdash \mathrm{Constr}(1,\ \vee)\ H \Rightarrow \mathrm{right}.\ p }

\inferrule[Rewrite]
  { \Gamma \vdash H_1 : x = y \\ \Gamma \vdash H_2 \Rightarrow p }
  { \Gamma \vdash \mathrm{Elim}(H_1,\ P) \{ x,\ H_2,\ y \} \Rightarrow \mathrm{symmetry}.\ \mathrm{rewrite}\ P\ H_1.\ p }\\

\inferrule[Induction]
  { \Gamma \vdash \vec{f} \Rightarrow \vec{p} }
  { \Gamma \vdash \mathrm{Elim}(t,\ P)\ \vec{f} \Rightarrow \mathrm{induction}\ P\ t\ \vec{p} }

\inferrule[Apply]
  { \Gamma \vdash t \Rightarrow p }
  { \Gamma \vdash f t \Rightarrow \mathrm{apply}\ f.\ p }

\inferrule[Base]
  { \\ }
  { \Gamma \vdash t \Rightarrow \mathrm{apply}\ t }
\end{mathpar}
\vspace{-0.4cm}
\caption{Qtac decompiler semantics.}
\label{fig:someantics}
\end{figure*}

\mysubsubsec{Equality}
To ensure coherence with equality (Figure~\ref{fig:spec}, right),
\lstinline{Eta} and \lstinline{Iota} must prove $\eta$ and $\iota$.
That is, \lstinline{Eta} must have the same definitional behavior as the dependent eliminator (\lstinline{elim_eta}),
and must behave like identity (\lstinline{eta_ok}).
Each \lstinline{Iota} must prove and rewrite along the simplification (\textit{refolding}~\cite{boutillier:tel-01054723}) behavior that corresponds to a case of the dependent eliminator (\lstinline{iota_ok}).
This makes it possible for the transformation to
avoid applying \lstinline{section} and \lstinline{retraction}.

\mysubsubsec{Correctness}
With these correctness criteria for a configuration, we get the completeness result (proven in Coq~\href{https://github.com/uwplse/pumpkin-pi/blob/v2.0.0/plugin/coq/playground/arbitrary.v}{\circled{8}}) that every equivalence induces a configuration. % arbitrary.v
We also obtain an algorithm for the soundness result that every configuration induces an equivalence.
The algorithm to prove \lstinline{section} is as follows (\lstinline{retraction} is similar):
replace \lstinline{a} with \lstinline{Eta(A) a} by \lstinline{eta_ok(A)}.
Then, induct using \lstinline{DepElim} over \A.
For each case $i$, the proof obligation is to show that \lstinline{g (f a)} is equal to \lstinline{a},
where \lstinline{a} is \lstinline{DepConstr(A, i)} applied to the non-inductive arguments (by \lstinline{elim_eta(A)}).
Expand the right-hand side using \lstinline{Iota(A, i)}, then expand it again using \lstinline{Iota(B, i)}
(destructing over each \lstinline{eta_ok} to apply the corresponding \lstinline{Iota}).
The result follows by definition of \lstinline{g} and \lstinline{f}, and by reflexivity.

\mysubsubsec{Automatic Configuration}
\toolname implements four search procedures for automatic configuration~\href{https://github.com/uwplse/pumpkin-pi/blob/v2.0.0/plugin/src/automation/lift/liftconfig.ml}{\circled{6}}.
Three of the four procedures are based on the search procedure from 
\textsc{Devoid}~\cite{Ringer2019},
while the remaining procedure instantiates the types \A and \B of a generic configuration that can be defined inside of Coq directly.
%Two use similar algorithms,
%due to space constraints, 
%we do not discuss these in detail.
The algorithm above is essentially what \textbf{Configure} uses to generate functions \lstinline{f} and \lstinline{g} for these configurations~\href{https://github.com/uwplse/pumpkin-pi/blob/v2.0.0/plugin/src/automation/search/search.ml}{\circled{9}}, % search.ml
and also generate proofs \lstinline{section} and \lstinline{retraction} that these functions form an equivalence~\href{https://github.com/uwplse/pumpkin-pi/blob/v2.0.0/plugin/src/automation/search/equivalence.ml}{\circled{10}}. % equivalence.ml
To minimize dependencies, \toolname does not produce proofs of \lstinline{constr_ok} and \lstinline{elim_ok} directly,
as stating these theorems cleanly would require either a special framework~\cite{tabareau2017equivalences}
or a univalent type theory~\cite{univalent2013homotopy}.
If the proof engineer wishes, it is possible to prove these in individual cases~\href{https://github.com/uwplse/pumpkin-pi/blob/v2.0.0/plugin/coq/playground/arbitrary.v}{\circled{8}}, % arbitray.v
but this is not necessary in order to use \toolname. %---they simply need to hold.

%First we need that \lstinline{DepElim} over $A$ into \lstinline{DepConstr} over $B$ and \lstinline{DepElim} over $B$ into
%\lstinline{DepConstr} over $A$ form an equivalence between $A$ and $B$. When that's true, I think it should hold that \lstinline{DepElim} over $A$
%and \lstinline{DepElim} over $B$ are in univalent relation with one another. If not, then that's an extra condition.
%Finally, we need the transformation to preserve definitional equalities. Not sure about the general case, but for vectors and lists,
%we need:

%\begin{lstlisting}
%  $\forall$ A l (f : $\forall$ (l : sigT (Vector.t A)), l = l),
%    vect_dep_elim A (fun l => l = l) (f nil) (fun t s _ => f (cons t s)) l = f (id_eta l).
%\end{lstlisting}
%and:

%\begin{lstlisting}
%Definition elim_id (A : Type) (s : {H : nat & t A H}) :=
%  vect_dep_elim
%    A
%    (fun _ => {H : nat & t A H})
%    nil
%    (fun (h : A) _ IH =>
%      cons h IH)
%    s.

% $\forall$ A h s,
%    exists (H : cons h (elim_id A s) = elim_id A (cons h s)),
%      H = eq_refl.
%\end{lstlisting}
%More generally, for each constructor index $j$, define:

%\begin{lstlisting}
%  eqc (j, B) (f : $\forall$ b : B, b = b) :=
%    fun ... (* TODO get the hypos from the type of the eliminator *) =>
%      f (DepConstr (j, B)) (* TODO args *)%%

  %elim_id := (* TODO *)
%\end{lstlisting}
%Then we need:

%\begin{enumerate}
%\item $\forall b f, \mathrm{DepElim}(b,\ p_{b}) \{\mathrm{eqc} (1, B) f, \ldots, \mathrm{eqc} (n, B) f\} = f (\mathrm{Eta}(A) a) $
%\item Something relating the constructors and \lstinline{elim_id} to reflexivity
%\end{enumerate}
%and similarly for $A$.

%Really the point of these conditions is that from them, with some restrictions on input terms, we can get
%that lifting terms gives us the same type that we'd get from lifting the type. But there are still
%some restrictions (see the few that fail).

%It's probably not always possible to define these three things for every equivalence.
%Could generalize by rewriting. But this lets us avoid the rewriting problem from Nicolas' paper.

% TODO how does this get us something like primitive projections? Just makes Eta definitionally equal to regular Id?

% TODO so we can probably just frame search in terms of DepConstr and DepElim and then generate proofs about this on an ad-hoc basis
% and get away with not including the specific details of our instantiations. We can give examples instead, give intuition, and say we generate
% the proofs in Coq

%For the second one we need not just an eliminator rule but also an identity rule.
%DEVOID assumed primitive projections which let them get away without thinking of this,
%but then had this weirdly ad-hoc ``repacking'' thing in their implementation.
%It turns out this is just a more general identity rule, which basically says what
%the identity function should lift to so that the transformation preserves definitional equalities.
%Actually deciding when to run this rule is one of the biggest challenges in practice,
%so we'll talk about that more in the implementation section.

