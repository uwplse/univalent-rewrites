\section{Implementation}
\label{sec:impl}

The configurable proof term transformation helped us build a flexible proof repair tool.
However, it alone was not enough to build a tool that reaches real proof engineers.
Similarly, the mini decompiler abstracted a lot of the details that make Ltac so useful to proof engineers---and so painful to 
reason about automatically.
This section describes a sample of the implementation challenges that we encountered and how we solved them,
both for the transformation (Section~\ref{sec:implementation}) and for the decompiler (Section~\ref{sec:decompiler}).
Section~\ref{sec:discussion} elaborates on some of the remaining challenges and our plans to address them in the future.

Interested readers may wish to consult the auxiliary material for help navigating the corresponding code.
The auxiliary material contains a guide that explains where in the implementation to find each of \lstinline{Configure},
\lstinline{Transform}, and \lstinline{Decompile}, as well as any other code referenced 
in this section or elsewhere in this paper. % TODO make this guide, do the asterisk thing

\subsection{Implementing the Transformation}
\label{sec:implementation}

\paragraph{From CIC$_{\omega}$ to Coq}

The implementation of \lstinline{Transform} must handle language differences to scale from CIC$_{\omega}$ to Coq.
We use an existing command called \lstinline{Preprocess}~\cite{Ringer2019} to turn pattern matching and fixpoints into 
applications of eliminators.
We handle non-primitive projections using \lstinline{Eta}, and we handle refolding of constants in constructors
using \lstinline{DepConstr}.

\paragraph{Termination \& Intent}

When the correctness criteria for a configuration hold and a subterm unifies with a configuration term, this suggests that \toolname \textit{can}
run the transformation rule, but it does not necessarily mean that it \textit{should}.
In some cases, repeatedly running a matching transformation rule would result in nontermination.
For example, if \B is a refinement of \A, then we can always run \textsc{Equivalence}
over and over again, forever.
%\textsc{Devoid} ruled out this case by simply prohibiting the case where \B refers to \A, but we found it sometimes
%useful to support this case.
We thus include some simple termination checks in our code. % (\href{https://github.com/uwplse/pumpkin-pi/blob/master/plugin/src/automation/lift/liftrules.ml}{liftrules.ml}).

Even when termination is guaranteed, whether to transform a subterm depends on intent.
For example, our industrial proof engineer sometimes wished to port only some occurrences of \A,
especially when \A was a tuple that could appear elsewhere
with a different meaning.
\toolname has some support for this using an interactive workflow.
%We helped the proof engineer do this by interacting with \toolname using a particular workflow.
%We plan to support this automatically using type-directed search in the future.

\paragraph{Reaching Real Proof Engineers}
Many of our design decisions in implementing \toolname were informed by our partnership with
an industrial proof engineer (see Section~\ref{sec:industry}).
For example, we found that the proof engineer rarely had the patience to wait more than ten seconds
for \toolname to port a function or proof.
In response, we implemented aggressive caching (with an option to disable the cache), even caching intermediate subterms that
we encounter in the course of running our proof term transformation.
We also added a cache to tell \toolname not to $\delta$-reduce certain terms.
%These caches are implemented in \href{https://github.com/uwplse/pumpkin-pi/blob/master/plugin/src/cache/caching.ml}{caching.ml}.
%or recurse into certain modules.
% set certain terms or modules as opaque to \toolname, to prevent unnecessary $\delta$-reduction.

The experiences of proof engineers also informed features that we exposed.
For example, we implemented special search procedures to generate custom eliminators %(\href{https://github.com/uwplse/pumpkin-pi/blob/master/plugin/src/automation/search/smartelim.ml}{smartelim.ml})
to make it easier to reason about types refined by equalities like $\Sigma$\lstinline{(l : list T).length l = n}
by breaking them into parts and reasoning separately about the projections.
These features along with our tactic decompiler helped with integration into proof engineering workflows.

\subsection{Implementing the Decompiler}
\label{sec:second}

\paragraph{Better Proof Scripts}
The mini decompiler reasons about tactics one subterm at a time, and produces simple tactic scripts.
To produce a more natural set of tactics, \textbf{Decompile} first it runs something similar to the mini decompiler, and then it modifies those tactics to produce a more natural proof script.
For example, it cancels out sequences of \lstinline{intros} and \lstinline{revert} tactics.
It removes redundant arguments to the \lstinline{apply} tactic, ensuring the result still holds. % TODO rewrite
It can also take suggested tactics (like part of the old version of the proof script) from the proof engineer as hints,
then iteratively replace tactics with those hints, checking the result as it recurses.
This makes it possible for the tactic scripts \toolname suggests to include custom tactics and decision procedures.
Further improvements could come from preserving comments and indentation, or automatically using information from the old 
version of the proof script rather than asking for it explicitly.

\paragraph{Induction and Rewriting}
The mini decompiler includes more predictable versions of \lstinline{rewrite} and \lstinline{induction}
than those found in Coq. \textbf{Decompile} includes additional logic to reason about these tactics.
For example, Qtac assumes that there is only one \lstinline{rewrite} direction. Ltac has two rewrite directions,
and so the decompiler infers the right direction based on the motive used.

Qtac also assumes that both tactics take the inductive motive explicitly.
In Coq, however, both tactics infer the motive automatically.
Consequentially, Coq will sometimes infer the wrong motive without manipulation of goals and hypotheses,
or will fail to infer a motive at all.
This is especially common for the \lstinline{rewrite} tactic, which is purely syntactic.
To handle induction, the decompiler strategically uses the \lstinline{revert} tactic to manipulate the goal
so that Coq can better infer the motive.
To handle rewrites, it uses the \lstinline{simpl} tactic to refold the goal before rewriting.
Neither of these approaches are guaranteed to work, so the proof engineer may sometimes need to tweak the output proof script appropriately.
We have found that even if we pass Coq's induction principle an explicit motive, Coq still sometimes fails due
to unrepresented assumptions.
Long term, using another tactic like \lstinline{change} to manipulate hypotheses and goals before applying these tactics
may help with cases for which Coq cannot infer the correct motive.

\paragraph{Manipulating Hypotheses}
Changing from Qtac to Ltac is not the only challenge in writing the decompiler---we also scale from CIC$_{\omega}$ to Coq.
This introduces let bindings, which are generated by tactics like \lstinline{rewrite in}, \lstinline{apply in}, and \lstinline{pose}.
\textbf{Decompile} implements support for \lstinline{rewrite in} and \lstinline{apply in} similarly to how it implements support for
\lstinline{rewrite} and \lstinline{apply}, but with three differences:

\begin{enumerate}
\item it ensures that the unmanipulated hypothesis does not occur in the body of the let expression,
\item it swaps the direction of the rewrite, and
\item it checks for generated subgoals and recurses into those subgoals.
\end{enumerate}
In all other cases, the implementation uses the \lstinline{pose} tactic, a catch-all for let bindings.

\paragraph{Pretty Printing}
After decompiling proof terms, \textbf{Decompile} pretty prints the result to the proof engineer.
Like the mini decompiler, \textbf{Decompile} represents its output language using a predefined grammar of Ltac tactics,
albeit one that is larger than Qtac.
It maintains the recursive proof structure as it goes, then uses that structure to print proofs of subgoals using bullet points.
It displays the resulting proof script to the proof engineer, who can modify it as needed.
For convenience, it includes scripts that automate the process of printing all of these tactic proofs to a Coq file,
in case the proof engineer does not want such an interactive workflow.
\toolname keeps all output proof terms from the proof term transformation in the Coq environment in case the decompiler does not succeed.
Once the proof engineer has this new proof, she can remove the old specifications, functions, and proofs, using the repaired
versions from then on.

