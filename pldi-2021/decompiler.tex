\section{Decompiling Proof Terms to Tactics}
\label{sec:decompiler}

\textbf{Transform} produces a proof term,
while the proof engineer typically writes and maintains proof scripts made up of tactics.
We improve usability thanks to the realization that, since Coq's proof term language Gallina is very structured,
we can decompile these Gallina terms to suggested Ltac proof scripts for the proof engineer to maintain.

%\begin{quote}
%\textbf{Insight 3}: The transformed proof terms can then be translated back to tactic scripts.
%\end{quote}

\textbf{Decompile} implements a prototype of this translation~\circled{11}: % decompiler.ml
it translates a proof term to a suggested proof script that attempts to prove the same theorem the same way.
Note that this problem is not well defined: while there is always a proof script that 
works (applying the proof term with the \lstinline{apply} tactic), the result is often qualitatively unreadable.
This is the baseline behavior to which the decompiler defaults.
The goal of the decompiler is to improve on that baseline as much as possible,
or else suggest a proof script that is close enough to correct that the proof engineer can
manually massage it into something that works and is maintainable.

The output language for the implementation of \textbf{Decompile} is Ltac, the proof script language for Coq.
Ltac can be confusing to reason about, since Ltac tactics can refer to Gallina terms, and the semantics of Ltac depends both on the
semantics of Gallina and on the implementation of proof search procedures written in OCaml.
To give a sense of how the decompiler works without the clutter of these details, we start by defining a mini
decompiler from CIC$_{\omega}$ to a simple subset of Ltac containing just a few predefined tactics.
Section~\ref{sec:second} explains how we scale that up to the implementation. %, and where we plan to go from there.

\mysubsubsec{A Mini Decompiler}
The mini decompiler takes CIC$_{\omega}$ terms and produces tactics in a mini version of Ltac which we call Qtac.
The syntax for Qtac is in Figure~\ref{fig:ltacsyntax1}.
Qtac includes hypothesis introduction (\lstinline{intro}),
rewriting (\lstinline{rewrite}), symmetry of equality (\lstinline{symmetry}),
application of a term to prove the goal (\lstinline{apply}), induction (\lstinline{induction}),
case splitting of conjunctions (\lstinline{split}),
constructors of disjunctions (\lstinline{left} and \lstinline{right}), and
composition (\lstinline{.}).
Unlike in Ltac, \lstinline{induction} and \lstinline{rewrite} take a motive explicitly, rather than relying on unification.
Similarly, \lstinline{apply} leaves the arguments to proof obligations.
%The implementation reasons about Ltac and so does not make these assumptions.

The semantics for the mini decompiler $\Gamma \vdash t \Rightarrow p$ are in Figure~\ref{fig:someantics} (assuming $=$, \lstinline{eq_sym}, $\wedge$, and $\vee$ are defined as in Coq).
As with the real decompiler, the mini decompiler defaults to the proof script
that applies the entire proof term with \lstinline{apply} (\textsc{Base}).
Otherwise, it improves on that behavior by recursing over the proof term and constructing a proof script using a predefined set of tactics.

For the mini decompiler, this is straightforward: Lambda terms become introduction 	(\textsc{Intro}), since they introduce new bindings
in the environment of the body. Applications of \lstinline{eq_sym} become symmetry of equality (\textsc{Symmetry}).
Constructors of conjunction and disjunction map to the respective tactics (\textsc{Split}, \textsc{Left}, and \textsc{Right}).
Applications of equality eliminators compose symmetry (to orient the rewrite direction with the goal) with rewrites (\textsc{Rewrite}),
and all other applications of eliminators become induction (\textsc{Induction}).
The remaining applications become apply tactics (\textsc{Apply}).
In all cases, the decompiler recurses, breaking into cases when relevant, until only the \textsc{Base}
case holds. % at which point we are done.

While the mini decompiler is very simple, only a few small changes are needed
to move this from CIC$_{\omega}$ to Coq.
The result can already handle some of the example proofs \toolname has produced.
The generated proof term of \lstinline{rev_app_distr} from Section~\ref{sec:overview},
for example, consists only of induction, rewriting, simplification, and reflexivity (solved by \lstinline{auto}).
Figure~\ref{fig:rainbow} shows the proof term for the base case of \lstinline{rev_app_distr} 
alongside the decompiled tactic script that \toolname suggests.
This script is fairly low-level and close to the proof term, but it is already something that the proof engineer
can step through piece by piece to understand, modify, and maintain.
There are very few differences from the mini decompiler needed to produce this,
for example handling of rewrites in both directions (\lstinline{eq_ind_r} as opposed to \lstinline{eq_ind}),
simplifying to handle motive inference for rewrites,
and turning applications of \lstinline{eq_refl} into \lstinline{reflexivity} or \lstinline{auto}.

\mysubsubsec{Better Proof Scripts}
The mini decompiler produces simple tactic scripts.
The implementation of \textbf{Decompile} first runs something similar to the mini decompiler, and then modifies those tactics to produce a more natural proof script~\circled{11}. % decompiler.ml
For example, it cancels out sequences of \lstinline{intros} and \lstinline{revert} tactics,
inserts semicolons, and removes redundant arguments to the \lstinline{apply} tactic. %, ensuring the result still holds. % TODO rewrite
It can also take suggested tactics (like part of the old version of the proof script) from the proof engineer as hints,
then iteratively replace tactics with those hints, checking the result as it recurses.
This makes it possible for the tactic scripts \toolname suggests to include custom tactics and decision procedures.
%Further improvements could come from preserving comments and indentation, or automatically using information from the old 
%version of the proof script rather than asking for it explicitly.

%In fact, since \toolname uses an existing command to translate pattern matching and fixpoints to eliminators,
%\textit{all} of the proof terms that \toolname produces will use induction and rewriting instead.
%Because we have control over output terms, even a mini decompiler gets us pretty far.

% TODO add any new things RanDair implements, like exists



